Automatisches Schlussfolgern spielt eine zentrale Rolle für
Beschreibungslogiken. Insbesondere ist die Ausdrucksstärke von
Beschreibungslogiken stark darauf zugeschnitten.
Dabei ist aber wichtig, dass die relevanten Schlussfolgerungsprobleme
entscheidbar sind, sie ein möglichst geringe Komplexität haben und
Algorithmen existieren, die sich in der Praxis performant verhalten.
Von uns wird hauptsächlich das Problem der Erfüllbarkeit betrachtet.

In der Praxis haben sich hauptsächlich Tableau-Algorithmen und
Resolutionsverfahren als effizient herausgestellt.

\subsection{\texorpdfstring{$\ALC$}{ALC} ohne TBoxen}\label{alc-ohne-tboxen}

\subsubsection{Negationsnormalform}\label{negationsnormalform}

\begin{definition}[Negationsnormalform]
Ein Konzept ist in \emph{Negationsnormalform} (NNF) gdw. Negation nur auf
Konzeptnamen (also nicht auf zusammengesetzte Konzepte) angewendet wird.
\end{definition}

\begin{lemma}[Linearzeit Umwandlung in NNF]
Jedes Konzept kann in Linearzeit in ein äquivalentes Konzept in
Negationsnormalform umgewandelt werden.
\end{lemma}

\begin{tafel}[Beweis von Linearzeit Umwandlung in NNF]

    \begin{proof}
        Jedes Konzept kann in NNF umgewandelt werden durch erschöpfendes Anwenden von folgenden Regeln
        \begin{itemize}
            \item doppelte Negation: $\neg \neg C \rightsquigarrow C$
            \item de Morgan: \begin{align*}
                    \neg (C \sqcap D) \rightsquigarrow \neg C \sqcup \neg D\\
                    \neg (C \sqcup D) \rightsquigarrow \neg C \sqcap \neg D
                \end{align*}
            \item Dualität: \begin{align*}
                    \neg \exists r.C \rightsquigarrow \forall r.\neg C\\
                    \neg \forall r.C \rightsquigarrow \exists r.\neg C
                \end{align*}
        \end{itemize}
        Jede Regelanwendung ist äquivalenzerhaltend und schiebt Negation immer
        weiter nach innen. Man kann zeigen, dass linear viele Regelanwendungen
        ausreichen.
    \end{proof}
\end{tafel}


\subsubsection{I-Baum}\label{i-baum}

I-Bäume bilde eine Datenstruktur, die ein (partielles) Baummodell
repräsentieren.

\begin{definition}[I-Baum]

Ein \emph{I-Baum} für $C_{0}$ (in NNF) ist knoten- und kantenbeschrifteter
Baum $\left( V,E,\ML \right)$ mit

\begin{itemize}
\item $V$ Knotenmenge
\item $E$ ist Menge beschrifteter Kanten $\left( v,r,v^{'} \right)$ mit
  $v$,$\ v^{'} \in V$, $r$ Rollenname
\item $\ML$: $V \rightarrow 2^{sub(C_{0})}$ ist eine Knotenbeschriftung aus den Teilkonzepten von $C_0$.
\end{itemize}
\end{definition}

\begin{tafel}[Beispiel I-Baum]
    Beispiel:
    \begin{align*}
        C_0 &= A \sqcap \forall r.(\neg A \sqcap \exists r.B)\\
        \sub(C_0) &= \{A, \neg A, B, \exists r.B, \neg A \sqcap \exists r.B, \forall r.(\neg A \sqcap \exists r.B), C_0\}
    \end{align*}

    \includegraphics[width=3.71910in,height=1.83200in]{media/42ibaum.png}
\end{tafel}

\subsubsection{Tableau-Algorithmus}\label{tableau-algorithmus}

Der Tableau-Algorithmus berechnet eine Folge $$M_0,M_1,\ldots$$ von Mengen von I-Bäumen:
\begin{itemize}
    \item $M_0 = \left\{ B_{\text{ini}} \right\}$ wobei $B_{\text{ini}}$ der \emph{initiale I-Baum} für $C_0$ ist:
        \begin{align*}
            V &:= \left\{ v_{\text{ini}} \right\}\\
            E &:= \emptyset\\
            \ML\left(v_{\text{ini}}\right) &:= \left\{ C_0 \right\}
        \end{align*}
    \item $M_{i + 1}$ entsteht aus $M_i$ durch Anwendung einer Tableau-Regel
        auf einen der I-Bäume in $M_i$, die einen oder mehrere neue I-Bäume
        erzeugt, die den Baum ersetzen.
\end{itemize}

Sei $\left( V,E,\ML \right)$ ein I-Baum.

\begin{itemize}
\item $\sqcap$-Regel
  \begin{itemize}
  \item
    Wähle $v \in V$ und $C \sqcap D \in \ML\left( v \right)$ so dass
    $\left\{ C,D \right\}\  \not\subseteq \ML\left( v \right)$
  \item
    erweitere $\ML(v)$ um $C$ und $D$
  \end{itemize}
\item $\sqcup$-Regel

  \begin{itemize}
  \item
    Wähle $v \in V$ und $C \sqcup D \in \ML\left( v \right)$ so dass
    $\left\{ C,D \right\} \cap \ML\left( v \right) = \emptyset$
  \item
      erweitere $L(v)$ um $C$ \emph{oder} um $D$ (ergibt \emph{zwei} I-Bäume)
  \end{itemize}
\item $\exists$-Regel

  \begin{itemize}
  \item
    Wähle $v \in V$ und $\exists r.C \in \ML\left( v \right)$ so dass
    es kein $v' \in V$ gibt mit $\left( v,r,v' \right) \in E$
    und $C \in \ML\left( v' \right)$
  \item
    erweitere V um neuen Konten $v'$ und $E$ um
    $\left( v,r,v' \right)$, setze
    $\ML\left( v' \right) = \left\{ C \right\}$
  \end{itemize}
\item $\forall$-Regel
  \begin{itemize}
  \item
    Wähle $v,v' \in V$ und $\forall r.C \in \ML\left( v \right)$ so
    dass $\left( v,r,v' \right) \in E$ und
    $C\  \notin \ML\left( v \right)$
  \item
    erweitere $\ML(v')$ um $C$
  \end{itemize}
\end{itemize}

Berechnung von $M_{i + 1}$ aus $M_i$ erfolgt durch Auswahl eines $B \in M$ und
Anwendung einer der vier Regeln. Eine Regelanwendung ersetzt $B$ durch einen
neuen I-Baum bzw.\ zwei neue I-Bäume ($\sqcup$-Regel). Ein I-Baum ist \emph{vollständig}, wenn keine Regel darauf anwendbar ist. Intuitiv machen Regelanwendungen implizites Wissen explizit.

Der Algorithmus stoppt, wenn alle I-Bäume vollständig sind. Anschließend wird
\begin{itemize}
    \item \enquote{erfüllbar} zurückgegeben, wenn ein I-Baum gefunden wurde, der keinen \emph{offensichtlichen Widerspruch} enthält. Ein offensichtlicher Widerspruch liegt vor, wenn es einen Knoten $v$ gibt mit $\left\{A, \neg A\right\} \subseteq \ML(v)$ für einen Konzeptnamen $A$.
    \item \enquote{unerfüllbar} andernfalls, also wenn es in \emph{allen} vollständigen I-Bäumen einen offensichtlichen Widerspruch gibt.
\end{itemize}

\begin{tafel}[Beispielanwendung des Tableau-Algorithmus]

Beispiel:

\includegraphics[width=5.71910in,height=3.83200in]{media/43taleau.png}

\end{tafel}

Wir müssen nun zeigen, dass der Algorithmus
\begin{enumerate}
    \item terminiert:
        Nach endlicher Zeit sind alle I-Bäume vollständig und der Algorithmus kann stoppen und eine Antwort (\enquote{erfüllbar}/\enquote{unerfüllbar}) zurückgeben.
    \item korrekt(\enquote{sound}) ist:
        Wenn der Algorithmus \enquote{erfüllbar} zurückgibt, dann ist das Eingabekonzept auch tatsächlich Erfüllbar.
    \item vollständig(\enquote{complete}) ist:
        Wenn das Eingabekonzept erfüllbar ist, dann gibt der Algorithmus auch \enquote{erfüllbar} zurück.
\end{enumerate}

Wir beginnen mit Terminierung.

\subsubsection{Rollentiefe}
\label{sec:def-rollentiefe}

Die Rollentiefe $\rd\left(C\right)$ von Konzepten
$C \in \sub\left( C_{0} \right)$ ist induktiv definiert:

\begin{itemize}
\item
  $\rd\left( A \right) = \rd\left( \neg A \right) = 0$
\item
  $\rd\left( C \sqcap D \right) = \rd\left( C \sqcup D \right) = \max\left( \rd\left( C \right),\rd\left( D \right) \right)$
\item
  $\rd\left( \exists r.C \right) = \rd\left( \forall r.C \right) = 1 + \rd\left( C \right)$
\end{itemize}

\begin{lemma}
Für alle $C \in \sub\left( C_{0} \right)$ gilt
$\rd\left( C \right) \leq \left| C \right|$.
\end{lemma}

\subsubsection{Multimengen}\label{multimengen}

Multimengen sind Mengen, in denen Elemente mehrfach vorkommen dürfen, z.B.:
$$\{1,1,2,3,4,4,5,6,6,6\}$$

Eine Multimengen über einer Menge $S$ ist eine Abbildung
$$M:S\rightarrow\mathbb{N}$$, die die Anzahl des Vorkommen der Elemente aus
$S$ beschreibt.  Die meisten Begriffe übertragen sich von Mengen auf
Multimengen:

\begin{itemize}
	\item Leere Menge $\emptyset: s \mapsto 0$ für alle $s \in S$
	\item Vereinigung: $(M_1 \cup M_2)(s) := M_1(s) + M_2(s)$ 
	\item Element: $s \in M\ gdw.\ M(s)>0$
    \item Differenz: $(M_1 \setminus M_2)(s) =
        \begin{cases}
                M_1(s) - M_2(s) & \text{wenn } M_1(s) \geq M_2(s)\\
                0 & \text{sonst}
        \end{cases}$
\end{itemize}

$MM(S)$ bezeichnet die Menge aller Multimengen über der Menge S.

Gegeben eine strikte partielle Ordnung $\left( S, < \right)$, ist die
\emph{Multimengenerweiterung} $\left( \text{MM}\left( S \right), <_{\text{mul}} \right)$ definiert als: 

$M_2 <_{\text{mul}}M_1$ gdw. $\exists X,Y \in MM(S)$, so dass

\begin{itemize}
\item
  $\emptyset \neq X \subseteq M_1$
\item
  $M_2 = \left( M_1 \setminus X \right) \cup Y$
\item
    Für alle $y \in Y$ gibt es ein $x \in X$ mit $x > y$.
\end{itemize}

Also erhält man $M_2$ aus $M_1$, indem man einige Elemente entfernt und durch endlich viele \emph{kleinere} ersetzt.
Beispiel:
\begin{align*}
\left\{ 3,1 \right\} >_{\text{mul}}\{ 2,2,2\} >_{\text{mul}}\{ 2,2\} >_{\text{mul}}\{ 2,1,1,1\}
\end{align*}

\setcounter{definition}{5}
\begin{lemma}
    Auch $(MM(S), <_{\text{mul}}$ ist eine strikte partielle Ordnung.
\end{lemma}

Eine partielle Ordnung $<$ heißt wohlfundiert, wenn $<$ keine unendlichen absteigenden Ketten hat. Zum Beispiel ist $(\mathbb{N}, <)$ wohlfundiert, aber $(\mathbb{Z}, <)$ und $(\left[0, 1\right]_{\mathbb{R}})$ nicht.

\begin{theorem}[Wohlfundiertheit von $(MM(S), <_{\text{mul}}$]
Wenn $\left( S, < \right)$ wohlfundiert ist, dann ist auch $\left( \text{MM}\left( S \right), <_{\text{mul}} \right)$
wohlfundiert.
\end{theorem}


\subsubsection{Terminierung}

\setcounter{definition}{4}
\begin{theorem}[Terminierung des Tableau-Algorithmus]
Der Tableau-Algorithmus stoppt nach endlicher Zeit.
\end{theorem}
\setcounter{definition}{7}

Beweis in 4 Schritten:
\begin{itemize}
    \item Behauptung 1.
  Es werden nur I-Bäume mit einem Verzweigungsgrad
  von maximal $|C_0|$ generiert.
  \begin{tafel} \begin{proof}
        Nur die $\exists$-Regel generiert Nachfolgeknoten, aber
        höchstens einen pro Konzept $\exists r.C \in \sub(C_0)$. Nach
        \autoref{lem:num-sub}  enthält
        $\sub\left( C_{o} \right)$ höchstens $\left| C_{0} \right|$
        viele Teilkonzepte.
    \end{proof}\end{tafel}
\item Behauptung 2.
  Es werden nur I-Bäume mit einer Tiefe von maximal $\left| C_{0} \right|$
  generiert.
    \begin{tafel} \begin{proof}
        Hilfsaussage: Wenn $v$ Knoten mit Tiefe $i$ ist, dann $\rd(C) \leq \rd(C_0) - i$ für alle $C \in \ML(v)$. Daraus folgt Behauptung 2, denn jeder Knoten ist mit mindestens einem Konzept beschriftet. Wenn es einen Knoten in Tiefe $k > \rd(C_0)$ gäbe, dann gäbe es auch $C\in \ML(v)$ und wegen der Hilfsaussage wäre die Rollentiefe negativ, was per Definition der Rollentiefe unmöglich ist.

        Beweis der Hilfsaussage per Induktion über Anzahl an Regelanwendungen:

        \textbf{I.A.} Es gibt nur den Knoten $v_{\text{ini}}$ mit
$L\left( v_{\text{ini}} \right) = \left\{ C_{0} \right\}$. I.V. gilt,
da $i = 0$.

\textbf{I.S.} Fallunterscheidung nach angewandter Regel (exemplarisch
$\sqcap$, $\exists$):

\begin{enumerate}
\item $\sqcap$-Regel. Dann $C \sqcap D \in L(v)$ und $L(v)$ wird um $C$ und $D$ erweitert.
Nach I.V. gilt: $\rd\left( C \sqcap D \right) \leq \rd\left( C_{0} \right) - i$,
also auch $\rd\left( C \right) \leq rd\left( C_{0} \right) - i$,
weil $\rd\left( C \right) \leq \rd\left( C \sqcap D \right)$.
Analog für $D$.

\item $\exists$-Regel. Dann $\exists r.C \in L\left( v \right)$ und es wird
    ein neuer Knoten $v'$ auf Tiefe $i + 1$ generiert mit
$\ML\left( v' \right) = \left\{ C \right\}$. Es gilt
\begin{align*}
    \rd\left( C \right) &= \rd\left( \exists r.C \right) - 1 & \text{(Def rd.)}\\
                        &\leq \rd\left( C_{0} \right) - i - 1 & \text{(IV)}\\
                        &= \rd\left( C_{0} \right) - (i + 1)
\end{align*} \end{enumerate} \end{proof}
\end{tafel}

\item Behauptung 3.
  Sei $M_{0},M_1,\ldots$ die erzeugte Folge und $B \in M_i$ für
  ein $i \geq 0$. Dann ist $B$ durch die Anwendung von maximal
  \begin{align*}
      \underbrace{\left| C_{0} \right|^{\left| C_{0} \right|}}_\text{\# Knoten im Baum} \cdot \underbrace{\left| C_{0} \right|}_\text{Größe Knotenbeschriftungen} \leq 2^{2\left| C_{0} \right|^{2}} =: n
  \end{align*}
  Regeln entstanden. (folgt kombinatorisch aus Behauptung 1 und 2)

\end{itemize}

Nun kann die Terminierung mittels Behauptung 3 beweisen werden:

\begin{tafel}[Beweis Terminerung Tableau-Algorithmus]

\begin{proof}
    Wir ordnen jedem $M_i$ eine Multimenge $MM_i$ zu. Für jeden Baum $B \in
    M_i$ enthält $MM_i$ die Zahl der Regelanwendungen, mittels derer B
    generiert wurde.
    Weil $<$ auf $\mathbb{N}$ wohldefiniert ist, ist $<_{\text{mul}}$ auf $MM(\{0,\ldots,n\})$ wohldefiniert.

    Offenbar gilt $MM_{i+1} <_{\text{mul}} MM_i$ für alle $i \geq 0$. Also ist die Folge der $MM_i$ endlich.
\end{proof}

\includegraphics[width=3.71910in,height=1.83200in]{media/45mm.png}
\end{tafel}

\subsubsection{Korrektheit und Vollständigkeit}

\begin{theorem}
    Wenn der Tableau-Algorithmus \enquote{erfüllbar} zurückgibt, dann ist $C_{0}$
erfüllbar.
\end{theorem}

\begin{proof}
    Wenn der Algorithmus \enquote{erfüllbar} zurückgibt, so hat er einen
    \emph{vollständigen} I-Baum $B = (V, E, \ML)$ \emph{ohne offensichtlichen
    Widerspruch} gefunden. Aus $B$ konstruieren wir Interpretation $\MI$ wie
    folgt:
    \begin{align*}
        \Delta^\MI &= V\\
        r^\MI &= \left\{(v, v') \middle| (v, r, v') \in E\right\} & \text{für alle Rollennamen $r$}\\
        A^\MI &= \left\{v \middle| A \in \ML(v)\right\} & \text{für alle Konzeptnamen $A$}
    \end{align*}
    Die Erfüllbarkeit von $C_0$ folgt dann aus:

    \emph{Behauptung}: Für alle Konzepte $C$ und $v \in V$ gilt

    \begin{center}$C \in L\left( v \right)$ impliziert $v \in C^\MI$\end{center}
\end{proof}

\begin{tafel}[Beweis der Behauptung für Korrektheit]
        Die Behauptung impliziert Erfüllbarkeit von $C_0$. Da $C_0 \in \ML(v_{ini})$ in $B_{ini}$ ist, gilt auch $C_0 \in \ML(v_{ini})$ in $B$. Knotenbeschriftungen werden nie gelöscht. Also ist $v_{ini} \in C_0^\MI$ nach der Behauptung, folglich ist $C_0$ erfüllbar.
    \begin{proof}
        Beweis der Behauptung per Induktion über Struktur von $C$.
        
        \textbf{I.A.} $C = A$ (Konzeptname) Gilt nach Definition von
        $\MI$.
        
        \textbf{I.S.}

\begin{itemize}
\item $C = \neg A$

$A$ ist ein Konzeptname. Da $B$ keinen offensichtlichen Widerspruch enthält,
folgt aus $\neg A \in \ML(v)$, dass i $A \notin \ML(v)$. Nach Definition von
$\MI$ gilt $v \notin A^\MI$. Also $v \in \left( \neg A \right)^\MI$.

\item $C = D \sqcap E$
    \begin{align*}
        &C \in \ML(v)\\
        &\implies (\sqcap\text{-Regel nicht anwendbar}) D \in \ML(v), E \in \ML(v)\\
        &\implies (\text{I.V.}) v \in D^\MI,\ v \in E^\MI\\
        &\implies (\text{Semantik}) v \in \left( D \sqcap E \right)^\MI
    \end{align*}

\item $C = D \sqcup E$

analog zu $\sqcap$.

\item $C = \exists r.D$

Da die $\exists$-Regel nicht anwendbar ist, gibt es $v'\in V$ mit $(v,r,v')
\in E$ und $D \in \ML(v')$

Nach I.V. gilt $v' \in D^\MI$; nach Konstruktion $(v,v') \in r^\MI$, woraus
$v \in (\exists r.D^\MI)$ folgt.

\item $C = \forall r.D$

analog zu $\exists$.
\end{itemize}
    \end{proof}
\end{tafel}

\begin{definition}[Realisierbarkeit]

Sei $B = \left( V,E,\ML \right)$ ein I-Baum und $\MI$ eine Interpretation. $\MI$
\emph{realisiert} $B$, wenn es eine Funktion
\begin{align*}
    \pi : V \rightarrow \Delta^\MI
\end{align*}
gibt, so dass gilt:
\begin{itemize}
\item
  $\left( v,r,v^{'} \right) \in E$ impliziert
  $\left( \pi\left( v \right),\pi\left( v^{'} \right) \right) \in r^\MI$
\item
  $C \in L\left( v \right)$ impliziert
  $\pi\left( v \right) \in C^\MI$
\end{itemize}

$B$ ist \emph{realisierbar}, wenn es Interpretation $\MI$ gibt, die $B$
realisiert. Eine Menge $M$ von I-Bäumen ist \emph{realisierbar} gdw. ein
$B \in M$ realisierbar.
\end{definition}

Beachte: ein realisierbarer I-Baum enthält keinen offensichtlichen Widerspruch.

\begin{tafel}[Beispiel Realisierbarkeit]\mbox{}\\

\includegraphics[width=2.0in,height=4.0in]{media/47real.png}
\end{tafel}

\begin{theorem}[Vollständigkeit]
    Wenn $C_{0}$ erfüllbar ist, so gibt der Tableau-Algorithmus \enquote{erfüllbar}
zurück.
\end{theorem}

\textbf{T4.8}

\begin{proof}

Per Induktion über $\MI$. 

Sei $C_{0}$ erfüllbar. Nach \protect\hyperlink{proposition-4.5-terminierung}{Proposition 4.5} berechnet der Algorithmus endlich Folge $M_{0},\ldots,M_{n}$. Wir
zeigen: 

$M_{i}$ ist realisierbar für alle $0 \leq i \leq n$. 

Daraus folgt: Es gibt realisierbaren Baum $B \in M_{n}$ und damit enthält $B$ keinen offensichtlichen Widerspruch. Also gibt der Algorithmus „erfüllbar`` zurück.

\textbf{I.A.} $i = 0$. $M_{0} = {\{ B}_{\text{ini}}\}$.
$B_{\text{ini}}$ ist realisierbar, weil $C_{0}$ erfüllbar.

\textbf{I.S.} Fallunterscheidung gemäß der Regel, mit der $M_{i + 1}$
aus $M_\MI$ erzeugt wurde. Sei $B$ realisierbarer Baum aus
$M_i$, auf welchen Regel angewandt wird. Beispielhaft
$\sqcup$-Regel:

\begin{enumerate}
\def\labelenumi{\arabic{enumi}.}
\item
  $\sqcup$-Regel
\end{enumerate}

\begin{quote}
Dann wird $B = \left( V,E,L \right)$ ersetzt durch
$B^{'} = \left( V,E,L^{'} \right) \in M_{i + 1}$ und
$B^{''} = \left( V,E,L^{''} \right) \in M_{i + 1}$ und es gibt
$v \in V$ mit

\begin{itemize}
\item
  $\left( C \sqcup D \right) \in L(v)$
\item
  $L^{'}\left( v \right) = L\left( v \right) \cup \left\{ C \right\}$,
  $L^{''}\left( v \right) = L\left( v \right) \cup \left\{ D \right\}$
\item
  $L^{'}\left( u \right) = L^{''}\left( u \right) = L\left( u \right)$
  für alle $u \neq v$
\end{itemize}

Es genügt zu zeigen, dass wenn $B$ realisiertbar, dann $B'$ oder
$B''$ realisierbar. 

Sei $\MI$ Interpretation, die $B$ realisiert und $\pi\ :V \rightarrow \Delta^\MI$ Abbildung wie in \protect\hyperlink{realisierbarkeit}{Definition 4.8}. Dann gilt $\pi\left( v \right) \in \left( C \sqcup D \right)^\MI$. Nach Semantik: $\pi\left( v \right) \in C^\MI$ oder $\pi\left( v \right) \in D^\MI$. Also realisiert $\MI$ den Baum $B'$ oder $B''$.
\end{quote}
\end{proof}

\subsubsection{Komplexitätsanalyse}\label{praktikabilituxe4t}

Wir beobachten: 

I-Bäume können höchstens exponentiell groß werden.

Dieser Fall kann tastsächlich eintreffen. Beispielhaft, der Erfüllbarkeitstest von:
$$\bigsqcap_{i < n} \forall r^i .(\exists r.B \sqcap \exists r.\neg B)$$
generiert Baum der Größe $2^n$.

Also: exponentieller Zeit- und Platzverbrauch (sogar 2-exponentiell)

\subsubsection{Praktikabilität}

Offenbar wäre eine naive Implementierung nicht effizient. Dabei kann man aber einige Hinweise/Optimierungen bei der Implementierung beachten:

\begin{itemize}
	\item Es wird nur ein Baum zur Zeit generiert, keine Menge
	\item bei der $\sqcup$-Regel muss man sich also entscheiden (Heuristik); ggf. Entscheidung revidiieren (Backtracking).
	\item Es wird nur ein Teil des Baumes (Pfad) im Speicher gehalten.
    \item Backjumping: Führe Buch über die \enquote{Herkunft} von Knotenbeschriftungen und Kanten mittels Dependenzmengen. Wenn Backtracking nötig, springe direkt zu einer der Ursachen des Widerspruches zurück.
\end{itemize}

\subsection{ALC mit generellen TBoxen}\label{alc-mit-generellen-tboxen}

Nun wollen wir einen Tableau-Algorithmus für die Erfüllbarkeit in $\ALC$  \emph{bzgl. TBoxen}.

Jede TBox $\MT$ ist äquivalent zu einer TBox der Form
$\left\{ \top \sqsubseteq C_{\MT} \right\}$: 

\begin{center}setze $C_{\MT} := \prod_{C \sqsubseteq D \in \MT}^{}{\neg C \sqcup D}$.\end{center}

\textbf{T4.11} Beispiel

$$\MT = \{A \sqsubseteq \exists r.B, A \sqcup B \sqsubseteq \forall r.B\}$$

Daraus wird

$$\{\top \sqsubseteq (\neg A \sqcup \exists r.B) \sqcap (\neg (A \sqcup B) \sqcup \forall r.B)\}$$

in NNF:

$$\MT' = \{\top \sqsubseteq (\neg A \sqcup \exists r.B) \sqcap ((\neg A \sqcap \neg B) \sqcup \forall r.B)\}$$

Desweiteren nehmen wir an, dass:

\begin{itemize}
	\item Eingabe $C_0$ in NNF;
	\item Eingabe $\MT$ hat Form $\{\top \sqsubseteq C_{\MT}\}$ mit $C_{\MT}$ in NNF
\end{itemize}

Nun modifiziere den vorigen Algorithmus durch Hinzufügen folgender Regel:

\subsubsection{TBox-Regel}\label{tbox-regel}

Wähle $v \in V$ so dass $C_{\MT} \notin L\left( V \right)$ und
erweitere $L\left( v \right)$ um $C_{\MT}$.

Problem: Terminiert nicht!

\textbf{T4.12}

\includegraphics[width=3.71910in,height=1.83200in]{media/412endl.png}

\subsubsection{Blockieren}\label{blockieren}

Dies lösen wir, indem wir nur ein endliches Anfangsstück eines Baummodells anhand dessen sich die Existenzeines vollständigen Modells entscheiden lässt konstruieren. Dazu müssen wir die Anwendung der $\exists$-Regel einschränken.

\begin{definition}[Blockiert]

Sei $\left( V,E,L \right)$ ein I-Baum und $u,v \in V$. Dann ist
$v$ direkt blockiert durch $u$, wenn

\begin{enumerate}
\def\labelenumi{\arabic{enumi}.}
\item
  $u$ Vorgänger von $v$ in $B$ ist und
\item
  $L\left( v \right) \subseteq L\left( u \right)$
\end{enumerate}

$v$ ist blockiert, wenn $v$ direkt blockiert ist oder einen direkt
blockierten Vorgänger hat.
\end{definition}

\subsubsection{\texorpdfstring{Neue $\exists$-Regel
($\exists^{'}$-Regel)}{Neue \textbackslash{}exists-Regel (\textbackslash{}exists\^{}\{'\}-Regel)}}\label{neue-exists-regel-exists-regel}

\begin{itemize}
\item
  Wähle $v \in V$ und $\exists r.C \in L\left( v \right)$ so dass
  $v$ \emph{nicht blockiert ist und} es kein $v^{'} \in V$ gibt mit
  $\left( v,r,v^{'} \right) \in E$ und $C\  \in L\left( v' \right)$
\item
  erweitere V um neuen Konten $v^{'}$ und $E$ um
  $\left( v,r,v^{'} \right)$; setze
  $L\left( v^{'} \right) = \left\{ C \right\}$
\end{itemize}

\includegraphics[width=5.71910in,height=2.33200in]{media/414block.png}

\subsubsection{Vollständigkeit}\label{vollstuxe4ndigkeit}

\begin{proposition}
Wenn $C_0$ erfüllbar bzgl. $\MT$, so gibt der Algorithmus erfüllbar zurück.
\end{proposition}

Beweis wie ohne TBoxen: Alle $M_{0},\ldots,\ M_{n}$ sind realisierbar
bzgl. $\MI$ (Induktion), also enthält $M_{n}$ einen Baum ohne
offensichtlichen Widerspruch (Nur neue Fallunterscheidung für TBox-Regel und Realisierbarkeitsbegriff auf TBoxen erweitert).

\subsubsection{Korrektheit}\label{korrektheit}

\begin{proposition}
    Wenn der Algorithmus \enquote{erfüllbar} zurückgibt, so ist $C_0$ erfüllbar bzgl. $\MT$
\end{proposition}

\textbf{T4.15}

Beweisskizze per Induktion über die Struktur von $C$. Definiere
Interpretation $\MI$:

\begin{itemize}
\item
  $\Delta^\MI = \left\{ v \in V \middle| \text{v\ }\mathrm{\text{nicht\ blockiert}} \right\}$
\item
  $r^\MI = \left\{ \left( v,v^{'} \right)\  \middle| \ \left( v,r,v^{'} \right) \in E \right\} \cup \left\{ \left( v,u \right)\  \middle| \ \exists\left( v,r,v^{'} \right) \in E\mathrm{\ }\mathrm{\text{und}}\mathrm{\ }v^{'}\mathrm{\ }\mathrm{\text{direkt\ blockiert\ durch}}\mathrm{\ }u \right\}$
\item
  $A^\MI = \left\{ v \middle| A \in L\left( v \right) \right\}$
\end{itemize}

Behauptung: Für alle ALC-Konzepte $C$ und $v \in \Delta^\MI$ gilt:
$$C \in L\left( v \right) \Rightarrow v \in C^\MI$$
Die Behauptung impliziert wie gewünscht, dass

\begin{itemize}
\item
  $\MI$ Modell von $\MT$ ist.

Da die TBox-Regel nicht anwendbar ist, gilt $C_{\MT} \in L\left( v \right)$ für alle $v \in V$. Also $v \in C_{\MT}^\MI$ für alle $v \in \Delta^\MI$.

\item
  $\MI$ Modell von $C_{0}$ ist.

Da $C_{0} \in L\left( v_{\text{ini}} \right)$ gilt nach Behauptung
$v_{\text{ini}} \in C_{0}^\MI$.
\end{itemize}

\textbf{I.A.} Siehe Beweis zu
\protect\hyperlink{proposition-4.7-korrektheit}{Proposition 4.7}.

\textbf{I.S.} Schritte wie in Beweis zu Proposition 4.7, außer:

\begin{itemize}
\item
  $C = \exists r.D$
\end{itemize}

Sei $\exists r.D \in L\left( v \right)$. Da die $\exists'$-Regel
nicht anwendbar ist, gibt es $v^{'} \in V$ mit
$\left( v,r,v^{'} \right) \in E$ und $D \in L\left( v \right)$.
Fallunterscheidung:

\begin{enumerate}
\def\labelenumi{\arabic{enumi}.}
\item
  $v'$ unblockiert. Dann $\left( v,v^{'} \right) \in r^\MI$
  (Definition $\MI$), $v^{'} \in D^\MI$ (\textbf{I.V.})
  $\Rightarrow v \in \left( \exists r.D \right)^\MI$
\item
  $v^{'}$ blockiert. Da der direkte Vorgänger $v$ von $v'$
  unblockiert ist, ist $v'$ direkt blockiert von unblockiertem
  Vorgänger $u$. Es gilt:
\end{enumerate}

\begin{itemize}
\item
  $\left( v,u \right) \in r^\MI$ nach Definition $r^\MI$
\item
  $D \in L\left( v \right) \subseteq L\left( u \right)$
  (Blockierungsbedingung)
\item
  $\Rightarrow u \in D^\MI$ (\textbf{I.V.})
\end{itemize}

\begin{quote}
Also $v \in \left( \exists r.D \right)^\MI$.
\end{quote}

\begin{itemize}
\item
  $C = \forall r.D$
\end{itemize}

Ähnlich zu oberem Fall.

\subsubsection{Terminierung}\label{terminierung}

\begin{proposition}
Der Tableau-Algorithmus stoppt nach endlicher Zeit.
\end{proposition}

Beweis analog zu den ohne TBoxen (Prop. 4.5), aber mit Einbezug der TBox. Wir zeigen es also in den selben Schritten:

Beweis in 4 Schritten:

\begin{enumerate}
\def\labelenumi{\arabic{enumi}.}
\item
  Es werden nur I-Bäume mit einem Verzweigungsgrad $\leq \left| C_{0} \right| + \color{red} \left| \MT \right|$ generiert.
\item
  Es werden nur I-Bäume mit einer Tiefe $\color{red} 2^k$ generiert.
  \begin{quote}
  \textbf{T4.16}

  Angenommen, es wird ein I-Baum der Tiefe $> 2^k$ erzeugt.

  Dann wird irgendwann die $\exists '$-Regel auf einen Knoten $v$ der Tiefe $2^k$ angewendet.

  Betrachte Pfad $v_0, \ldots , v_{2^k}$ von der Wurzel bis v. Dieser Pfad hat $2^k+1$ Knoten.

  Weil es nur $2^k$ möglich Knotenbeschriftungen gibt, muss es auf dem Pfad zwei Knoten $v_i$ und $v_j$ geben, mit $0 \leq i < j \leq 2^k$, welche dieselben Knotenbeschriftungen haben, also $L(v_i) = L(v_j)$. Also ist $v_j$ durch $v_i$ blockiert, weswegen auch $v$ blockiert ist. $\lightning$

  Widerspruch zur Anwendung der $\exists '$-Regel auf $v_j$, also ist die Annahme falsch.
  \end{quote}
\item
  Sei $M_{0},M_1,\ldots$ die erzeugte Folge und $B \in M_\MI$ für
  ein $i \geq 0$. Dann ist $B$ durch die Anwendung von maximal
  $\color{red} k^{2^{k}} \cdot k \leq 2^{2^{3k}}= n$ Regeln entstanden (Knoten im Baum mal Größe Knotenbeschriftung).
\end{enumerate}

Danach kann Terminierung wie gehabt mittels Behauptung 3 bewiesen werden.

\subsubsection{Komplexität}

Im Beweis zum 3. Schritt der Terminierung haben wir gesehen, dass die I-Bäume höchstens doppelt exponentiell groß werden.

Dieser Worst-Case kann eintreten!

\begin{lemma}
Es gibt Eingabe $C_0$, $\MT$ für die der Tableau-Algorithmus einen Baum von exponentieller Tiefe generiert.
\end{lemma}

Also: 2-exponentieller Zeit- und Platzaufwand (sogar 3-exponentiell!).

\subsubsection{Bemerkung zur TBox-Regel}

TBoxen führen zu Backtracking:

\begin{center}Normalisierung von $\MT$ zu $\{\top \sqsubseteq \bigsqcap_{C \sqsubseteq D \in \MT} \neg C \sqcup D\}$\end{center}

Daher wird jede $\sqcup$-Regel für \emph{Konzeptinklusion} auf \emph{jeden} Knoten angewendet!

Also braucht man für eine effiziente Implementierung Optimerungstechniken, die die Dijsunktionen, soweit möglich, eliminieren (\enquote{Absorption}).

\subsubsection{Erweiterungen}

Der Algorithmus kann auch auf $\ALCI$, $\ALCQ$ und $\ALCQI$ erweitert werden.

Das ist teilweise subtiler als erwartet, z.B.:

\begin{itemize}
	\item $\ALCI$
	Offensichtlich: Hinzufügen von Regeln für $\exists r^{-}.C$ und $\forall r^-.C$

	Weniger offensichtlich Blockierungsbedingungen muss verschärft werden, sonst ist Algorithmus nicht korrekt.
\end{itemize}

Für $\ALCQI$ ist eine noch aufwendigere Blockierungsbedingung nötig.
