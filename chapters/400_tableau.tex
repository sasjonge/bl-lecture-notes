Automatisches Schlussfolgern spielt eine zentrale Rolle für
Beschreibungslogiken. Insbesondere ist die Ausdrucksstärke von
Beschreibungslogiken stark darauf zugeschnitten.
Dabei ist aber wichtig, dass die relevanten Schlussfolgerungsprobleme
entscheidbar sind, sie ein möglichst geringe Komplexität haben und
Algorithmen existieren, die sich in der Praxis performant verhalten.
Von uns wird hauptsächlich das Problem der Erfüllbarkeit betrachtet.

In der Praxis haben sich hauptsächlich Tableau-Algorithmen und
Resolutionsverfahren als effizient herausgestellt.

\subsection{\texorpdfstring{\ALC}{ALC} ohne TBoxen}\label{alc-ohne-tboxen}

\subsubsection{Negationsnormalform}\label{negationsnormalform}

\begin{definition}[Negationsnormalform]
Ein Konzept ist in \emph{Negationsnormalform} (NNF) gdw. Negation nur auf
Konzeptnamen (also nicht auf zusammengesetzte Konzepte) angewendet wird.
\end{definition}

\begin{lemma}[Linearzeit Umwandlung in NNF]
Jedes Konzept kann in Linearzeit in ein äquivalentes Konzept in
Negationsnormalform umgewandelt werden.
\end{lemma}

\begin{tafel}[Beweis von Linearzeit Umwandlung in NNF]

    \begin{proof}
        Jedes Konzept kann in NNF umgewandelt werden durch erschöpfendes Anwenden von folgenden Regeln
        \begin{itemize}
            \item doppelte Negation: $\neg \neg C \rightsquigarrow C$
            \item de Morgan: \begin{align*}
                    \neg (C \sqcap D) \rightsquigarrow \neg C \sqcup \neg D\\
                    \neg (C \sqcup D) \rightsquigarrow \neg C \sqcap \neg D
                \end{align*}
            \item Dualität: \begin{align*}
                    \neg \exists r.C \rightsquigarrow \forall r.\neg C\\
                    \neg \forall r.C \rightsquigarrow \exists r.\neg C
                \end{align*}
        \end{itemize}
        Jede Regelanwendung ist äquivalenzerhaltend und schiebt Negation immer
        weiter nach innen. Man kann zeigen, dass linear viele Regelanwendungen
        ausreichen.
    \end{proof}
\end{tafel}


\subsubsection{I-Baum}\label{i-baum}

I-Bäume bilde eine Datenstruktur, die ein (partielles) Baummodell
repräsentieren.

\begin{definition}[I-Baum]

Ein \emph{I-Baum} für $C_{0}$ (in NNF) ist knoten- und kantenbeschrifteter
Baum $\left( V,E,\ML \right)$ mit

\begin{itemize}
\item $V$ Knotenmenge
\item $E$ ist Menge beschrifteter Kanten $\left( v,r,v' \right)$ mit
  $v$,$\ v' \in V$, $r$ Rollenname
\item $\ML$: $V \rightarrow 2^{sub(C_{0})}$ ist eine Knotenbeschriftung aus den Teilkonzepten von $C_0$.
\end{itemize}
\end{definition}

\begin{tafel}[Beispiel I-Baum]
    Beispiel:
    \begin{align*}
        C_0 &= A \sqcap \forall r.(\neg A \sqcap \exists r.B)\\
        \sub(C_0) &= \{A, \neg A, B, \exists r.B, \neg A \sqcap \exists r.B, \forall r.(\neg A \sqcap \exists r.B), C_0\}
    \end{align*}

    \includegraphics[width=3.71910in]{media/42ibaum.png}
\end{tafel}

\subsubsection{Tableau-Algorithmus}\label{tableau-algorithmus}

Der Tableau-Algorithmus berechnet eine Folge $$M_0,M_1,\ldots$$ von Mengen von I-Bäumen:
\begin{itemize}
    \item $M_0 = \left\{ B_{\text{ini}} \right\}$ wobei $B_{\text{ini}}$ der \emph{initiale I-Baum} für $C_0$ ist:
        \begin{align*}
            V &:= \left\{ v_{\text{ini}} \right\}\\
            E &:= \emptyset\\
            \ML\left(v_{\text{ini}}\right) &:= \left\{ C_0 \right\}
        \end{align*}
    \item $M_{i + 1}$ entsteht aus $M_i$ durch Anwendung einer Tableau-Regel
        auf einen der I-Bäume in $M_i$, die einen oder mehrere neue I-Bäume
        erzeugt, die den Baum ersetzen.
\end{itemize}

Sei $\left( V,E,\ML \right)$ ein I-Baum.

\begin{itemize}
\item $\sqcap$-Regel
  \begin{itemize}
  \item
    Wähle $v \in V$ und $C \sqcap D \in \ML\left( v \right)$ so dass
    $\left\{ C,D \right\}\  \not\subseteq \ML\left( v \right)$
  \item
    erweitere $\ML(v)$ um $C$ und $D$
  \end{itemize}
\item $\sqcup$-Regel

  \begin{itemize}
  \item
    Wähle $v \in V$ und $C \sqcup D \in \ML\left( v \right)$ so dass
    $\left\{ C,D \right\} \cap \ML\left( v \right) = \emptyset$
  \item
      erweitere $\ML(v)$ um $C$ \emph{oder} um $D$ (ergibt \emph{zwei} I-Bäume)
  \end{itemize}
\item $\exists$-Regel

  \begin{itemize}
  \item
    Wähle $v \in V$ und $\exists r.C \in \ML\left( v \right)$ so dass
    es kein $v' \in V$ gibt mit $\left( v,r,v' \right) \in E$
    und $C \in \ML\left( v' \right)$
  \item
    erweitere V um neuen Konten $v'$ und $E$ um
    $\left( v,r,v' \right)$, setze
    $\ML\left( v' \right) = \left\{ C \right\}$
  \end{itemize}
\item $\forall$-Regel
  \begin{itemize}
  \item
    Wähle $v,v' \in V$ und $\forall r.C \in \ML\left( v \right)$ so
    dass $\left( v,r,v' \right) \in E$ und
    $C\  \notin \ML\left( v \right)$
  \item
    erweitere $\ML(v')$ um $C$
  \end{itemize}
\end{itemize}

Berechnung von $M_{i + 1}$ aus $M_i$ erfolgt durch Auswahl eines $B \in M$ und
Anwendung einer der vier Regeln. Eine Regelanwendung ersetzt $B$ durch einen
neuen I-Baum bzw.\ zwei neue I-Bäume ($\sqcup$-Regel). Ein I-Baum ist \emph{vollständig}, wenn keine Regel darauf anwendbar ist. Intuitiv machen Regelanwendungen implizites Wissen explizit.

Der Algorithmus stoppt, wenn alle I-Bäume vollständig sind. Anschließend wird
\begin{itemize}
    \item \enquote{erfüllbar} zurückgegeben, wenn ein I-Baum gefunden wurde, der keinen \emph{offensichtlichen Widerspruch} enthält. Ein offensichtlicher Widerspruch liegt vor, wenn es einen Knoten $v$ gibt mit $\left\{A, \neg A\right\} \subseteq \ML(v)$ für einen Konzeptnamen $A$.
    \item \enquote{unerfüllbar} andernfalls, also wenn es in \emph{allen} vollständigen I-Bäumen einen offensichtlichen Widerspruch gibt.
\end{itemize}

\begin{tafel}[Beispielanwendung des Tableau-Algorithmus]

Beispiel:

\includegraphics[height=3.83200in]{media/43taleau.png}

\end{tafel}

Wir müssen nun zeigen, dass der Algorithmus
\begin{enumerate}
    \item terminiert:
        Nach endlicher Zeit sind alle I-Bäume vollständig und der Algorithmus kann stoppen und eine Antwort (\enquote{erfüllbar}/\enquote{unerfüllbar}) zurückgeben.
    \item korrekt(\enquote{sound}) ist:
        Wenn der Algorithmus \enquote{erfüllbar} zurückgibt, dann ist das Eingabekonzept auch tatsächlich Erfüllbar.
    \item vollständig(\enquote{complete}) ist:
        Wenn das Eingabekonzept erfüllbar ist, dann gibt der Algorithmus auch \enquote{erfüllbar} zurück.
\end{enumerate}

Wir beginnen mit Terminierung.

\subsubsection{Rollentiefe}
\label{sec:def-rollentiefe}

Die Rollentiefe $\rd\left(C\right)$ von Konzepten
$C \in \sub\left( C_{0} \right)$ ist induktiv definiert:

\begin{itemize}
\item
  $\rd\left( A \right) = \rd\left( \neg A \right) = 0$
\item
  $\rd\left( C \sqcap D \right) = \rd\left( C \sqcup D \right) = \max\left( \rd\left( C \right),\rd\left( D \right) \right)$
\item
  $\rd\left( \exists r.C \right) = \rd\left( \forall r.C \right) = 1 + \rd\left( C \right)$
\end{itemize}

\begin{lemma}
Für alle $C \in \sub\left( C_{0} \right)$ gilt
$\rd\left( C \right) \leq \left| C \right|$.
\end{lemma}

\subsubsection{Multimengen}\label{multimengen}

Multimengen sind Mengen, in denen Elemente mehrfach vorkommen dürfen, z.B.:
$$\{1,1,2,3,4,4,5,6,6,6\}$$

Eine Multimengen über einer Menge $S$ ist eine Abbildung
$$M:S\rightarrow\mathbb{N}$$, die die Anzahl des Vorkommen der Elemente aus
$S$ beschreibt.  Die meisten Begriffe übertragen sich von Mengen auf
Multimengen:

\begin{itemize}
	\item Leere Menge $\emptyset: s \mapsto 0$ für alle $s \in S$
	\item Vereinigung: $(M_1 \cup M_2)(s) := M_1(s) + M_2(s)$ 
	\item Element: $s \in M\ gdw.\ M(s)>0$
    \item Differenz: $(M_1 \setminus M_2)(s) =
        \begin{cases}
                M_1(s) - M_2(s) & \text{wenn } M_1(s) \geq M_2(s)\\
                0 & \text{sonst}
        \end{cases}$
\end{itemize}

$MM(S)$ bezeichnet die Menge aller Multimengen über der Menge S.

Gegeben eine strikte partielle Ordnung $\left( S, < \right)$, ist die
\emph{Multimengenerweiterung} $\left( \text{MM}\left( S \right), <_{\text{mul}} \right)$ definiert als: 

$M_2 <_{\text{mul}}M_1$ gdw. $\exists X,Y \in MM(S)$, so dass

\begin{itemize}
\item
  $\emptyset \neq X \subseteq M_1$
\item
  $M_2 = \left( M_1 \setminus X \right) \cup Y$
\item
    Für alle $y \in Y$ gibt es ein $x \in X$ mit $x > y$.
\end{itemize}

Also erhält man $M_2$ aus $M_1$, indem man einige Elemente entfernt und durch endlich viele \emph{kleinere} ersetzt.
Beispiel:
\begin{align*}
\left\{ 3,1 \right\} >_{\text{mul}}\{ 2,2,2\} >_{\text{mul}}\{ 2,2\} >_{\text{mul}}\{ 2,1,1,1\}
\end{align*}

%TODO: repeat
\setcounter{definition}{5}
\begin{lemma}
    Auch $(MM(S), <_{\text{mul}})$ ist eine strikte partielle Ordnung.
\end{lemma}

Eine partielle Ordnung $<$ heißt wohlfundiert, wenn $<$ keine unendlichen absteigenden Ketten hat. Zum Beispiel ist $(\mathbb{N}, <)$ wohlfundiert, aber $(\mathbb{Z}, <)$ und $({\left[0, 1\right]}_{\mathbb{R}})$ nicht.

\begin{theorem}[Wohlfundiertheit von $(MM(S), <_{\text{mul}})$]
Wenn $\left( S, < \right)$ wohlfundiert ist, dann ist auch $\left( \text{MM}\left( S \right), <_{\text{mul}} \right)$
wohlfundiert.
\end{theorem}


\subsubsection{Terminierung}

%TODO: repeat
\setcounter{definition}{4}
\begin{theorem}[Terminierung des Tableau-Algorithmus]\label{thm:termination}
Der Tableau-Algorithmus stoppt nach endlicher Zeit.
\end{theorem}
\setcounter{definition}{7}

Beweis in 4 Schritten:
\begin{itemize}
    \item Behauptung 1.
  Es werden nur I-Bäume mit einem Verzweigungsgrad
  von maximal $|C_0|$ generiert.
  \begin{tafel}[Beweis der Behauptung für Terminerung] \begin{proof}
        Nur die $\exists$-Regel generiert Nachfolgeknoten, aber
        höchstens einen pro Konzept $\exists r.C \in \sub(C_0)$. Nach
        \autoref{lem:num-sub}  enthält
        $\sub\left( C_{o} \right)$ höchstens $\left| C_{0} \right|$
        viele Teilkonzepte.
    \end{proof}\end{tafel}
\item Behauptung 2.
  Es werden nur I-Bäume mit einer Tiefe von maximal $\left| C_{0} \right|$
  generiert.
  \begin{tafel}[Beweis der Behauptung für Terminierung]\begin{proof}
        Hilfsaussage: Wenn $v$ Knoten mit Tiefe $i$ ist, dann $\rd(C) \leq \rd(C_0) - i$ für alle $C \in \ML(v)$. Daraus folgt Behauptung 2, denn jeder Knoten ist mit mindestens einem Konzept beschriftet. Wenn es einen Knoten in Tiefe $k > \rd(C_0)$ gäbe, dann gäbe es auch $C\in \ML(v)$ und wegen der Hilfsaussage wäre die Rollentiefe negativ, was per Definition der Rollentiefe unmöglich ist.

        Beweis der Hilfsaussage per Induktion über Anzahl an Regelanwendungen:

        \textbf{I.A.} Es gibt nur den Knoten $v_{\text{ini}}$ mit
$L\left( v_{\text{ini}} \right) = \left\{ C_{0} \right\}$. I.V. gilt,
da $i = 0$.

\textbf{I.S.} Fallunterscheidung nach angewandter Regel (exemplarisch
$\sqcap$, $\exists$):

\begin{enumerate}
\item $\sqcap$-Regel. Dann $C \sqcap D \in \ML(v)$ und $\ML(v)$ wird um $C$ und $D$ erweitert.
Nach I.V. gilt: $\rd\left( C \sqcap D \right) \leq \rd\left( C_{0} \right) - i$,
also auch $\rd\left( C \right) \leq rd\left( C_{0} \right) - i$,
weil $\rd\left( C \right) \leq \rd\left( C \sqcap D \right)$.
Analog für $D$.

\item $\exists$-Regel. Dann $\exists r.C \in L\left( v \right)$ und es wird
    ein neuer Knoten $v'$ auf Tiefe $i + 1$ generiert mit
$\ML\left( v' \right) = \left\{ C \right\}$. Es gilt
\begin{align*}
    \rd\left( C \right) &= \rd\left( \exists r.C \right) - 1 & \text{(Def rd.)}\\
                        &\leq \rd\left( C_{0} \right) - i - 1 & \text{(IV)}\\
                        &= \rd\left( C_{0} \right) - (i + 1)
\end{align*} \end{enumerate} \end{proof}
\end{tafel}

\item Behauptung 3.
  Sei $M_{0},M_1,\ldots$ die erzeugte Folge und $B \in M_i$ für
  ein $i \geq 0$. Dann ist $B$ durch die Anwendung von maximal
  \begin{align*}
      \underbrace{\left| C_{0} \right|^{\left| C_{0} \right|}}_\text{\# Knoten im Baum} \cdot \underbrace{\left| C_{0} \right|}_\text{Größe Knotenbeschriftungen} \leq 2^{2\left| C_{0} \right|^{2}} =: n
  \end{align*}
  Regeln entstanden. (folgt kombinatorisch aus Behauptung 1 und 2)

\end{itemize}

Nun kann die Terminierung mittels Behauptung 3 beweisen werden:

\begin{tafel}[Beweis Terminerung Tableau-Algorithmus]

\begin{proof}
    Wir ordnen jedem $M_i$ eine Multimenge $MM_i$ zu. Für jeden Baum $B \in
    M_i$ enthält $MM_i$ die Zahl der Regelanwendungen, mittels derer B
    generiert wurde.
    Weil $<$ auf $\mathbb{N}$ wohldefiniert ist, ist $<_{\text{mul}}$ auf $MM(\{0,\ldots,n\})$ wohldefiniert.

    Offenbar gilt $MM_{i+1} <_{\text{mul}} MM_i$ für alle $i \geq 0$. Also ist die Folge der $MM_i$ endlich.
\end{proof}

\includegraphics[width=3.71910in]{media/45mm.png}
\end{tafel}

\subsubsection{Korrektheit und Vollständigkeit}

\begin{theorem}\label{thm-korrektheit1}
    Wenn der Tableau-Algorithmus \enquote{erfüllbar} zurückgibt, dann ist $C_{0}$
erfüllbar.
\end{theorem}

\begin{proof}
    Wenn der Algorithmus \enquote{erfüllbar} zurückgibt, so hat er einen
    \emph{vollständigen} I-Baum $B = (V, E, \ML)$ \emph{ohne offensichtlichen
    Widerspruch} gefunden. Aus $B$ konstruieren wir Interpretation $\MI$ wie
    folgt:
    \begin{align*}
        \Delta^\MI &= V\\
        r^\MI &= \left\{(v, v') \middle| (v, r, v') \in E\right\} & \text{für alle Rollennamen $r$}\\
        A^\MI &= \left\{v \middle| A \in \ML(v)\right\} & \text{für alle Konzeptnamen $A$}
    \end{align*}
    Die Erfüllbarkeit von $C_0$ folgt dann aus:

    \emph{Behauptung}: Für alle Konzepte $C$ und $v \in V$ gilt

    \begin{center}$C \in L\left( v \right)$ impliziert $v \in C^\MI$\end{center}
\end{proof}

\begin{tafel}[Beweis der Behauptung für Korrektheit]
        Die Behauptung impliziert Erfüllbarkeit von $C_0$. Da $C_0 \in \ML(v_{ini})$ in $B_{ini}$ ist, gilt auch $C_0 \in \ML(v_{ini})$ in $B$. Knotenbeschriftungen werden nie gelöscht. Also ist $v_{ini} \in C_0^\MI$ nach der Behauptung, folglich ist $C_0$ erfüllbar.
    \begin{proof}
        Beweis der Behauptung per Induktion über Struktur von $C$.
        
        \textbf{I.A.} $C = A$ (Konzeptname) Gilt nach Definition von
        $\MI$.
        
        \textbf{I.S.}

\begin{itemize}
\item $C = \neg A$

$A$ ist ein Konzeptname. Da $B$ keinen offensichtlichen Widerspruch enthält,
folgt aus $\neg A \in \ML(v)$, dass i $A \notin \ML(v)$. Nach Definition von
$\MI$ gilt $v \notin A^\MI$. Also $v \in \left( \neg A \right)^\MI$.

\item $C = D \sqcap E$
    \begin{align*}
        &C \in \ML(v)\\
        &\implies (\sqcap\text{-Regel nicht anwendbar}) D \in \ML(v), E \in \ML(v)\\
        &\implies (\text{I.V.}) v \in D^\MI,\ v \in E^\MI\\
        &\implies (\text{Semantik}) v \in \left( D \sqcap E \right)^\MI
    \end{align*}

\item $C = D \sqcup E$

analog zu $\sqcap$.

\item $C = \exists r.D$

Da die $\exists$-Regel nicht anwendbar ist, gibt es $v'\in V$ mit $(v,r,v')
\in E$ und $D \in \ML(v')$

Nach I.V. gilt $v' \in D^\MI$; nach Konstruktion $(v,v') \in r^\MI$, woraus
$v \in (\exists r.D^\MI)$ folgt.

\item $C = \forall r.D$

analog zu $\exists$.
\end{itemize}
    \end{proof}
\end{tafel}

\begin{definition}[Realisierbarkeit]
    \label{def:realisierbarkeit}
Sei $B = \left( V,E,\ML \right)$ ein I-Baum und $\MI$ eine Interpretation. $\MI$
\emph{realisiert} $B$, wenn es eine Funktion
\begin{align*}
    \pi : V \rightarrow \Delta^\MI
\end{align*}
gibt, so dass gilt:
\begin{itemize}
\item
  $\left( v,r,v' \right) \in E$ impliziert
  $\left( \pi\left( v \right),\pi\left( v' \right) \right) \in r^\MI$
\item
  $C \in L\left( v \right)$ impliziert
  $\pi\left( v \right) \in C^\MI$
\end{itemize}

$B$ ist \emph{realisierbar}, wenn es Interpretation $\MI$ gibt, die $B$
realisiert. Eine Menge $M$ von I-Bäumen ist \emph{realisierbar} gdw. ein
$B \in M$ realisierbar.
\end{definition}

Beachte: ein realisierbarer I-Baum enthält keinen offensichtlichen Widerspruch.

\begin{tafel}[Beispiel Realisierbarkeit]\mbox{}\\

\includegraphics[width=2.0in,height=4.0in]{media/47real.png}
\end{tafel}

\begin{theorem}[Vollständigkeit des Tableau Algorithmus]
    Wenn $C_{0}$ erfüllbar ist, dann gibt der Tableau-Algorithmus \enquote{erfüllbar}
zurück.
\end{theorem}

\begin{proof}
    Sei $C_0$ erfüllbar. Nach \autoref{thm:termination} berechnet der Algorithmus eine \emph{endliche} Folge
    \begin{align*}
        M_0,\ldots,M_n
    \end{align*}
von Mengen von I-Bäumen. Behauptung: Für alle $i\leq n$ ist $M_i$ realisierbar.

\begin{tafel}[Beweis der Behauptung für Vollständigkeit]
Beweis der Behauptung per Induktion über $i$.

\textbf{I.A.} $i = 0, M_0 = \{ B_{ini}\}$. $B_{ini}$ ist realisierbar, da $C_0$ erfüllbar ist.

\textbf{I.S.} $i \rightarrow i + 1$. Fallunterscheidung gemäß der Regel mit der $M_{i + 1}$ aus $M_i$ erzeugt wird. Sei B ein realisierbar Baum aus $M_i$, auf den eine Regel angewandt wird. Zu zeigen: veränderter Baum ist auch realisierbar.

Exemplarisch an zwei Regeln:
\begin{itemize}
    \item $\sqcup$-Regel: Dann wird $B = (V, E, \ML) \in M_i$ ersetzt durch $B' = (V, E, \ML') \in M_{i + 1}$, $B'' = (V, E, \ML'') \in M_{i + 1}$ und es gibt $v \in V$ mit 
        \begin{align*}
            &C \sqcup D \in \ML(v)\\
            \ML'(v) &= \ML(v) \cup \{C\}\\
            \ML''(v) &= \ML(v) \cup \{D\}\\
            \ML(u) &= \ML'(u) = \ML''(u) \text{ für alle } u \neq v
        \end{align*}
        Es genügt zu zeigen, dass wenn $B$ realisierbar ist, dann ist $B'$ oder $B''$ realisierbar. Sei $\MI$ eine Interpretation, die $B$ realisiert und $\pi: V \rightarrow \Delta^\MI$ eine Abbildung wie in \autoref{def:realisierbarkeit}. Dann gilt $\pi(v) \in (C \cup D)^\MI$. Nach Semantik ist $\pi(v) \in C^\MI \vee \pi(v) \in D^\MI$. Im ersten Fall realisiert $\MI$ den Baum $B'$, im zweiten $B''$.
    \item $\exists$-Regel: Dann wird $B$ ersetzt durch $B' = (V', E', \ML') \in M_{i + 1}$ und es gibt $v, v' \in V$ mit
        \begin{align*}
            &\exists r.C \in \ML(v)\\
            V' &= V \cup \{V'\}\\
            E' &= E \cup \{(v, r, v')\}\\
            \ML'(v') &= \{c\}\\
            \ML'(u) &= \ML(u) \text{ für alle } u \in V
        \end{align*}
        Es genügt zu zeigen, dass wenn $B$ realisierbar ist, auch $B'$ realisierbar ist. Sei $\MI$ eine Interpretation, die $B$ realisiert und $\pi$ eine Abbildung wie in \autoref{def:realisierbarkeit}. Dann gilt $\pi(v) \in (\exists r.C)^\MI$, dann gibt es $d \in \Delta^\MI$ mit $(\pi(u), d) \in r^\MI$ und $d \in C^\MI$. $\pi$ erweitert um $\pi(v') = d$ bezeugt, dass $\MI$ $B'$ realisiert.
        %TODO: there is a mistake in the above

\end{itemize}

\end{tafel}


\end{proof}

\subsubsection{Komplexitätsanalyse}\label{praktikabilituxe4t}

Nach dem Beweis von \autoref{thm:termination}, können im Tableau Algorithmus die I-Bäume höchstens exponentiell groß werden. Dieser Fall kann tatsächlich eintreffen:
\begin{itemize}
    \item Sei $\forall r^i.C$ eine Abkürzung für $\underbrace{\forall r.\ldots \forall r.}_{i\text{-mal}}C$
    \item Dann generiert der Erfüllbarkeitstest von
        \begin{align*}
            \bigsqcap_{i < n} \forall r^i.(\exists r.B \sqcap \exists r.\neg B)
        \end{align*}
        einen Baum der Größe $2^n$.
\end{itemize}
\begin{tafel}[TODO]

\end{tafel}

Also: mindestens \emph{exponentieller} Zeit- und Platzverbrauch (wegen der Baum\emph{mengen} sogar \emph{doppelt} exponentiell, also $2^{2^n}$).

\subsubsection{Praktikabilität}

Offenbar ist eine naive Implementierung nicht effizient. Dabei kann man aber einige Hinweise/Optimierungen bei der Implementierung beachten:

\begin{itemize}
	\item Es wird nur ein Baum zur Zeit generiert, keine Menge
	\item bei der $\sqcup$-Regel muss man sich entscheiden (Heuristik); ggf. Entscheidung revidieren (Backtracking).
	\item Es wird nur ein Teil des Baumes (Pfad) im Speicher gehalten.
    \item Backjumping: Führe Buch über die \enquote{Herkunft} von Knotenbeschriftungen und Kanten mittels Dependenzmengen. Wenn Backtracking nötig, springe direkt zu einer der Ursachen des Widerspruches zurück.
\end{itemize}

\begin{tafel}[Demonstration des Vorteils von Backjumping]
    Sei 
    \begin{align*}
    \ML(v) = \{\overbrace{\exists r.A \sqcup B}^1, \overbrace{A_1 \sqcup A_1'}^2, \ldots, \overbrace{A_n \sqcup A_n'}^{n + 1}, \forall r.\neg A\}
    \end{align*}
    und angenommen der Tableau-Algorithmus verarbeitet die Konzepte aus $\ML(x)$ der Reihe nach und wählt immer zuerst das linke Disjunkt.
    \begin{center}
    \begin{tikzpicture}
        \begin{scope}[every node/.style={circle,draw,thick}]
            \node (s1) at (0, 3) {$v$};
            \node (s2) at (0, 0) {$v'$};
        \end{scope}
        \begin{scope}[every node/.style={fill=white}]
            \path [->] (s1) edge node {\scriptsize{r}} (s2);
            \node at (s1) [right = 0mm of s1]
            {$\ML(v) \cup \{\overbrace{\exists r.A}^1, \overbrace{A_1}^2, \ldots, \overbrace{A_n}^{n + 1}\}$};
            \node at (s2) [right = 0mm of s2] {$\{A, \neg A\}$};
        \end{scope}
    \end{tikzpicture}
    \end{center}
    Standard Backtracking probiert $2^n$ sinnlose Alternativen bevor die Ursache des Problems gefunden wird ($2, \ldots, n + 1$). Mit Backjumping wird Buch über dir Herkunft von Konzepten geführt und direkt zur Ursache des Problems zurückgesprungen ($1$).
\end{tafel}

\subsection{\texorpdfstring{\ALC}{ALC} mit generellen TBoxen}\label{alc-mit-generellen-tboxen}

Nun wollen wir einen Tableau-Algorithmus für die Erfüllbarkeit in \ALC  \emph{bzgl. TBoxen}. Wir beobachten, dass jede TBox $\MT$ äquivalent ist zu einer TBox der Form
$\left\{ \top \sqsubseteq C_{\MT} \right\}$ mit: 
\begin{align*}
C_{\MT} := \bigsqcap_{C \sqsubseteq D \in \MT}\neg C \sqcup D
\end{align*}

\begin{tafel}[Beispiel TBox Normalisierung]

$$\MT = \{A \sqsubseteq \exists r.B, A \sqcup B \sqsubseteq \forall r.B\}$$

Daraus wird

$$\MT' = \{\top \sqsubseteq (\neg A \sqcup \exists r.B) \sqcap (\neg (A \sqcup B) \sqcup \forall r.B)\}$$

in NNF:

$$\MT'' = \{\top \sqsubseteq (\neg A \sqcup \exists r.B) \sqcap ((\neg A \sqcap \neg B) \sqcup \forall r.B)\}$$
\end{tafel}

Daher können wir im Folgenden annehmen:

\begin{itemize}
	\item die Eingabe $C_0$ ist in NNF und
	\item die Eingabe $\MT$ hat Form $\{\top \sqsubseteq C_{\MT}\}$ mit $C_{\MT}$ in NNF
\end{itemize}

Nun modifiziere den vorigen Algorithmus durch Hinzufügen folgender Regel:

\begin{center}
\emph{TBox-Regel}: wähle $v \in V$ so dass $C_{\MT} \notin L\left( V \right)$ und
erweitere $L\left( v \right)$ um $C_{\MT}$.
\end{center}

Problem: der resultierende Algorithmus Terminiert nicht!

\begin{tafel}[Beispiel Nicht-Terminierung]\mbox{}\\
    Sei $C_0 = A$ und $\MT = \{\top \sqsubseteq \exists r.A\}$.

    \begin{center}
    \begin{tikzpicture}
        \begin{scope}[every node/.style={circle,thick,draw}]
            \node (s1) at (0, 3) {};
            \node (s2) at (0, 1.5) {};
            \node (s3) at (0, 0) {};
        \end{scope}
        \begin{scope}[every node/.style={fill=white}]
            \path [->] (s1) edge node {r} (s2);
            \path [->] (s2) edge node {r} (s3);
            \node at (s1) [right = 1mm of s1] {$A, \exists r.A$};
            \node at (s2) [right = 1mm of s2] {$A, \exists r.A$};
            \node at (s3) [right = 1mm of s3] {$A, \exists r.A$};
            \node (s8) at (0, -0.7) {$\vdots$};
        \end{scope}
    \end{tikzpicture}
    \end{center}
\end{tafel}

\subsubsection{Blockieren}\label{blockieren}

Der Algorithmus terminiert nicht, da die Tiefe der Baummodelle unendlich sein kann. Dies lösen wir, indem wir nur ein \emph{endliches Anfangsstück} eines Baummodells anhand dessen sich die Existenz eines vollständigen Modells entscheiden lässt konstruieren. Dazu müssen wir die Anwendung der $\exists$-Regel einschränken.

\begin{definition}[Blockieren]

Sei $\left( V,E,\ML \right)$ ein I-Baum und $u,v \in V$. Dann ist
$v$ \emph{direkt blockiert durch} $u$, wenn

\begin{enumerate}
\item $u$ Vorgänger von $v$ in $B$ ist und
\item $L\left( v \right) \subseteq L\left( u \right)$
\end{enumerate}

$v$ ist \emph{blockiert}, wenn $v$ direkt blockiert ist oder einen direkt
blockierten Vorgänger hat. Beachte, dass \enquote{Vorgänger} kein direkter Vorgänger sein muss.
\end{definition}

\begin{tafel}[Beispiel Blockieren]\mbox{}\\
    \begin{center}
    \begin{tikzpicture}
        \begin{scope}[every node/.style={circle,thick,draw}]
            \node (v0) at (1, 1.5) {$v_0$};
            \node (v1) at (0, 0) {$v_1$};
            \node (v2) at (2, 0) {$v_2$};
            \node (v3) at (2, -1.5) {$v_3$};
            \node (v4) at (2, -3) {$v_4$};
        \end{scope}
        \begin{scope}[every node/.style={fill=white}]
            \path [->] (v0) edge node {r} (v1);
            \path [->] (v0) edge node {r} (v2);
            \path [->] (v2) edge node {r} (v3);
            \path [->] (v3) edge node {r} (v4);
            \node at (v0) [right = 1mm of v0]
            {$A \sqcap \exists r.A, A, \exists r.A, \exists r.\exists r.(A\sqcap \exists r.A)$};
            \node at (v1) [left = 1mm of v1] {$A$};
            \node at (v2) [right = 1mm of v2] {$\exists r.(A\sqcap \exists r.A)$};
            \node at (v3) [right = 1mm of v3] {$A \sqcap \exists r.A, A, \exists r.A$};
            \node at (v4) [right = 1mm of v4] {$A, B$};
        \end{scope}
    \end{tikzpicture}
    \end{center}
    Dann ist 
    \begin{itemize}
        \item $v_1$ direkt blockiert durch $v_0$,
        \item $v_3$ direkt blockiert durch $v_0$,
        \item $v_1, v_3, v_4$ sind blockiert und
        \item $v_0, v_2$ sind nicht blockiert.
    \end{itemize}
\end{tafel}

\emph{$\exists'$-Regel}
\begin{itemize}
\item
  Wähle $v \in V$ und $\exists r.C \in \ML\left( v \right)$ so dass
  $v$ \emph{nicht blockiert ist und} es kein $v' \in V$ gibt mit
  $\left( v,r,v' \right) \in E$ und $C \in \ML\left( v' \right)$
\item
  erweitere V um neuen Knoten $v'$ und $E$ um
  $\left( v,r,v' \right)$; setze
  $\ML\left( v' \right) = \left\{ C \right\}$
\end{itemize}

\begin{tafel}[Beispiel $\exists'$-Regel]\mbox{}\\

\includegraphics[width=5.71910in,height=2.33200in]{media/414block.png}

\end{tafel}

\subsubsection{Vollständigkeit}\label{vollstuxe4ndigkeit}

\begin{theorem}
    Wenn $C_0$ bezüglich $\MT$ erfüllbar ist, dann gibt der Algorithmus \enquote{erfüllbar} zurück.
\end{theorem}

Beweis wie ohne TBoxen: Alle $M_{0},\ldots,\ M_{n}$ sind realisierbar
bzgl. $\MI$ (Induktion), also enthält $M_{n}$ einen Baum ohne
offensichtlichen Widerspruch. Unterschied hierbei ist, dass Realisierbarkeit bezüglich Modellen von $\MT$ definiert ist und es einen neuen Fall für die TBox-Regel gibt.

\subsubsection{Korrektheit}\label{korrektheit}

\begin{theorem}
    Wenn der Algorithmus \enquote{erfüllbar} zurückgibt, so ist $C_0$ erfüllbar bzgl. $\MT$
\end{theorem}

\begin{proof}
    Analog zu \autoref{thm-korrektheit1} konstruieren wir aus einem vollständigen I-Baum $B = (V, E, \ML)$ ohne offensichtlichen Widerspruch ein Modell $\MI$:
    \begin{align*}
        \Delta^\MI = &\{ v \in V \mid v \text{ unblockiert } \}\\
        r^\MI = &\{(v, v') \in \Delta^\MI \times \Delta^\MI \mid (v, r, v') \in E\}\\
                &\cup \{(v, u) \in \Delta^\MI \times \Delta^\MI \mid \exists (v, r, v') \in E, v' \text{ direkt blockiert von u }\}\\
        A^\MI = & \{ v \mid A \in \ML(v)\}
    \end{align*}
    Behauptung: Für alle Konzepte $C$ und Knoten $v \in \Delta^\MI$ gilt:
    \begin{align*}
        C \in \ML(v) \text{ impliziert } v \in C^\MI 
    \end{align*}
    \begin{tafel}[Beweis der Behauptung für Korrektheit]
    Die Behauptung impliziert, dass $C_0$ bezüglich $\MT$ erfüllbar ist:
    \begin{itemize}
        \item Da $C_0 \in \ML(v_{\text{ini}}) \implies v_{\text{ini}} \in C^\MI$
        \item Da die TBox-Regel nicht anwendbar ist, gilt $C_\MT \in \ML(v)$ für alle $v$, also $v \in C^\MI_\MT$
        \item Also ist $\MI$ Modell von $\MT$ ($\MI \models \MT$).
    \end{itemize}
    Beweis der Behauptung per Induktion über Struktur von $C$. Die Fälle $C = A$, $C=\neg A$, $C=D\sqcap E$, $C = D\sqcup E$ sind analog zum Fall ohne TBoxen.
    \begin{itemize}
        \item $C = \exists r.D$:  Sei $\exists r.D \in \ML(v)$, zu zeigen $v \in {(\exists r.D)}^\MI$.
            Da die $\exists'$-Regel nicht mehr anwendbar ist, gibt es $v' \in V$ mit $(v, r, v') \in E$ und $D \in \ML(v')$.
            \begin{itemize}
                \item 1.Fall $v'$ unblockiert. Dann  $(v, v') \in r^\MI$ (Definition von $\MI$) und $v' \in D^\MI$ (IV).
                \item 2.Fall $v'$ ist blockiert. Da $v'$ direkter Nachfolger von $v$, ist $v'$ direkt blockiert von unblockierten Vorgänger $u$ von $v$ (oder $u = v$). Es gilt $(v, u) \in r^\MI$ (Definition $\MI$), außerdem $D \in \ML(v') \subseteq \ML(u)$, also $u \in D^\MI$ nach IV.
            \end{itemize}
        \item $C = \forall r.D$:  Sei $(v, u) \in r^\MI$ zu zeigen $u \in D^\MI$. Sei $(\forall r.D) \in \ML(v)$.
            \begin{itemize}
                \item 1.Fall $(v, r, u) \in E$ $u$ unblockiert. Da $\forall$-Regel nicht anwendbar ist, gilt $D \in \ML(u)$, also $u \in D^\MI$ nach IV.
                \item 2.Fall $(v, r, v') \in E$, $v'$ direkt blockiert von $u$. Da $\forall$-Regel nicht anwendbar, gilt $D \in \ML(v')$. Da $\ML(v') \subseteq \ML(u)$ gilt $D \in \ML(u)$, also auch $u \in D^\MI$ nach IV.
            \end{itemize}
    \end{itemize}
    \end{tafel}
\end{proof}

\subsubsection{Terminierung}\label{terminierung}

\begin{theorem}[Terminierung]
    \label{thm:termination2}
Der Tableau-Algorithmus stoppt nach endlicher Zeit.
\end{theorem}

\begin{proof}
    Beweis analog zu den ohne TBoxen (\autoref{thm:termination}), aber mit Einbezug der TBox. Wir zeigen es also in den selben Schritten:

\begin{enumerate}
\item
  Es werden nur I-Bäume mit einem Verzweigungsgrad $\leq k := \left| C_0 \right| + \left| \MT \right|$ generiert.
\item
  Es werden nur I-Bäume mit einer Tiefe von maximal $2^k$ generiert.
  \begin{tafel}[Beweis der maximalen I-Baum Tiefe]
  Angenommen, es wird ein I-Baum der Tiefe $> 2^k$ erzeugt. Dann wird irgendwann die $\exists '$-Regel auf einen Knoten $v$ der Tiefe $2^k$ angewendet.

  Betrachte Pfad $v_0, \ldots , v_{2^k}$ von der Wurzel bis v. Dieser Pfad hat $2^k+1$ Knoten.

  Weil es nur $2^k$ möglich Knotenbeschriftungen gibt, muss es auf dem Pfad zwei Knoten $v_i$ und $v_j$ geben, mit $0 \leq i < j \leq 2^k$, welche dieselben Knotenbeschriftungen haben, also $\ML(v_i) = \ML(v_j)$. Also ist $v_j$ durch $v_i$ blockiert, weswegen auch $v$ blockiert ist. $\lightning$

  Widerspruch zur Anwendung der $\exists '$-Regel auf $v_j$, also ist die Annahme falsch.
  \end{tafel}
\item
  Sei $M_0,M_1,\ldots$ die erzeugte Folge und $B \in M_\MI$ für
  ein $i \geq 0$. Dann ist $B$ durch die Anwendung von maximal
  \begin{align*}
      \underbrace{k^{2^k}}_\text{Knoten im Baum} \cdot
      \underbrace{k}_\text{Größe Knotenbeschriftungen} \leq 2^{2^{3k}}
  \end{align*}
  Regeln entstanden.
\end{enumerate}

Danach kann Terminierung wie gehabt mittels Behauptung 3 und einer Multimengen-Ordnung bewiesen werden.

\end{proof}

\subsubsection{Komplexitätsanalyse}

Es kann beobachtet werden, dass die I-Bäume höchstens \emph{doppelt} exponentiell groß werden. (Beweis von \autoref{thm:termination2})

Dieser Worst-Case kann eintreten:

\begin{lemma}
Es gibt Eingabe $C_0$, $\MT$, für die der Tableau-Algorithmus einen Baum von exponentieller Tiefe generiert.
\end{lemma}

Idee: Verwende Konzeptnamen $A_0,\ldots, A_{n - 1}$ als Bits eines binären Zählers. Setze
\begin{align*}
    \top \sqsubseteq \exists r. \top \sqcap \exists s.\top
\end{align*}
und erzwinge, dass der Zähler entlang der r/s-Kanten inkrementiert wird (modulo $2^n$). Dann wiederholt sich erst nach $2^n + 1$ Schritten der Wert des Zählers und es kann blockiert werden.
Setze also $C_0 = \neg A_0 \sqcap \dots \sqcap \neg A_{n - 1}$ und
\begin{align*}
    \MT = \{&\top \sqsubseteq \exists r.\top \sqcap \exists s.\top \sqcap\\
            &\bigsqcap_{k<n} (\overbrace{(A_0 \sqcap \dots \sqcap A_{k - 1} \implies \forall r/s.\neg A_k)}^1\\
            & \overbrace{(A_0\sqcap \dots \sqcap A_{k - 1} \sqcap \neg A_k \implies \forall r/s. A_k)}^2\\
            & \overbrace{((\neg A_0 \sqcup \dots \sqcup \neg A_{k - 1}) \sqcap A_k \implies \forall r/s A_k)}^3\\
            & \overbrace{((\neg A_0 \sqcup \dots \sqcup \neg A_{k - 1}) \sqcap \neg A_k \implies \forall r/s \neg A_k)}^4)
\end{align*}

1 + 2: k-tes Bit wird geändert falls alle Bits $0 \ldots k-1$ den Wert 1 haben. 3 + 4: sonst bleibt das k-te Bit gleich.
\begin{align*}
    \left| \MT \right| = \mathcal{O}(n^2)\\
    \left| C \right| = \mathcal{O}(n)
\end{align*}


Also: mindestens 2-exponentieller Zeit- und Platzaufwand (wegen der Baummengen sogar sogar 3-exponentiell!).

\subsubsection{Bemerkung zur TBox-Regel}

TBoxen führen zu erheblich mehr Backtracking:
\begin{align*}
    \text{Normalisierung von }\MT\text{ zu }\{\top \sqsubseteq \bigsqcap_{C \sqsubseteq D \in \MT} \neg C \sqcup D\}
\end{align*}

Daher wird jede $\sqcup$-Regel für \emph{Konzeptinklusion} auf \emph{jeden} Knoten angewendet!

Also braucht man für eine effiziente Implementierung Optimerungstechniken, die die \enquote{TBox-Disjunktionen}, soweit möglich, eliminieren (\enquote{Absorption}).

\begin{tafel}[Absorption --- TODO]

\end{tafel}

\subsection{Erweiterungen von \texorpdfstring{\ALC}{ALC}}

Der Algorithmus kann auch auf $\ALCI$, $\ALCQ$ und $\ALCQI$ erweitert werden.

Das ist teilweise subtiler als erwartet, z.B.:

\begin{itemize}
	\item $\ALCI$
	Offensichtlich: Hinzufügen von Regeln für $\exists r^{-}.C$ und $\forall r^-.C$

	Weniger offensichtlich Blockierungsbedingungen muss verschärft werden, sonst ist Algorithmus nicht korrekt.
\end{itemize}

Für $\ALCQI$ ist eine noch aufwendigere Blockierungsbedingung nötig.
