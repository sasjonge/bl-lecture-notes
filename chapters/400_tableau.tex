\subsection{ALC ohne TBoxen}\label{alc-ohne-tboxen}

\subsubsection{Negationsnormalform}\label{negationsnormalform}

Konzept ist in \emph{Negationsnormalform} (NNF) gdw. Negation nur auf
Konzeptnamen angewendet wird.

\subsubsection{Lemma 4.2}\label{lemma-4.2}

Jedes Konzept kann in Linearzeit in ein äquivalentes Konzept in NNF
umgewandelt werden.

Beweisskizze. Wende Gesetzte der doppelten Negation, de Morgan und
Dualität von $\exists$, $\forall$ an.

\subsubsection{I-Baum}\label{i-baum}

\emph{I-Baum} für $C_{0}$ (in NNF) ist knoten- und kantenbeschrifteter
Baum $\left( V,E,L \right)$ mit

\begin{itemize}
\item
  $V$ Knotenmenge
\item
  $E$ ist Menge beschrifteter Kanten $\left( v,r,v^{'} \right)$ mit
  $v$,$\ v^{'} \in V$, $r$ Rollenname
\item
  $L$: $V \rightarrow 2^{sub(C_{0})}$
\end{itemize}

\subsubsection{Tableau-Algorithmus}\label{tableau-algorithmus}

Berechnet Folge $M_{0},M_{1},\ldots$ von Mengen von I-Bäumen:

$M_{0} = \left\{ B_{\text{ini}} \right\}$ mit $B_{\text{ini}}$
\emph{initialer I-Baum} für $C_{0}$:

\begin{itemize}
\item
  $V \left\{ v_{\text{ini}} \right\}$
\item
  $E \varnothing$
\item
  $L\left( v_{\text{ini}} \right) \left\{ C_{0} \right\}$
\end{itemize}

$M_{i + 1}$ entsteht aus $M_{i}$ durch Anwendung der Tableau-Regeln
auf irgendeinen I-Baum in $M_{i}$ und anschließendes Austauschen des
verwendeten Baumes durch den neu erzeugten (Sei $\left( V,E,L \right)$
I-Baum):

\begin{itemize}
\item
  $\sqcap$-Regel

  \begin{itemize}
  \item
    Wähle $v \in V$ und $C \sqcap D \in L\left( v \right)$ so dass
    \emph{nicht} $\left\{ C,D \right\}\  \subseteq L\left( v \right)$
  \item
    erweitere $L(v)$ um $C$ und $D$
  \end{itemize}
\item
  $\sqcup$-Regel

  \begin{itemize}
  \item
    Wähle $v \in V$ und $C \sqcup D \in L\left( v \right)$ so dass
    $\left\{ C,D \right\}\  \cap L\left( v \right) = \varnothing$
  \item
    erweitere $L(v)$ um $C$ oder $D$ (ergibt zwei I-Bäume)
  \end{itemize}
\item
  $\exists$-Regel

  \begin{itemize}
  \item
    Wähle $v \in V$ und $\exists r.C \in L\left( v \right)$ so dass
    es kein $v^{'} \in V$ gibt mit $\left( v,r,v^{'} \right) \in E$
    und $C\  \in L\left( v' \right)$
  \item
    erweitere V um neuen Konten $v^{'}$ und $E$ um
    $\left( v,r,v^{'} \right)$; setze
    $L\left( v^{'} \right) = \left\{ C \right\}$
  \end{itemize}
\item
  $\forall$-Regel

  \begin{itemize}
  \item
    Wähle $v,v^{'} \in V$ und $\forall r.C \in L\left( v \right)$ so
    dass $\left( v,r,v^{'} \right) \in E$ und
    $C\  \notin L\left( v \right)$
  \item
    erweitere $L(v')$ um $C$
  \end{itemize}
\end{itemize}

Stoppe, wenn alle Regeln erschöpfend angewandt wurden. Gib „erfüllbar``
zurück, falls es einen I-Baum ohne offensichtlichen Widerspruch
($\left\{ A,\neg A \right\} \subseteq L\left( v \right)$) gibt;
„unerfüllbar`` sonst.

\subsubsection{Definition Rollentiefe}\label{definition-rollentiefe}

Rollentiefe $\text{rd}\left( C \right)$ von Konzepten
$C \in sub\left( C_{0} \right)$ ist induktiv definiert:

\begin{itemize}
\item
  $\text{rd}\left( A \right) = rd\left( \neg A \right) = 0$
\item
  $\text{rd}\left( C \sqcap D \right) = rd\left( C \sqcup D \right) = \max\left( \text{rd}\left( C \right),rd\left( D \right) \right)$
\item
  $\text{rd}\left( \exists r.C \right) = rd\left( \forall r.C \right) = 1 + rd\left( C \right)$
\end{itemize}

\paragraph{Lemma 4.4}\label{lemma-4.4}

Für alle $C \in sub\left( C_{0} \right)$ gilt
$\text{rd}\left( C \right) \leq \left| C \right|$.

\subsubsection{Multimengen}\label{multimengen}

Multimengen sind Mengen, in denen Elemente mehrfach vorkommen dürfen.
Formal: Abbildung $M:S\mathbb{\rightarrow N}$, welche jedes Element
auf die Anzahl seines Vorkommens abbildet.

$MM(S)$ ist die Menge aller Multimengen über der Menge S.

Gegeben strikte partielle Ordnung $\left( S, < \right)$, ist die
Multimengenerweiterung
$\left( \text{MM}\left( S \right), <_{\text{mul}} \right)$ definiert
als $M_{2} <_{\text{mul}}M_{1}$ gdw. $\exists X,Y \in MM(S)$, so
dass

\begin{itemize}
\item
  $\varnothing \neq X \subseteq M_{1}$
\item
  $M_{2} = \left( M_{1} \smallsetminus X \right) \cup Y$
\item
  $\forall y \in Y\exists x \in X : x > y$
\end{itemize}

Beispiel:
$\left\{ 3,1 \right\} >_{\text{mul}}\{ 2,2,2\} >_{\text{mul}}\{ 2,2\} >_{\text{mul}}\{ 2,1,1,1\}$

\paragraph{Theorem 4.6}\label{theorem-4.6}

Wenn $\left( S, < \right)$ wohlfundiert (hat keine unendlichen
absteigenden Ketten) ist, dann ist auch
$\left( \text{MM}\left( S \right), <_{\text{mul}} \right)$
wohlfundiert.

\hypertarget{proposition-4.5-terminierung}{\subsubsection{Proposition
4.5 (Terminierung)}\label{proposition-4.5-terminierung}}

Der Tableau-Algorithmus stoppt nach endlicher Zeit.

Beweis in 4 Schritten:

\begin{enumerate}
\def\labelenumi{\arabic{enumi}.}
\item
  Es werden nur I-Bäume mit einem Verzweigungsgrad
  $\leq \left| C_{0} \right|$ generiert.
\end{enumerate}

Beweisskizze. Nur die $\exists$-Regel generiert Nachfolger, aber
höchstens einen für jedes Konzept $\exists r.C \in sub(C_{0})$. Nach
\protect\hyperlink{lemma-3.13}{Lemma 3.13} enthält
$\text{sub}\left( C_{o} \right)$ höchstens $\left| C_{0} \right|$
viele Konzepte.

\begin{enumerate}
\def\labelenumi{\arabic{enumi}.}
\item
  Es werden nur I-Bäume mit einer Tiefe $\leq \left| C_{0} \right|$
  generiert.
\end{enumerate}

Beweisskizze über die Anzahl der Regelanwendungen. Zu zeigen: Wenn $v$
Knoten mit Tiefe $i$ ist, dann gilt
$\text{rd}\left( C \right) \leq rd\left( C_{0} \right) - i$ für alle
$C \in L\left( v \right)$.

\textbf{I.A.} Es gibt nur Knoten $v_{\text{ini}}$ mit
$L\left( v_{\text{ini}} \right) = \left\{ C_{0} \right\}$. I.V. gilt,
da $i = 0$.

\textbf{I.S.} Fallunterscheidung nach angewandter Regel (exemplarisch
$\sqcap$, $\exists$):

\begin{enumerate}
\def\labelenumi{\alph{enumi}.}
\item
  $\sqcap$-Regel
\end{enumerate}

\begin{quote}
$C \sqcap D \in L(v)$ und $L(v)$ wird durch $C$, $D$ erweitert.
Nach I.V. gilt:
$\text{rd}\left( C \sqcap D \right) \leq rd\left( C_{0} \right) - i$,
also auch $\text{rd}\left( C \right) \leq rd\left( C_{0} \right) - i$,
weil $\text{rd}\left( C \right) \leq rd\left( C \sqcap D \right)$.
Analog für $D$.
\end{quote}

\begin{enumerate}
\def\labelenumi{\alph{enumi}.}
\item
  $\exists$-Regel
\end{enumerate}

\begin{quote}
Dann $\exists r.C \in L\left( v \right)$ und es wird neues $v^{'}$
auf Tiefe $i + 1$ generiert mit
$L\left( v^{'} \right) = \left\{ C \right\}$. Es gilt
$\text{rd}\left( C \right) = rd\left( \exists r.C \right) - 1 \leq rd\left( C_{0} \right) - i - 1 = rd\left( C_{0} \right) - (i + 1)$
\end{quote}

\begin{enumerate}
\def\labelenumi{\arabic{enumi}.}
\item
  Sei $M_{0},M_{1},\ldots$ die erzeugte Folge und $B \in M_{i}$ für
  ein $i \geq 0$. Dann ist $B$ durch die Anwendung von maximal
  $\left| C_{0} \right|^{\left| C_{0} \right|} \cdot \left| C_{0} \right| \leq 2^{2\left| C_{0} \right|^{2}} = n$
  Regeln entstanden (Knoten im Baum mal Größe Knotenbeschriftung).
\end{enumerate}

\hypertarget{proposition-4.7-korrektheit}{\subsubsection{Proposition 4.7
(Korrektheit)}\label{proposition-4.7-korrektheit}}

Wenn der Tableau-Algorithmus „erfüllbar`` zurückgibt, so ist $C_{0}$
erfüllbar.

Beweisskizze per Induktion über die Struktur von $C$.
„erfüllbar``-Ausgabe bedeutet widerspruchsfreier, vollständiger I-Baum
$B = \left( V,E,L \right)$ gefunden. Konstruiere Interpretation $I$:

\begin{itemize}
\item
  $\Delta^{I} = V$
\item
  $r^{I} = \left\{ \left( v,v^{'} \right) \middle| \left( v,r,v^{'} \right) \in E \right\}$
  für alle Rollennamen $r$
\item
  $A^{I} = \left\{ v \middle| A \in L\left( v \right) \right\}$ für
  alle Konzeptnamen $A$
\end{itemize}

Behauptung: Für alle Konzepte $C$ und $v \in V$ gilt
$C \in L\left( v \right)$ impliziert $v \in C^{I}$. Da
$C_{0} \in L\left( v_{\text{ini}} \right)$ in $B_{\text{ini}}$ gilt
auch $C_{0} \in L\left( v_{\text{ini}} \right)$ in $B$. Also
$v_{\text{ini}} \in C_{0}^{I}$ nach Behauptung, weswegen dann
$C_{0}$ erfüllbar.

\textbf{I.A.} $C = A$ (Konzeptname) I.B. gilt nach Definition von
$I$.

\textbf{I.S.} An den Beispielen $\neg$, $\sqcap$

\begin{itemize}
\item
  $C = \neg A$
\end{itemize}

$A$ Konzeptname. Da $B$ keinen offensichtlichen Widerspruch hat,
folgt das $A \notin L(v)$. Nach Definition von $I$ gilt
$v \notin A^{I}$. Also $v \in \left( \neg A \right)^{I}$.

\begin{itemize}
\item
  $C = D \sqcap E$
\end{itemize}

$C \in L\left( v \right) \Rightarrow$ ($\sqcap$-Regel nicht
anwendbar)
$D \in L\left( v \right),\ E \in L\left( v \right) \Rightarrow$ (I.V.)
$v \in D^{I},\ v \in E^{I} \Rightarrow$ (Semantik)
$v \in \left( D \sqcap E \right)^{I}$

\hypertarget{realisierbarkeit}{\subsubsection{Realisierbarkeit}\label{realisierbarkeit}}

Sei $B = \left( V,E,L \right)$ ein I-Baum. Interpretation $I$
\emph{realisiert} $B$ gdw. es gibt eine Funktion
$\pi\ :V \rightarrow \Delta^{I}$ so dass

\begin{itemize}
\item
  $\left( v,r,v^{'} \right) \in E$ impliziert
  $\left( \pi\left( v \right),\pi\left( v^{'} \right) \right) \in r^{I}$
\item
  $C \in L\left( v \right)$ impliziert
  $\pi\left( v \right) \in C^{I}$
\end{itemize}

B ist realisierbar, wenn es Interpretation $I$ gibt, die $B$
realisiert. Menge $M$ von I-Bäumen ist realisierbar gdw. ein
$B \in M$ realisierbar.

\subsubsection{Proposition 4.9
(Vollständigkeit)}\label{proposition-4.9-vollstuxe4ndigkeit}

Wenn $C_{0}$ erfüllbar, so gibt der Tableau-Algorithmus „erfüllbar``
zurück.

Beweisskizze per Induktion über $i$. Sei $C_{0}$ erfüllbar. Nach
\protect\hyperlink{proposition-4.5-terminierung}{Proposition 4.5}
berechnet der Algorithmus endlich Folge $M_{0},\ldots,M_{n}$. Wir
zeigen: $M_{i}$ ist realisierbar für alle $0 \leq i \leq n$. Daraus
folgt: Es gibt realisierbaren Baum $B \in M_{n}$ und damit enthält
$B$ keinen offensichtlichen Widerspruch. Also gibt der Algorithmus
„erfüllbar`` zurück.

\textbf{I.A.} $i = 0$. $M_{0} = {\{ B}_{\text{ini}}\}$.
$B_{\text{ini}}$ ist realisierbar, weil $C_{0}$ erfüllbar.

\textbf{I.S.} Fallunterscheidung gemäß der Regel, mit der $M_{i + 1}$
aus $M_{i}$ erzeugt wurde. Sei $B$ realisierbarer Baum aus
$M_{i}$, auf welchen Regel angewandt wird. Beispielhaft
$\sqcup$-Regel:

\begin{enumerate}
\def\labelenumi{\arabic{enumi}.}
\item
  $\sqcup$-Regel
\end{enumerate}

Dann wird $B = \left( V,E,L \right)$ ersetzt durch
$B^{'} = \left( V,E,L^{'} \right) \in M_{i + 1}$ und
$B^{''} = \left( V,E,L^{''} \right) \in M_{i + 1}$ und es gibt
$v \in V$ mit

\begin{itemize}
\item
  $\left( C \sqcup D \right) \in L(v)$
\item
  $L^{'}\left( v \right) = L\left( v \right) \cup \left\{ C \right\}$,
  $L^{''}\left( v \right) = L\left( v \right) \cup \left\{ D \right\}$
\item
  $L^{'}\left( u \right) = L^{''}\left( u \right) = L\left( u \right)$
  für alle $u \neq v$
\end{itemize}

Es genügt zu zeigen, dass wenn $B$ realisiertbar, dann $B'$ oder
$B''$ realisierbar. Sei $\text{I\ }$Interpretation, die $B$
realisiert und $\pi\ :V \rightarrow \Delta^{I}$ Abbildung wie in
\protect\hyperlink{realisierbarkeit}{Definition 4.8}. Dann gilt
$\pi\left( v \right) \in \left( C \sqcup D \right)^{I}$. Nach
Semantik: $\pi\left( v \right) \in C^{I}$ oder
$\pi\left( v \right) \in D^{I}$. Also realisiert $I$ den Baum $B'$
oder $B''$.

\subsubsection{Praktikabilität}\label{praktikabilituxe4t}

I-Bäume können höchstens exponentiell groß werden.

\subsection{ALC mit generellen TBoxen}\label{alc-mit-generellen-tboxen}

Jede TBox $T$ ist äquivalent zu einer TBox der Form
$\left\{ \top \sqcap C_{T} \right\}$. Setzte
$C_{T} \prod_{C \sqsubseteq D \in T}^{}{\neg C \sqcup D}$.

\subsubsection{TBox-Regel}\label{tbox-regel}

Wähle $v \in V$ so dass $C_{T} \notin L\left( V \right)$ und
erweitere $L\left( v \right)$ um $C_{T}$.

Problem: Terminiert nicht.

\subsubsection{Blockieren}\label{blockieren}

Sei $\left( V,E,L \right)$ ein I-Baum und $u,v \in V$. Dann ist
$v$ direkt blockiert durch $u$, wenn

\begin{enumerate}
\def\labelenumi{\arabic{enumi}.}
\item
  $u$ Vorgänger von $v$ in $B$ ist und
\item
  $L\left( v \right) \subseteq L\left( u \right)$
\end{enumerate}

$v$ ist blockiert, wenn $v$ direkt blockiert ist oder einen direkt
blockierten Vorgänger hat.

\subsubsection{\texorpdfstring{Neue $\exists$-Regel
($\exists^{'}$-Regel)}{Neue \textbackslash{}exists-Regel (\textbackslash{}exists\^{}\{'\}-Regel)}}\label{neue-exists-regel-exists-regel}

\begin{itemize}
\item
  Wähle $v \in V$ und $\exists r.C \in L\left( v \right)$ so dass
  $v$ \emph{nicht blockiert ist und} es kein $v^{'} \in V$ gibt mit
  $\left( v,r,v^{'} \right) \in E$ und $C\  \in L\left( v' \right)$
\item
  erweitere V um neuen Konten $v^{'}$ und $E$ um
  $\left( v,r,v^{'} \right)$; setze
  $L\left( v^{'} \right) = \left\{ C \right\}$
\end{itemize}

\subsubsection{Vollständigkeit}\label{vollstuxe4ndigkeit}

Beweis wie ohne TBoxen: Alle $M_{0},\ldots,\ M_{n}$ sind realisierbar
bzgl. $I$ (Induktion), also enthält $M_{n}$ einen Baum ohne
offensichtlichen Widerspruch (Nur neue Fallunterscheidung für TBox-Regel
und Realisierbarkeitsbegriff auf TBoxen erweitert).

\subsubsection{Korrektheit}\label{korrektheit}

Beweisskizze per Induktion über die Struktur von $C$. Definiere
Interpretation $I$:

\begin{itemize}
\item
  $\Delta^{I} = \left\{ v \in V \middle| \text{v\ }\mathrm{\text{nicht\ blockiert}} \right\}$
\item
  $r^{I} = \left\{ \left( v,v^{'} \right)\  \middle| \ \left( v,r,v^{'} \right) \in E \right\} \cup \left\{ \left( v,u \right)\  \middle| \ \exists\left( v,r,v^{'} \right) \in E\mathrm{\ }\mathrm{\text{und}}\mathrm{\ }v^{'}\mathrm{\ }\mathrm{\text{direkt\ blockiert\ durch}}\mathrm{\ }u \right\}$
\item
  $A^{I} = \left\{ v \middle| A \in L\left( v \right) \right\}$
\end{itemize}

Behauptung: Für alle ALC-Konzepte $C$ und $v \in \Delta^{I}$ gilt:
$C \in L\left( v \right) \Rightarrow v \in C^{I}$. Die Behauptung
impliziert wie gewünscht, dass

\begin{itemize}
\item
  $I$ Modell von $T$ ist.
\end{itemize}

Da die TBox-Regel nicht anwendbar ist, gilt
$C_{T} \in L\left( v \right)$ für alle $v \in V$. Also
$v \in C_{T}^{I}$ für alle $v \in \Delta^{I}$.

\begin{itemize}
\item
  $I$ Modell von $C_{0}$ ist.
\end{itemize}

Da $C_{0} \in L\left( v_{\text{ini}} \right)$ gilt nach Behauptung
$v_{\text{ini}} \in C_{0}^{I}$.

\textbf{I.A.} Siehe Beweis zu
\protect\hyperlink{proposition-4.7-korrektheit}{Proposition 4.7}.

\textbf{I.S.} Schritte wie in Beweis zu Proposition 4.7, außer:

\begin{itemize}
\item
  $C = \exists r.D$
\end{itemize}

Sei $\exists r.D \in L\left( v \right)$. Da die $\exists'$-Regel
nicht anwendbar ist, gibt es $v^{'} \in V$ mit
$\left( v,r,v^{'} \right) \in E$ und $D \in L\left( v \right)$.
Fallunterscheidung:

\begin{enumerate}
\def\labelenumi{\arabic{enumi}.}
\item
  $v'$ unblockiert. Dann $\left( v,v^{'} \right) \in r^{I}$
  (Definition $I$), $v^{'} \in D^{I}$ (\textbf{I.V.})
  $\Rightarrow v \in \left( \exists r.D \right)^{I}$
\item
  $v^{'}$ blockiert. Da der direkte Vorgänger $v$ von $v'$
  unblockiert ist, ist $v'$ direkt blockiert von unblockiertem
  Vorgänger $u$. Es gilt:
\end{enumerate}

\begin{itemize}
\item
  $\left( v,u \right) \in r^{I}$ nach Definition $r^{I}$
\item
  $D \in L\left( v \right) \subseteq L\left( u \right)$
  (Blockierungsbedingung)
\item
  $\Rightarrow u \in D^{I}$ (\textbf{I.V.})
\end{itemize}

\begin{quote}
Also $v \in \left( \exists r.D \right)^{I}$.
\end{quote}

\begin{itemize}
\item
  $C = \forall r.D$
\end{itemize}

Ähnlich zu oberem Fall.

\subsubsection{Terminierung}\label{terminierung}

Beweis analog, aber mit Einbezug der TBox.