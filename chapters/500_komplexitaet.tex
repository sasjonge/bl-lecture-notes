\subsection{Komplexität mit TBoxen, obere
Schranke}\label{komplexituxe4t-mit-tboxen-obere-schranke}

\subsubsection{Obere Schranke}

Wir wollen zeigen:

\begin{theorem}
In $\ALC$ ist die Erfüllbarkeit von Konzepten bzgl. TBoxen
ExpTime-Vollständig.
\end{theorem}

Mit Lemma 2.9: Subsumtion und Äquivalenz ExpTime-vollständig.

Wir beginnen mit der oberer Schranke (Enthaltensein in ExpTime):

\begin{itemize}
  \item wir verwenden ein Verfahren aus der Modallogik: Typelimination
  \item basiert auf \emph{syntaktischem} Typ-Begriff
\end{itemize}

\subsubsection{Syntaktische Typen}\label{synt-typ}

Wir nehmen oBdA. an, dass das Eingabe-Konzept $C_0$ in NNF ist und die Eingabe-TBox die Form $\{\top \sqsubseteq C_{\MT}\}$ hat mit $C_{\MT}$ in NNF.

\begin{definition}[Typ]

Ein Typ für $C_{0}$ und $T$ ist Teilmenge
$t \subseteq \sub\left( C_{0},T \right)$, so dass

\begin{enumerate}
\item
  $A \in t$ gdw. $\neg A \notin t$ für alle
  $\neg A \in \sub\left( C_{0},T \right)$
\item
  $C \sqcap D \in t$ gdw. $C \in t$ und $D \in t$ für alle
  $C \sqcap D \in \sub\left( C_{0},T \right)$
\item
  $C \sqcup D \in t$ gdw. $C \in t$ oder $D \in t$ für alle
  $C \sqcup D \in \sub\left( C_{0},T \right)$
\item
  $C_{T} \in t$
\end{enumerate}
\end{definition}

\begin{tafel}[Beispiel syntaktischer Typ]
Sei $C_0 = A$, $\MT = \{\top \subseteq C_{\MT}\}$ mit $$C_{\MT} = \exists r.\exists r.A \sqcap \forall r.A' \sqcap (\neg A \sqcup \neg A')$$

Dann ist $$\sub(C_0, \MT) = \{A, C_{\MT}, \exists r.\exists r.A, \forall r.A',\neg A \sqcup \neg A', \exists r.A, A', \neg A, \neg A'\}$$

Sei $M = \{C_{\MT}, \exists r.\exists r. A, \forall r.A', \neg A \sqcup \neg A'\}$

Die Typen für $C_0$ und $\MT$ sind:

\begin{tabular}{ll}
    $t_0 = M \cup \{\neg A, \neg A'\}$ &  $t_0' = M \cup \{ \neg A, \neg A', \exists r.A\}$\\
    $t_1 = M \cup \{\neg A, A'\}$ & $t_1' = M \cup \{\neg A, A', \exists r.A\}$ \\
    $t_2 = M \cup \{A, \neg A'\}$ & $t_2' = M \cup \{A, \neg A', \exists r.A\}$\\
\end{tabular}
\end{tafel}

\subsubsection{Typelimination}\label{typelimination}

Die generelle Idee der Typelimination bei Eingabe $C_0, \MT$:
\begin{itemize}
\item
  Generiere alle Typen für $C_{0}$ und $\MT$ (exponentiell viele)
\item
  Eliminiere wiederholt Typen, die in keinem Modell von $\MT$ vorkommen
  können
\item
  Überprüfe, ob ein Typ überlebt hat, der $C_{0}$ enthält
\item
    Wenn ja, antworte \enquote{erfüllbar}, sonst \enquote{unerfüllbar}
\end{itemize}

\subsubsection{Schlechte Typen}\label{schlechter-typ}

Wir formalisieren \enquote{Typen, die in keinem Modell vorkommen können}.

\begin{definition}[schlechter Typ]\label{def-badtype}

Sei $\Gamma$ Typenmenge und $t \in \Gamma$.

Dann ist $t$ \emph{schlecht in} $\Gamma$, wenn für ein $\exists r.C \in t$ gilt:

\begin{center}Es gibt kein $t^{'} \in \Gamma$ mit
$\left\{ C \right\} \cup \left\{ \text{D\ } \middle| \ \forall r.D \in t \right\} \subseteq t'$.\end{center}
\end{definition}

Erklärung: $t$ braucht einen \enquote{Zeugen}, es gibt aber keinen
geeigneten.

%TODO: fix this
\setcounter{tafel}{0}
\begin{tafel}[Fortsetzung]

    Beispielsweise ist $t_0'$ schlecht in der Menge $\{t_0,t_1,t_2,t_0',t_1',t_2'\}$. Für $\exists r.A \in t_0'$ ist die Menge aus \autoref{def-badtype} $\{A,A'\}$. Kein Typ enhält $A$ und $A'$. Analog: $t_1'$, $t_2'$ sind schlecht. Entfernt man diese drei so erhält man die Menge $\Gamma_1 = \{t_1,t_2,t_2\}$. 

Nun sind aber $\{t_1,t_2,t_3\}$ schlecht in $\Gamma_1$: Sie enthalten $\exists r. \exists r.A$, aber kein Typ in $\Gamma_1$ enthält $\exists r.A$.
\end{tafel}

\begin{algorithmic}
    \Procedure{$\ALC$-Elim}{}
    \State \textbf{input}: $\ALC$-Konzept $C_0$, $\ALC$-TBox $\MT$
    \State \textbf{output}: \enquote{erfüllbar} / \enquote{unerfüllbar}
    \State Berechne $\Gamma_0$: Menge aller Typen für $C_0$ und $\MT$
    \State $i := 0$
    \Repeat
        \State $i := i + 1$
        \State $\Gamma_i := \{ t \in \Gamma_{i - 1} \mid t \text{ nicht schlecht in } \Gamma_{i - 1}\}$
    \Until $\Gamma_i = \Gamma_{i - 1}$
    \If{es gibt $t \in \Gamma_i$ mit $C_0 \in t$}
        \Return \enquote{erfüllbar}
    \Else{}
        \Return \enquote{unerfüllbar}
    \EndIf
    \EndProcedure{}
\end{algorithmic}

%TODO: fix this
\setcounter{tafel}{0}
\begin{tafel}[Fortsetzung 2]
    Mit dem Beispiel aus T 5.1 enstehen folgende $\Gamma_i$:
    \begin{align*}
        \Gamma_0 &= \{t_0, t_1, t_2, t_0', t_1', t_2'\}\\
        \Gamma_1 &= \{t_0, t_1, t_2\}\\
        \Gamma_2 &= \emptyset
    \end{align*}
\end{tafel}

Wir müssen nun zeigen, dass der Algorithmus
\begin{enumerate}
    \item in exponentieller Zeit terminiert,
    \item korrekt ist,
    \item vollständig ist.
\end{enumerate}

\begin{lemma}
    $\ALC$-Elim($C_0,\MT$) terminiert nach $2^{\text{poly}(|C_0|+|\MT|)}$ Schritten.
\end{lemma}

\begin{proof}
Sei $n = \left| C_{0} \right| + |T|$. Dann gilt:

\begin{enumerate}
\item Es gibt nur $2^{n}$ Typen (\autoref{lem:num-sub} $|\sub(C, \MT)| \leq |C| + |\MT|$)
\item
  In jedem Schritt der repeat-Schleife wird mindestens ein Typ
  eliminiert; die Schleife terminiert nach maximal $2^{n}$
  Durchläufen.
\item
  Die restlichen Operationen (prüfen, ob ein Typ schlecht ist usw.)
  können leicht in Zeit $2^{\text{poly}\left( n \right)}$ implementiert werden.
\end{enumerate}
\end{proof}

\begin{lemma}
    Wenn $\ALC$-Elim($C_0,\MT$) \enquote{erfüllbar} zurückgibt, dann ist $C_0$ erfüllbar bzgl. $\MT$.
\end{lemma}

\begin{proof}
    Wenn der Algorithmus \enquote{erfüllbar} zurückgibt, dann gibt es in der resultierenden Typenmenge $\Gamma_i$ einen Typen $t_0 \in \Gamma_i$ mit $C_0 \in t_0$. Aus $\Gamma_i$ konstruieren wir Interpretation $\MI$ wie folgt:
\begin{align*}
    \Delta^\MI &= \Gamma_{i}\\
    A^\MI &= \{ t \in \Gamma_{i} \mid A \in t \} \text{ für alle Konzeptnamen } A\\
    r^\MI &= \{ (t,t') \mid \forall r.C \in t \text{ impliziert }C \in t' \} \text{ für alle Rollennamen } r
\end{align*} 

\begin{tafel}[Beispiel Modellkonstruktion]
    \begin{align*}
        C_0 &= A, \MT = \{ \top \sqsubseteq \forall r.\exists r.A \sqcap(\neg A \sqcup \exists r.A)\}\\
        \sub(C_0, \MT) &= \{ A, C_\MT, \forall r.\exists r.A, \neg A \sqcup \exists r.A, \exists r.A, \neg A\}\\
        M &= \{C_\MT, \forall r.\exists r.A, \neg A \sqcup \exists r.A\}\\
        t_0 &= M \cup \{\neg A\}\\
        t_1 &= M \cup \{\neg A, \exists r.A\}\\
        t_2 &= M \cup \{A, \exists r.A\}\\
    \end{align*}

    \begin{center}
    \begin{tikzpicture}
        \begin{scope}[every node/.style={circle,thick,draw}]
            \node (s1) at (1.5, 2) {$t_0$};
            \node (s2) at (0, 0) {$t_1$};
            \node (s3) at (3, 0) {$t_2$};
        \end{scope}
        \begin{scope}[every node/.style={fill=white}]
            \path [->] (s1) edge node {r} (s2);
            \path [->] (s1) edge node {r} (s3);
            \node at (s3) [right = 1mm of s3] {$A$};
            \path [->] (s2) edge [bend left] node {r} (s3);
            \path [->] (s3) edge [bend left] node {r} (s2);
            \path [->] (s2) edge [loop left] node {r} (s2);
            \path [->] (s3) edge [loop below] node {r} (s3);
        \end{scope}
    \end{tikzpicture}
    \end{center}
\end{tafel}

Die Erfüllbarkeit von $C_0$ bezüglich $\MT$ folgt dann aus:

Behauptung: Für alle $C \in \sub(C_0, \MT)$ und alle $t \in \Gamma_i$ gilt:
\begin{align*}
    C \in t \implies t \in C^\MI
\end{align*}

\setcounter{tafel}{1}
\begin{tafel}[Beweis der Behauptung]
    Die Behauptung impliziert das Lemma.
    \begin{enumerate}
        \item Wegen $C_0 \in t_0$ ist $t_0 \in C_0^\MI$
        \item Wegen $C_\MT \in t$ für alle $t \in \Gamma_i$ folgt $t \in C_\MT^\MI$ für alle $t \in \Gamma_i$. Also ist $\MI$ auch Modell von $\MT$.
    \end{enumerate}
    Beweis der Behauptung per Induktion über Struktur von $C$:

\textbf{I.A.}
\begin{itemize}
    \item $C = A$. Folgt direkt aus Definition $\MI$.
    \item $C = \neg A$. Wenn $\neg A \in t$, dann ist nach der Definition von Typ (a) $A \notin t$. Nach Definition von $\MI$ ist $t \in {(\neg A)}^\MI$.
\end{itemize}
\textbf{I.S.}
\begin{itemize}
    \item $C = D \sqcap D'$: Wenn $C \in t$, dann ist nach der Definition von  Typ (b) $D, D' \in t$. Nach IV ist $t \in D^\MI$ und $t \in D'^\MI$. Also ist $t$ auch Instanz von ${(D \sqcap D')}^\MI$
    \item $C = D \sqcup D'$: analog.

\item $C = \forall r.D$:
Sei $\forall r.D \in t$ und $\left( t,t' \right) \in r^\MI$. Nach
Definition $r^\MI$ gilt $D \in t'$. Nach I.V.:
$t' \in D^\MI$, also $t \in {( \forall r.D )}^\MI$

\item $C = \exists r.D$:
Sei $\exists r.D \in t$. Da $t \in \Gamma_{i}$ (also nicht
schlecht), gibt es $t' \in \Gamma_{i}$ mit $D \in t'$ und
$E \in t'$ für alle $\forall r.E \in t$. Nach I.V. gilt
$t' \in D^\MI$ und nach Definition von $r^\MI$ gilt
$\left( t,t' \right) \in r^\MI$. Also
$t \in {(\exists r.D )}^\MI$
\end{itemize}
\end{tafel}
\end{proof}


\begin{lemma}
    Wenn $C_0$ bezüglich $\MT$ erfüllbar ist, dann gibt $\ALC-\texttt{Elim}(C_0, \MT)$ \enquote{erfüllbar} zurück.
\end{lemma}

\begin{proof}
    Sei $C_0$ erfüllbar bezüglich $\MT$ und $\MI$ ein Modell von $C_0$ und $\MT$ mit $d_0 \in C_0^\MI$. Sei
    \begin{align*}
        \Gamma = \{ t_\MI(d) \mid d \in \Delta^\MI\}
    \end{align*}
    und $\Gamma_0, \Gamma_1, \ldots \Gamma_k$ die von $\ALC-\texttt{Elim}(C_0, \MI)$ erzeugte Sequenz.
    Wir zeigen:

    Behauptung: $\Gamma \subseteq \Gamma_i$ für alle $i \leq k$.
    \begin{tafel}[Beweis der Behauptung]
    Per Induktion über $i$:

    IA $i = 0$ trivial: jedes Element von $\Gamma$ (semantischer Typ) ist auch syntaktischer Typ (erfüllt a-d).

    IS $i \rightarrow i + 1$: Gelte $\Gamma \subseteq \Gamma_i$, zu zeigen $\Gamma \subseteq \Gamma_{i + 1}$.

    $t \in \Gamma \subseteq \Gamma_i$, es genügt zu zeigen, dass $t$ nicht schlecht in $\Gamma_i$. Sei $\exists r.C \in t$ und $S = \{C\} \cup \{D\mid \forall r.D \in t\}$. Da $t \in \Gamma$, gibt es $d \in \Delta^\MI$ mit $t_\MI(d) = t$. Also gibt es nach Semantik $e \in \Delta^\MI$ mit $(d, e) \in r^\MI$ und $e \in D'$ für alle $D \in S$.

    Es gilt also $S \subseteq t_\MI(e)$ und $t_\MI(e) \in \Gamma \subseteq \Gamma_i$. Also ist $t$ nicht schlecht in $\Gamma_i$.
    \end{tafel}
    Wegen $d_0 \in C_0^\MI$ ist also $C_0 \in t_\MI(d_0) \in \Gamma \subseteq \Gamma_k$. Also gibt der Algorithmus \enquote{erfüllbar} zurück.
\end{proof}


Aus den Lemmas 5.4--5.6 folgt nun:

\begin{theorem}
In $\ALC$ ist die Erfüllbarkeit von Konzepten bzgl. TBoxen entscheidbar in ExpTime.
\end{theorem}

\subsubsection{Zusammenhang mit Tableau-Algorithmen}

Offensichtliche Entsprechungen:

\begin{itemize}
  \item $\sqcap$-Regel, $\sqcup$-Regel, TBox-Regel finden sich wieder in der Definition eines Typs.
  \item $\exists$-Regel und $\forall$-Regel finden sich wieder in Definition von \enquote{schlecht}.
  \item Freiheit von offensichtlichen Widersprüchen findet sich wieder in der Definition eines Typs
\end{itemize}

Unterschiede:

\begin{itemize}
  \item Tableau-Algorithmus benötigt im Worst Case dreifach exponentielle Laufzeit
  \item Typelimination benötigt im Best Case exponentielle Laufzeit
\end{itemize}

\subsection{Komplexität mit TBoxen, untere Schranke}\label{komplexituxe4t-mit-tboxen-untere-schranke}

Um die ExpTime-Schwere zu zeigen, reduzieren wir auf ein spieltheoretisches Problem:

\subsubsection{ExpTime-Spiele}\label{exptime-spiele}

\begin{itemize}
\item Zwei Spielerinnen spielen auf gegebener aussagenlogischen Formel $\varphi$
\item Jede Variable in $\varphi$ gehört entweder Spielerin 1 oder Spielerin 2
\item Das Spiel beginnt auf einer gegebenen Anfangsbelegung $\pi_0$ der Variablen in $\varphi$.
\item Spielerin 1 beginnt, die Spielerinnen wechseln sich ab.
\item In jedem Zug ändert Spieler Wahrheitswert einer ihrer Variablen; es ist erlaubt, zu passen
\item Spielerin 1 gewinnt, wenn $\varphi$ jemals wahr wird (egal, welche Spielerin gezogen hat).
\item Spielerin 2 gewinnt, wenn das Spiel unendlich weitergeht ohne dass $\varphi$ wahr wird
\end{itemize}

\begin{tafel}[Beispiel ExpTime-Spiele]\mbox{}
    \begin{itemize}
        \item Spiel A: $\varphi = \neg (p_1 \leftrightarrow p_2) \wedge \neg (q_1 \leftrightarrow q_2)$,
            Sp1 kontrolliert $p_1, q_1$, Sp2 kontrolliert $p_2, q_2$. $\pi_0$ macht alle Variablen falsch.

            Sp1 gewinnt wenn $p_1, p_2$ unterschiedlich belegt sind und ebenso $q_1, q_2$. Sp2 gewinnt, wenn sie den letzten Zug von Sp1 immer \enquote{komplementiert}.
            Möglicher Spielverlauf:
            \begin{center}
            \begin{tabular}{rrrrr}
                & $p_1$ & $p_2$ & $q_1$ & $q_2$\\
                & 0 & 0 & 0 & 0\\
                1 & 1 & 0 & 0 & 0\\
                2 & 1 & 1 & 0 & 0\\
                1 & 1 & 1 & 1 & 0\\
                2 & 1 & 1 & 1 & 1\\
            \end{tabular}
        \end{center}

        $\implies$ Sp2 gewinnt.

    \item Spiel B: wie A $\varphi' = \varphi \vee (p_1 \wedge p_2 \wedge q_1 \wedge q_2)$.
        Möglicher Spielverlauf:
    \end{itemize}
\end{tafel}

\begin{definition}[ExpTime-Spiele]

\begin{itemize}
\item
  \emph{Spiel}: Tupel
  $\left( \varphi,\ \Gamma_1,\Gamma_2,\pi_{0} \right)$ mit
  $\Gamma_1,\Gamma_2$ Partitionierung der Variablen in
  $\varphi_1$ und $\pi_{0}$ Anfangsbelegung
\item
  \emph{Konfiguration}: Paar $(i,\pi)$ mit
  $i \in \left\{ 1,2 \right\}$ aktiver Spieler und $\pi$ Belegung
\item
  $\pi$ ist $j$-\emph{Variation} von $\pi'$
  $\left( j \in \left\{ 1,2 \right\} \right)$ wenn $\pi = \pi'$ oder
  $\pi$ und $\pi^{'}$ unterscheiden sich nur in einer Variablen
  $p \in \Gamma_{j}$
\end{itemize}
\end{definition}

„$\pi$ ist $j$-Variation von $\pi'$`` bedeutet: Spieler $j$ kann
$\pi$ in $\pi'$ transformieren (oder umgekehrt).

Das hier relevante Entscheidungsproblem bezieht sich auf \emph{Gewinnstrategien} für Spieler 2.

\subsubsection{Gewinnstrategie}\label{definition-5.8-gewinnstrategie}

Intuitiv:

\begin{itemize}
  \item eine Gewinnstrategie sagt Spieler 2 nach jedem möglichen Spielverlauf wie er spielen muss um zu gewinnen.
  \item wenn Spieler 2 eine Gewinnstrategie hat, so kann er das Spiel gewinnen
\end{itemize}

\begin{definition}[Gewinnstrategie]

Gewinnstrategie für Spieler 2 in Spiel
$\left( \varphi,\Gamma_1,\Gamma_2,\pi_{0} \right)$ ist unendlicher
knotenbeschrifteter Baum $(V,E,l)$, wobei $l$ jedem Knoten
$v \in V$ Konfiguration $l(v)$ zuweist so, dass

\begin{enumerate}
\def\labelenumi{\alph{enumi})}
\item
  Wurzel beschriftet mit $\left( 1,\pi_{0} \right)$
\item
  wenn $l\left( v \right) = (2,\pi)$, dann hat $v$ Nachfolger
  $v^{'}$ mit $l\left( v' \right) = \left( 1,\pi^{'} \right)$, wobei
  $\pi^{'}$ $2$-Variation von $\pi$
\item
  wenn $l\left( v \right) = (1,\pi)$, dann hat $v$ Nachfolger
  $v_{0},\ldots,v_{\left| \Gamma_1 \right|}$ mit
  $l\left( v_1 \right) = (2,\pi_{i})$ wobei
  $\pi_{0},\ldots,\pi_{\left| \Gamma_1 \right|}$ alle existierenden
  $1$-Variationen von $\pi$
\item
  wenn $l\left( v \right) = (i,\pi)$, dann nicht
  $\pi \models \varphi$
\end{enumerate}
\end{definition}

\subsubsection{ExpTime-Spiele als Entscheidungsproblem}

\begin{definition}

\emph{Spiel\textsubscript{1}} ist das folgende Problem: Gegeben Spiel
$\left( \varphi,\Gamma_1,\Gamma_2,\pi_1 \right)$, entscheide ob
Spieler 2 eine Gewinnstrategie hat.
\end{definition}

\begin{theorem}
Spiel\textsubscript{1} ist ExpTime-Vollständig
\end{theorem}

Wir wollen nun die ExpTime-Härte von Erfüllbarkeit in $\ALC$ bzgl. TBoxen auf Spiel\textsubscript{1} reduzieren

\subsubsection{Reduktion}\label{reduktion}

Reduziere Spiel\textsubscript{1}: Gegeben Spiel
$\left( \varphi,\Gamma_1,\Gamma_2,\pi_{0} \right)$, konstruiere in
Polynomialzeit Konzept $C_{S}$ und TBox $\MT_{S}$ so, dass: 

\begin{lemma}Spieler 2 hat Gewinnstrategie in $S$ gdw. $C_{S}$ erfüllbar bzgl. $\MT_{S}$.\end{lemma}

Idee: (Baum)-Modelle von $C_S$ und $\MT_S$ kodieren Gewinnstrategien.

Beweisskizze:

Details der Reduktion:

\includegraphics[width=3.71910in,height=2.33200in]{media/5red1.png}

\includegraphics[width=3.71910in,height=2.33200in]{media/5red2.png}

\includegraphics[width=3.71910in,height=2.33200in]{media/5red3.png}

Setze $C_S = W$ 

\begin{itemize}
\item
  Hinrichtung: Erzeuge aus der Gewinnstrategie Interpretation und zeige, dass diese Modell ist.
\item
  Rückrichtung: Nimm an es gibt Baummodell und erzeuge daraus
  Gewinnstrategie.
\end{itemize}

\begin{theorem}
In $\ALC$ ist die Erfüllbarkeit von TBoxen ExpTime-hart.
\end{theorem}

Daraus ergibt sich zusammen mit \protect\hyperlink{theorem-5.6}{Theorem 5.6} ExpTime-Vollständigkeit.

\subsection{Komplexität ohne TBoxen obere
Schranke}\label{komplexituxe4t-ohne-tboxen-obere-schranke}

\subsubsection{Obere Schranke}

Wir wollen zeigen:

\begin{theorem}
In $\ALC$ ist die Erfüllbarkeit von Konzepten (ohne TBoxen)
PSpace-Vollständig.
\end{theorem}

Mit Lemma 2.8 sind dann auch Subsumtion und Äquivalenz PSpace-vollständig.

\subsubsection{ALC-Worlds}\label{alc-worlds}

Wenn $C$ erfüllbar, dann hat $C$ ein Baummodell (Theorem 3.4). Ohne TBox ist dessen Tiefe mit $\left| C \right|$ beschränkt. 

In PSpace:

\begin{itemize}
\item
  Ein linear Tiefer Baum ist exponentiell groß
\item
  Gesamtes Modell im Speicher: nicht PSpace
\item
  Stattdessen Prüfe Existenz des Baumes mittels Tiefensuche; halte zu
  jeder Zeit nur einen Pfad des Baumes im Speicher
\end{itemize}

\begin{theorem}
PSpace $=$ NPSpace
\end{theorem}

\begin{definition}[$i$-Konzepte]

Für $i \geq 0$ ist die Menge der $i$-Konzepte definiert als:
$$sub_{i} := \left( C_{0} \right) \left\{ C \in \sub\left( C_{0} \right)\  \middle| \text{\ rd}\left( C \right) \leq i \right\}$$
\end{definition}

\begin{definition}[$i$-Typ]

Sei $i \geq 0$. $\MI$-Typ für $C_{0}$ ist Teilmenge
$t \subseteq \sub_{i}\left( C_{0} \right)$ so, dass

\begin{enumerate}
\def\labelenumi{\arabic{enumi}.}
\item
  $A \in t$ gdw. $\neg A \notin t$ für alle
  $\neg A \in \sub_{i}\left( C_{0} \right)$
\item
  $C \sqcap D \in t$ gdw. $C \in t$ und $D \in t$ für alle
  $C \sqcap D \in \sub_{i}\left( C_{0} \right)$
\item
  $C \sqcup D \in t$ gdw. $C \in t$ oder $D \in t$ für alle
  $C \sqcup D \in \sub_{i}\left( C_{0} \right)$
\end{enumerate}
\end{definition}

\paragraph{ALC-Worlds}\label{alc-worlds-1}

Rekursion über $\MI$-Typen.

\begin{proposition}
ALC-Worlds($C_{0}$) terminiert und benötigt polynomiellen Platz (in $|C_0|$).
\end{proposition}

Beweisskizze. Stelle als Rekursionsbaum dar. Verzweigungsgrad beschränkt durch Anzahl der $\exists$. Tiefe Beschränkt durch $rd(C_{0})$. Der Rekursionsstack hat höchstens Tiefe $\left| C_{0} \right|$ und der Platzbedarf pro Aufruf ist polynomiell.

\subsubsection{Korrektheit und Vollständigkeit}\label{proposition-5.18}

\begin{proposition}
$\ALC$-Worlds($C_0$) = true gdw. $C_0$ erfüllbar
\end{proposition}

Korrektheitsbeweis per Induktion über $C_{0}$:

Zunächst definieren wir eine Interpretation $\MI$. Für jeden Knoten $v \in V_0 \setminus \{v_0\}$ sei $\sigma (v)$ der Rollenname $r$ des Konzeptes $\exists r.C$, für das der Aufruf $v$ gemacht wurde.

Definiere $\MI$ nun wie folgt:

\begin{itemize}
  \item $\Delta^\MI = V$
  \item $r^\MI = \{(v,v')\in E | \sigma (v') = r\}$
  \item $A^\MI = \{v | A \in p_1(v)\}$
\end{itemize}

($p_1(v)$ = 1. Parameter in $l(v)$)

Behauptung: Für alle $v \in V$ und $C \in \sub(C_0)$ gilt $$C \in p_1(v)\ impliziert v \in C^\MI$$

Da $C_0 \in p_1(v_0)$ ist auch $v_0 \in C_0^\MI$; also ist $\MI$ ein Modell von $C_0$.

\textbf{T5.9a}

\textbf{I.A.} $C = A$. Folgt direkt aus Definition $\MI$.

$C = \neg A$. Da der Lauf erfolgreich ist, ist
$p_1\left( v \right)$ Typ für $C_{0}$. Nach Definition „Typ`` gilt
$A \notin p_1\left( v \right)$. Nach Definition $\MI$ ist
$v \notin A^\MI$.

\textbf{I.S.} Fallunterscheidung:

\begin{itemize}
\item
  $C = D \sqcap D^{'}$
\end{itemize}

Nach Definition „Typ`` ist $D \in p_1\left( v \right)$ und
$D^{'} \in p_1\left( v \right)$. Nach \textbf{I.V}.: $v \in D^\MI$
und $v \in \left( D^{'} \right)^\MI$. Nach Semantik:
$v \in \left( D \sqcap D^{'} \right)^\MI$.

\begin{itemize}
\item
  $C = D \sqcup D^{'}$
\end{itemize}

Analog.

\begin{itemize}
\item
  $C = \forall r.D$
\end{itemize}

Sei $\left( v,v^{'} \right) \in r^\MI$. Dann ist
$\left( v,v^{'} \right) \in E$ und $\sigma\left( v^{'} \right) = r$.
Wegen $\forall r.D \in p_1(v)$ ist auch
$D \in p_1\left( v^{'} \right)$. Nach I.V. gilt $v^{'} \in D$.
Also $v \in \left( \forall r.D \right)^\MI$

\begin{itemize}
\item
  $C = \exists r.D$
\end{itemize}

Ähnlich

\textbf{Beweis der Vollständigkeit}

Sei $C_0$ erfüllbar und $\MI$ ein Modell von $C_0$ und $d_0 \in C_0^{\MI.}$

Beweisidee:

Verwenden $\MI$, um die nichtdeterministischen Entscheidungen von $\ALC$-Worlds($C_0$) zu einem erfolgreichen Lauf zu \enquote{lenken}.

\begin{theorem}
In $\ALC$ ist die Erfüllbarkeit von Konzepten in PSpace.
\end{theorem}

\subsection{Komplexität ohne TBoxen untere
Schranke}\label{komplexituxe4t-ohne-tboxen-untere-schranke}

\begin{definition}[PSpace-Spiel]

\begin{itemize}
\item
  \emph{Spiel}: Aussagenlogische Formel $\varphi$ mit Variablen
  $p_1,\ldots,p_{n}$, $n$ gradzahlig
\item
  \emph{Konfiguration}: Wort $\pi \in \left\{ 0,1 \right\}*$
\end{itemize}
\end{definition}

\begin{definition}[Gewinnstrategie]

Gewinnstrategie für Spieler 1 in Spiel $\varphi$ ist endlicher
knotenbeschrifteter Baum $(V,E,L)$, wobei $l$ jedem Koten
$v \in V$ Konfiguration $l(v)$ zuweist, sodass
\end{definition}

\begin{enumerate}
\def\labelenumi{\alph{enumi})}
\item
  Wurzel beschriftet mit $\varepsilon$ (leere Konfiguration)
\item
  wenn $l\left( v \right) = w$ mit $\left| w \right|$ gerade und
  $\left| w \right| < n$ (Also Spieler 1 am Zug), dann hat $v$
  Nachfolger $v^{'}$ mit $l\left( v^{'} \right) \in \{ w0,w1\}$
\item
  wenn $l\left( v \right) = w$ mit \textbar{}w\textbar{} ungerade
  (also Spieler 2 am Zug), dann hat $v$ Nachfolger $v^{'}$ und
  $v''$mit $l\left( v^{'} \right) = w0$ und
  $l\left( v^{''} \right) = w1$
\item
  wenn $l\left( v \right) = w$ mit $\left| w \right| = n$, dann
  $w \models \varphi$
\end{enumerate}

\subsubsection{Theorem 5.25}\label{theorem-5.25}

In $\ALC$ ist die Erfüllbarkeit von Konzepten bzgl. leerer TBoxen
PSpace-Hart.

Beweisskizze. Konstruiere Konzept $C_{\varphi}$, so dass Spieler 1 hat
Gewinnstrategie in $\varphi$ gdw. $C_{\varphi}$ erfüllbar.

\subsection{Unentscheidbare
Erweiterungen}\label{unentscheidbare-erweiterungen}

\subsection{Konkrete Bereiche}\label{konkrete-bereiche}

Ein \emph{Konkreter Bereich} ist ein Paar $B = (\Delta^{B},\Phi^{B})$
wobei

\begin{itemize}
\item
  $\Delta^{B}$ eine Menge von \emph{Werten} ist und
\item
  $\Phi^{B}$ eine Menge von \emph{Prädikaten}
\end{itemize}

sodass jedes $P \in \Phi^{B}$ mit einer Stelligkeit $n \geq 0$
ausgestattet ist und mit einer Extension
$P^{B} \subseteq \left( \Delta^{B} \right)^{n}$.

\subsubsection{Definition 5.27 (ALCB
Syntax)}\label{definition-5.27-alcb-syntax}

Sei $B$ ein konkreter Bereich. Mit ALC($B$) bezeichnen wir die
Erweiterung von $\ALC$ um $B$, d.h. um

\begin{itemize}
\item
  \emph{Featurenamen} (eine zusätzliche Art von Rolle) und
\item
  die Konstruktoren $\exists R_1,\ldots,\ R_{n}\text{.P}$ und
  $\forall R_1,\ldots,R_{n}\text{.P}$
\end{itemize}

wobei $P \in \Phi^{B}$ $n$-Stellig ist und die $R_{i}$
\emph{Rollenkomposition} der Form $r_1;\ldots;r_{k};f$ sind mit
$r_{j}$ Rollenname und $f$ Featurename.

\subsubsection{Definition 5.28 (ALCB
Semantik)}\label{definition-5.28-alcb-semantik}

Eine Interpretation $\MI$ ordnet nun zusätzlich zu jedem Featurenamen
$f$ eine Funktion $f^\MI:\Delta^\MI \rightarrow \Delta^{B}$ zu. Für
jede Rollenkomposition $r = r_1;\ldots r_{k};f$ bezeichnet $R^\MI$
die Komposition der Interpretationen:
$R^\MI = r_1^\MI \circ \ldots \circ r_{k}^\MI \circ f$

Die Semantik der zusätzlichen Konstruktoren ist nun:

\[\left( \exists R_1,\ldots,\ R_{k}\text{.P} \right)^\MI = \left\{ d \in \Delta^\MI\ |\ \exists d_1,\ldots,d_{k}:\left( d,d_{i} \right) \in R_{i}^\MI\ \mathrm{fur}\ 1 \leq i \leq k\ \mathrm{\text{und}}\ \left( d_1,\ldots,\ d_{k} \right) \in P^{B} \right\}\]

\[\left( \forall R_1,\ldots,\ R_{k}\text{.P} \right)^\MI = \left\{ d \in \Delta^\MI\ |\ \forall d_1,\ldots,d_{k}:\left( d,d_{i} \right) \in R_{i}^\MI\ \mathrm{fur}\ 1 \leq i \leq k\ \mathrm{\text{impliziert}}\ \left( d_1,\ldots,\ d_{k} \right) \in P^{B} \right\}\]

\subsubsection{2-Registermaschinen}\label{registermaschinen}

\begin{itemize}
\item
  Endlich viele Zustände
\item
  Zwei Register mit Werten aus $\mathbb{N}$
\item
  Instruktionen um

  \begin{itemize}
  \item
    Register zu inkrementieren
  \item
    Register auf null zu testen und bei Wert $\neq 0$ zu
    dekrementieren. Der Folgezustand hängt davon ab, ob das Register
    $0$ war.
  \end{itemize}
\end{itemize}

\subsubsection{Definition 5.30}\label{definition-5.30}

(Deterministische) \emph{2-Registermaschine} (2RM) ist Paar
$M = \left( Q,P \right)$ mit
$Q = \left\{ q_{0},\ldots,\ q_{l} \right\}$ Menge von \emph{Zuständen}
und $P = I_{0},\ldots,I_{l - 1}$ \emph{Instruktionsfolge}. Per
Definition ist $q_{0}$ Startzustand, $q_{l}$ Stoppzustand. Jede
Instruktion ist $I_{i}$ hat eine der folgenden Formen:

\begin{itemize}
\item
  $I_{i} = + (p,q_1)$ mit $p \in \left\{ 1,2 \right\}$
  \emph{Register} und $q_{j}$ Folgezustand: Inkrementierunsanweisung
\item
  $I_{i} = - (p,q_{j},q_{k})$ mit $p \in \left\{ 1,2 \right\}$
  Register und $q_{j},q_{k}$ Folgezustände: Dekrementierungsanweisung
  mit Folgezustand $q_{j}$, wenn Register $p$ den Wert $0$ enthält
  und $q_{k}$ sonst.
\end{itemize}

\subsubsection{Definition 5.31}\label{definition-5.31}

Konfiguration und Konfigurationsübergänge
$\left( q,m,n \right) \vdash_{M}(q^{'},m',n^{'})$. Berechnung als
eindeutige längste Konfigurationsfolge.

\subsubsection{Theorem 5.29}\label{theorem-5.29}

Das Erfüllbarkeitsproblem in $ALC(B_1)$ ist unentscheidbar.

\begin{itemize}
\item
  $\Delta^{B_1}\mathbb{= N}$
\item
  $\Phi^{B_1} = \left\{ =_{0}, = , +_1 \right\}$, wobei $=_{0}$
  einstellig, die anderen Zweistellig.
\end{itemize}

Beweisskizze. Gegeben 2RM $M$,
konstuiere$\text {$\ALC$}\left( B_1 \right)$-TBox $T_{M}$ und wähle
einen Konzeptnamen $J$ sodass: $M$ hält auf $\left( 0,0 \right)$
gdw. $J$ unerfüllbar bzgl. $T_{M}$. Zeige dies jeweils für
Hinrichtung und Rückrichtung per Kontraposition. \textbf{Wo kommen i und
j her?}
