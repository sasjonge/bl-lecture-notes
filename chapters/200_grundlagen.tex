\subsection{Attributive Language with Complement
(ALC)}\label{attributive-language-with-complement-alc}

\subsubsection{Konzept- und Rollennamen}\label{konzept--und-rollennamen}

Konzept- und Rollennamen abzählbar unendliche und disjunkte (durch Groß-
und Kleinschreibung) Mengen.

\subsubsection{Konzepte}\label{konzepte}

Rekursive Definition über Konzeptnamen, Negation, Konjunktion,
Disjunktion, Existenz---und Werterestriktion. Darüber hinaus Top-Konzept
und Bottom-Konzept.

\subsubsection{Interpretation}\label{interpretation}

$I$ ist Paar $\left( \Delta^{I}, \cdot^{I} \right)$ mit

\begin{itemize}
\item
  $\Delta^{I}$ nicht-leere Menge (\emph{Domäne})
\item
  $\cdot^{I}$ Interpretationsfunktion bildet ab:

  \begin{itemize}
  \item
    jeden Konzeptnamen $A$ auf Menge $A^{I} \subseteq \Delta^{I}$
  \item
    jeden Rollennamen $r$ auf Relation
    $r^{I} \subseteq \Delta^{I} \times \Delta^{I}$
  \end{itemize}
\end{itemize}

Außerdem rekursive Definition für Zusammengesetzte Konzeptnamen.

\subsubsection{Terminologien}\label{terminologien}

\begin{itemize}
\item
  $C^{I}$ ist \emph{Extension} des Konzeptes oder der Rolle $C$
\item
  jedes $d \in C^{I}$ ist eine Instanz des Konzeptes $C$
\item
  $r$-Nachfolger, $r$-Vorgänger
\end{itemize}

\hypertarget{erfuxfcllbarkeit-subsumtion-uxe4quivalenz}{\subsubsection{Erfüllbarkeit,
Subsumtion,
Äquivalenz}\label{erfuxfcllbarkeit-subsumtion-uxe4quivalenz}}

Definition: Seien $C$ und $D$ ALC-Konzepte. Dann

\begin{itemize}
\item
  ist $C$ \emph{erfüllbar}, wenn es eine Interpretation $I$ gibt mit
  $C^{I} \neq \varnothing$. $I$ ist dann ein \emph{Modell} von
  $C$.
\item
  wird $C$ von $D$ \emph{subsumiert}, wenn $C^{I} \subseteq D^{I}$
  in allen Interpretationen $I$ (Notation $C \sqsubseteq D$)
\item
  sind C und D \emph{äquivalent}, wenn $C = D$ in allen
  Interpretationen $I$ (Notation $C \equiv D$)
\end{itemize}

Es gelten die üblichen aussagenlogischen Äquivalenten, wie z.B. de
Morgan.

\subsection{TBoxen}\label{tboxen}

TBoxen definieren Konzepte und setzten diese zueinander in Beziehung.

\subsubsection{Definition}\label{definition-1}

\emph{Konzeptinklusion} ist Ausdruck $C \sqsubseteq D$. \emph{TBox}
ist endliche Menge von Konzeptinklusionen. $C \equiv D$ ist Abkürzung
für $C \sqsubseteq D$, $D \sqsubseteq C$. Eine Interpretation $I$

\begin{itemize}
\item
  erfüllt Konzeptinklusion $C \sqsubseteq D$ gdw.
  $C^{I} \subseteq D^{I}$
\item
  ist Modell von TBox $T$ gdw. $I$ alle Konzeptinklusionen in $T$
  erfüllt
\end{itemize}

\subsubsection{Modellierung}\label{modellierung}

\begin{itemize}
\item
  Konzeptinklusion $A \sqsubseteq C$: $C$ ist notwendige Bedingung
  dafür, eine Instanz von $A$ zu sein
\item
  Konzeptdefinition $A \equiv C$: \emph{C} ist hinreichende Bedingung
  dafür, eine Instanz von $A$ zu sein
\item
  Disjunktheitsconstraints $C \sqcap D \sqsubseteq \bot$: Kein Objekt
  kann gleichzeitig zu $C$ und $D$ gehören
\end{itemize}

\subsubsection{Erfüllbarkeit, Subsumtion,
Äquivalenz}\label{erfuxfcllbarkeit-subsumtion-uxe4quivalenz-1}

Wie in
\protect\hyperlink{erfuxfcllbarkeit-subsumtion-uxe4quivalenz}{2.1.5},
nur ist $I$ jeweils Modell von einer TBox $T$.

\begin{itemize}
\item
  Erfüllbarkeit zeigen: Modell angeben
\item
  Unerfüllbarkeit / Subsumption zeigen: semantisch Argumentieren
\item
  Nicht-Subsumtion zeigen: Gegenmodell angeben
\end{itemize}

\subsubsection{Monotonie}\label{monotonie}

\textbf{Lemma 2.7}: Seinen $T_{1}$ und $T_{2}$ TBoxen mit
$T_{1} \subseteq T_{2}$. Dann gilt:

\begin{enumerate}
\def\labelenumi{\arabic{enumi}.}
\item
  Wenn $C$ erfüllbar bzgl. $T_{2}$, dann ist $C$ erfüllbar bzgl.
  $T_{1}$.
\item
  Wenn $T_{1} \vDash C \sqsubseteq D$, dann
  $T_{2} \vDash C \sqsubseteq D$.
\end{enumerate}

Beweisskizze.

\begin{enumerate}
\def\labelenumi{\arabic{enumi}.}
\item
  Sei $C$ erfüllbar bzgl. $T_{2}$. Dann gibt es Modell $I$ von
  $T_{2}$ mit $C^{I} \neq \varnothing$. Also erfüllt $I$ alle
  Konzeptinklusionen in $T_{2}$ und wegen $T_{1} \sqsubseteq T_{2}$
  auch alle in $T_{1}$.
\item
  Kontraposition, analog zu 1.
\end{enumerate}

\subsection{Schlussfolgerungsprobleme}\label{schlussfolgerungsprobleme}

Die Probleme sind wechselseitig aufeinander reduzierbar:

\begin{itemize}
\item
  \emph{Erfüllbarkeitsproblem} auf Nicht-Äquivalenz
\end{itemize}

$C$ erfüllbar bzgl. $T$ gdw. $\neg T \vDash C \equiv \bot$

\begin{itemize}
\item
  \emph{Subsumtionsproblem} auf Unerfüllbarkeit
\end{itemize}

$T \vDash C \sqsubseteq D$ gdw. $C \sqcap \neg D$ unerfüllbar bzgl.
$T$

\begin{itemize}
\item
  \emph{Äquivalenzproblem} auf Subsumtion
\end{itemize}

$T \vDash C \equiv D$ \emph{gdw.}
$T \vDash \top \sqsubseteq \left( C \sqcap D \right) \sqcup \left( \neg C \sqcap \neg D \right)$

Motivation hinter den Problemen:

\begin{itemize}
\item
  Modellierungsfehler finden
\item
  Struktur der TBox explizit machen
\item
  Redundanzen finden
\end{itemize}

\subsubsection{Subsumtion als
Ordnungsrelation}\label{subsumtion-als-ordnungsrelation}

\textbf{Lemma 2.8}: Für jede TBox $T$ ist die Relation
„$\sqsubseteq$ bzgl. $T$``

\begin{itemize}
\item
  reflexiv ($T \vDash C \sqsubseteq C$) und
\item
  transitiv ($T \vDash C \sqsubseteq D$ und
  $T \vDash D \sqsubseteq E$ impliziert $T \vDash C \sqsubseteq E$).
\end{itemize}

Bis auf fehlende Antisymmetrie ist $\sqsubseteq$ also partielle
Ordnung (Hasse-Diagramm).

\subsection{Erweiterungen von ALC}\label{erweiterungen-von-alc}

\subsubsection{Inverse Rollen (ALCI)}\label{inverse-rollen-alci}

Definition: Für jeden Rollennamen $r$ ist $r^{-}$ die \emph{inverse
Rolle} zu $r$. Wir definieren
$\left( r^{-} \right)^{I} = \left\{ \left( e,d \right)\  \right|\ \left( d,e \right) \in r^{I}\}$

\subsubsection{Zahlenrestriktion (ALCQ)}\label{zahlenrestriktion-alcq}

Definition: Für jede natürliche Zahl $n$, jeden Rollennamen $r$ und
jedes Konzept $C$:

\begin{itemize}
\item
  Höchstens-Restriktion:
  $\left( \leq n\ r\ C \right)^{I} = \left\{ d \in \Delta^{I}\ |\ \#\left\{ e\ |\ \left( d,e \right) \in r^{I} \land e \in C^{I} \right\} \leq n \right\}$
\item
  Mindestens-Restriktion
  $( \leq n\ r\ C) = \left\{ d \in \Delta^{I}\ |\ \#\left\{ e\ |\ \left( d,e \right) \in r^{I} \land e \in C^{I} \right\} \geq n \right\}$
\end{itemize}