\theoremstyle{definition}
\begin{definition}{$\ALC$-Konzepte} \\
	Die Menge der $\ALC$-Konzepte is induktiv definiert:
	\begin{itemize}
		\item Jeder Konzeptname ist $\ALC$-Konzept
		\item Wenn $C$, $D$ $\ALC$-Konzepte, so auch
		\begin{itemize}
			\item $\neg C$ \tabto{2cm}(Negation)
			\item $C \sqcap D$ \tabto{2cm}(Konjunktion)
			\item $C \sqcup D$ \tabto{2cm}(Disjunktion)
		\end{itemize}
		\item {Wenn $C$ $\ALC$-Konzept und $r$ Rollenname, so sind
		\begin{itemize}
			\item $\exists r.C$ \tabto{2cm}(Existenzrestriktion)
			\item $\forall r.C$ \tabto{2cm}(Werterestriktion)
		\end{itemize}
		$\ALC$-Konzepte}
	\end{itemize}
\end{definition}

\textbf{T2.1 Beispiel}

Hier einige Beispiel für diese Syntax:

\begin{center}
$Student \sqcap \exists studiert.Naturwissenschaft$ \\
$Professor \sqcap Emeritus \sqcap \forall haelt.\neg PlichtVL$ \\
$VL \sqcap \neg PlichtVL \sqcap \forall hatUebungsaufgabe(Einfach \sqcup Interessant)$ \\
$A \sqcap \exists r.(\neg B \sqcup \forall r.A)$
\end{center}

\textbf{Weiteres zur Syntax}

Dabei verwenden wir folgende Symbole:

\begin{itemize}
	\item{$A$,$B$ für Konzeptnamen}
	\item{$C$,$D$ für zusammengesetzte Konzepte}
	\item{$r$,$s$ für Rollennamen}
\end{itemize}

Zudem benutzen zudem folgende Abkürzungen: wir schreiben

\begin{itemize}
	\item{$\top$ für $A \sqcup \neg A$}
	\item{$\perp$ für $A \sqcap \neg A$}
\end{itemize}

\textbf{Präzedenzregel}

\begin{itemize}
	\item{$\neg$,$\exists$,$\forall$ binden stärker als $\sqcap$ und $\sqcup$}
\end{itemize}

Also zum Beispiel steht $\forall r.(\exists r.A \sqcap B)$ für $\forall r.((\exists r.A) \sqcap B)$ und nicht für $\forall r.(\exists r.(A \sqcap B))$.

Desweiteren is keine Präzedenz zwischen $\sqcap$ und $\sqcup$ definiert worden: Daher müssen Klammern verwendet werden!

\begin{definition}{$\ALC$ Semantik} \\
Eine \emph{Interpretation} $\MI$ ist Paar ($\Delta^{\MI}$,$\cdot^{\MI}$) mit
	\begin{itemize}
		\item{$\Delta^{\MI}$ nicht leere Menge (\emph{Domäne})}
		\item{$\cdot^{\MI}$ \emph{Interpretationsfunktion} bildet ab:
		 \begin{itemize}
			 \item{jeden Konzeptnamen $A$ auf Menge $A^{\MI} \subseteq \Delta^{\MI}$}
			 \item{jeden Rollennamen $r$ auf Relation $r^{\MI} \subseteq \Delta^{\MI} \times \Delta^{\MI}$}
		 \end{itemize}}
	\end{itemize}
\end{definition}
