
\subsection{\texorpdfstring{$\EL$}{EL}}\label{sec:el}

Für manche Anwendungen ist $\ALC$ zu komplex:
\begin{itemize}
    \item Auch hoch-optimierte Reasoner können sehr große und komplexe Ontologien oft nicht verarbeiten (oder nur nach intensivem Tuning)
    \item In der Anfragebeantwortung muss man oft mit sehr großen Datenmengen umgehen und braucht schnelle Antworten (\autoref{sec:abox})
\end{itemize}

Wir betrachten die Beschreibungslogik $\EL$:
\begin{itemize}
    \item viel weniger ausdrucksstark als $\ALC$, Basisoperatoren nur $\sqcap$ und $\exists r.C$
    \item Erfüllbarkeit und Subsumtion in Polyzeit entscheidbar.
\end{itemize}

\begin{definition}[$\EL$]
Ein $\EL$-Konzept ist ein $\ALC$-Konzept, in dem nur die Konstruktoren
$\top, \sqcap$ und $\exists r.C$ verwendet werden.
\end{definition}

$\EL$ ist beliebt für biomedizinische Ontologien (die sind oft sehr groß und mit hohem Abstraktionsgrad):
\begin{align*}
    \text{Perikardium} &\sqsubseteq \text{Gewebe} \sqcap \exists \text{teilVon}.\text{Herz}\\
    \text{Perikarditis} &\equiv \text{Entzündung} \sqcap \exists \text{ort}.\text{Perikardium}\\
    \text{Entzündung} & \sqsubseteq \text{Krankheit} \sqcap \exists \text{wirktAuf}.\text{Gewebe}
\end{align*}

SNOMED CT ist in unwesentlicher Erweiterung von $\EL$ formuliert. $\EL$ ist Grundlage des EL-Profils von OWL 2.

\subsubsection{Simulation}\label{simulation}

Intuitiv: $\EL$ ist die \enquote{Hälfte von $\ALC$}; Simulation entspricht der \enquote{Hälfte von Bisimulation}.

\begin{definition}[Simulation]
    Seien $\MI_1$ und $\MI_2$ Interpretationen. Relation $\rho \subset \Delta^{\MI_1} \times \Delta^{\MI_2}$ ist Simulation von $\MI_1$ nach $\MI_2$, wenn gilt:
    \begin{enumerate}
        \item Wenn $d_1 \rho d_2$, dann gilt für alle Konzeptnamen $A$:
            \begin{center}
                $d_1 \in A^{\MI_1}$ impliziert $d_2 \in A^{\MI_2}$
            \end{center}
        \item Wenn $d_1 \rho d_2$ und $(d_1, d'_1) \in r^{\MI_1}$ für beliebigen Rollennamen $r$, dann gibt es ein $d'_2 \in \Delta^{\MI_2}$ mit $d'_1 \rho d'_2$ und $(d_2, d'_2) \in r^{\MI_2}$.
    \end{enumerate}
\end{definition}

\begin{tafel}[Beispiele für Simulationen]\mbox{}
    \begin{center}
        \begin{tikzpicture}
        \begin{scope}[every node/.style={circle,draw,thick}]
            \node (s0) at (0, 1) {};
            \node (s1) at (0, 0) {};
            \node (s2) at (-1, -1) {};
            \node (s3) at (1, -1) {};
            \node (s4) at (4, 1) {};
            \node (s5) at (3, -0) {};
            \node (s6) at (5, -0) {};
            \node (s7) at (3, -1) {};
        \end{scope}
        \begin{scope}[every node/.style={fill=white}]
            \node (name) at (0, 2) {$\MI_1$};
            \node (name) at (4, 2) {$\MI_2$};
            \node [left = 0.3cm of s0] at (s0){$A$};
            \node [left = 0.3cm of s2] at (s2){$A$};
            \node [left = 0.3cm of s3] at (s3){$B$};
            \node [right = 0.3cm of s4] at (s4){$A$};
            \node [right = 0.3cm of s6] at (s4){$A, B$};
            \node [right = 0.3cm of s7] at (s7){$A, B, C$};
            \path [->] (s0) edge node {\scriptsize{r}} (s1);
            \path [->] (s1) edge node {\scriptsize{r}} (s2);
            \path [->] (s1) edge node {\scriptsize{r}} (s3);
            \path [->] (s4) edge node {\scriptsize{r}} (s5);
            \path [->] (s4) edge node {\scriptsize{s}} (s6);
            \path [->] (s5) edge node {\scriptsize{r}} (s7);
            \path[dotted] [->] (s0) edge node {\scriptsize{$\rho$}}(s4);
            \path[dotted] [->] (s1) edge node {\scriptsize{$\rho$}}(s5);
            \path[dotted] [->] (s2) edge [bend right] node {\scriptsize{$\rho$}}(s7);
            \path[dotted] [->] (s3) edge node {\scriptsize{$\rho$}}(s7);
        \end{scope}
    \end{tikzpicture}
\end{center}
\begin{center}
        \begin{tikzpicture}
        \begin{scope}[every node/.style={circle,draw,thick}]
            \node (s0) at (1, 0) {};
            \node (s1) at (0, 1) {};
            \node (s2) at (-1, 0) {};
            \node (s3) at (0, -1) {};
            \node (s4) at (4, 0) {};
        \end{scope}
        \begin{scope}[every node/.style={fill=white}]
            \node (name) at (0, 2) {$\MI_1$};
            \node (name) at (4, 2) {$\MI_2$};
            \node [left = 0.3cm of s0] at (s0){};
            \node [left = 0.3cm of s1] at (s1){$A$};
            \node [left = 0.3cm of s2] at (s2){$A, B$};
            \node [left = 0.3cm of s3] at (s3){$B$};
            \node [right = 0.3cm of s4] at (s4){$A, B$};
            \path [->] (s0) edge node {\scriptsize{s}} (s1);
            \path [->] (s1) edge node {\scriptsize{r}} (s2);
            \path [->] (s2) edge node {\scriptsize{s}} (s3);
            \path [->] (s3) edge node {\scriptsize{r}} (s0);
            \path [->] (s4) edge [looseness=10] node [loop above] {\scriptsize{r, s}} (s4);
            \path[dotted] [->] (s0) edge node {\scriptsize{$\rho$}}(s4);
            \path[dotted] [->] (s1) edge node {\scriptsize{$\rho$}}(s4);
            \path[dotted] [->] (s2) edge [bend right] node {\scriptsize{$\rho$}}(s4);
            \path[dotted] [->] (s3) edge node {\scriptsize{$\rho$}}(s4);
        \end{scope}
    \end{tikzpicture}
    \end{center}
    Wenn in $\MI_1$ nur $A, B, r, s$ verwendet werden, dann gilt für jede Interpretation $\MI$ und jedes Element $d \in \Delta^\MI: (\MI, d) \precsim (\MI, e)$.
\end{tafel}

Beachte: im Gegensatz zu Bisimulation sind Simulationen gerichtet.

Seien $\MI_1$ und $\MI_2$ Interpretationen, $d_1 \in \Delta^{\MI_1}, d_2 \in  \Delta^{\MI_2}$. Wir schreiben  $(\MI_1,d_1) \precsim (\MI_2,d_2)$, wenn es eine Simulation $\rho$ von $\MI_1$ nach $\MI_2$ gibt mit $d_1 \rho d_2$ (wir sagen: $d_1$ wird simuliert von $d_2$).

\begin{theorem}
Seien $\MI_1,\MI_2$ Interpretationen, $d_1 \in \Delta^{\MI_1}$ und
$d_2 \in \Delta^{\MI_2}$. Wenn
$(\MI_1,d_1) \precsim (\MI_2,d_2)$, dann
gilt für alle $\EL$-Konzepte $C$:
\begin{center}
$d_1 \in C^{\MI_1}$ impliziert $d_2 \in C^{\MI_2}$.
\end{center}
\end{theorem}
\begin{tafel}[Beweis des Theorems 6.3]
    \begin{proof}
    per Induktion über $C$:
    \begin{itemize}
        \item $C=\top$: trivial
        \item $C=A$: nach Bedingung 1 einer Simulation
        \item $C=D\sqcap E$: nach IV und Semantik
        \item $C=\exists r.D$: nach Bedingung 2 einer Simulation und IV
    \end{itemize}
\end{proof}
\end{tafel}

\subsubsection{Lemma 6.4}\label{lemma-6.4}

Bisimulation und wechselseitige Simulation sind nicht dasselbe.

Beweisskizze: Zeige Wechselseitige Simulation, die keine Biosimulation
ist:

\includegraphics[width=3.71910in,height=1.83200in]{media/image1.png}

\subsubsection{Lemma 6.6}\label{lemma-6.6}

Jedes EL-Konzept ist erfüllbar bzgl. jeder TBox.

Beweisskizze per Induktion über die Struktur von $C$:

\includegraphics[width=2.50460in,height=1.15200in]{media/image2.png}

\subsection{Subsumtion ohne TBox}\label{subsumtion-ohne-tbox}

Eine Subsumption $C \sqsubseteq D$ gilt in EL im Prinzip gdw mann
$D$ syntaktisch in $C$ „wiederfindet``. Werden Konzepte als
Baummodell dargestellt entspricht „Wiederfinden`` Simulation von
$D$-Baum in $C$-Baum (also Teilgraphenproblem).

\subsubsection{Definition kanonisches
Modell}\label{definition-kanonisches-modell}

Baue aus dem gegebenen Konzept $C$ intuitiv Baummodell.

\hypertarget{lemma-6.8}{\subsubsection{Lemma 6.8}\label{lemma-6.8}}

Für alle EL-Konzepte $C$ gilt: Die Interpretation $I_{C}$ ist Modell
von $C$ mit $d_{w} \in C^{I_{C}}$.

Beweisskizze per Induktion über die Struktur von $C$.

\hypertarget{lemma-6.9}{\subsubsection{Lemma 6.9}\label{lemma-6.9}}

Für alle EL-Konzepte $C$, Interpretation $I$ und
$e \in \Delta^{I}$ gilt: $e \in C^{I}$ gdw.
$\left( I_{C},d_{w} \right) \precsim \left( I,e \right)$.

Beweisskizze.

\begin{itemize}
\item
  Hinrichtung per Induktion über $C$. Schau jeweils Simulation nach
  I.V. an und ergänze diese.
\item
  Rückrichtung. Angenommen
  $\left( I_{C},d_{w} \right) \precsim \left( I,e \right)$.
  \protect\hyperlink{lemma-6.8}{Lemma 6.8} liefert
  $d_{w} \in C^{I_{C}}$. Nach \protect\hyperlink{lemma-6.3}{Theorem
  6.3} ist $e \in C^{I}$.
\end{itemize}

\subsubsection{Lemma 6.10}\label{lemma-6.10}

Für alle EL-Konzepte $C$, $D$ gilt: $C \sqsubseteq D$ gdw.
$\left( I_{D},d_{w} \right) \precsim \left( I_{C},d_{w} \right)$

Beweisskizze.

\begin{itemize}
\item
  Hinrichtung: Betrachte kanonisches Modell $I_{C}$ von $C$. Wegen
  Lemma 6.8 gibt es $d_{w} \in C^{I_{C}}$. Mit $C \sqsubseteq D$
  folgt $d_{W} \in D^{I_{C}}$. Mit \protect\hyperlink{lemma-6.9}{Lemma
  6.9} folgt für $D,\ I_{C},d$ dass $d_{w} \in D^{I_{C}}$ gdw.
  $\left( I_{D},d_{w} \right) \precsim \left( I_{C},d_{w} \right)$.
\item
  Rückrichtung. Angenommen
  $\left( I_{D},d_{w} \right) \precsim \left( I_{C},d_{w} \right)$.
  Betrachte beliebige Interpretation $I$ und $d \in C^{I}$. Zu
  zeigen: $d \in D^{I}$. Wegen $d \in C^{I}$ und
  \protect\hyperlink{lemma-6.9}{Lemma 6.9} gilt
  $\left( I_{C},d_{w} \right) \precsim \left( I,d \right)$. Verkette
  die Simulationen so dass
  $\left( I_{D},d_{w} \right) \precsim \left( I,d \right)$. Mit
  \protect\hyperlink{lemma-6.9}{Lemma 6.9} folgt
  $d_{w} \in D^{I_{C}}$.
\end{itemize}

\subsubsection{Theorem 6.11}\label{theorem-6.11}

Subsumtion in EL kann in polynomieller Zeit entschieden werden:

\begin{itemize}
\item
  Konstruiere $I_{C}$ und $I_{D}$ in polynomieller Zeit.
\item
  Überprüfe in polynomieller Zeit, ob
  $\left( I_{D},d_{w} \right) \precsim \left( I_{C},d_{w} \right)$

  \begin{itemize}
  \item
    Berechne maximale Simulation $\varsigma$
  \item
    Teste ob $\left( d_{w},d_{W} \right) \in \varsigma$
  \end{itemize}
\end{itemize}

\subsection{Subsumption mit TBox}\label{subsumption-mit-tbox}

\subsubsection{Lemma 6.12}\label{lemma-6.12}

Seien $C,D$ zwei beliebige $\text{EL}$-Konzepte und $T$ eine
EL-TBox. Sei weiterhin
$T^{'} = T \cup \left\{ A_{C} \sqsubseteq C,D \sqsubseteq A_{D} \right\}$.
mit Konzeptnamen $A_{C},A_{D}$, die nicht in $C,D,T$ vorkommen. Dann
gilt; $T \models C \sqsubseteq D$ gdw
$T^{'} \models A_{C} \sqsubseteq A_{D}$.

\subsubsection{ Normalform}\label{normalform}

Eine TBox ist in \emph{Normalform}, wenn sie nur Inklusionen folgender
Form enthält:

$A_1 \sqcap \ldots \sqcap A_{n} \sqsubseteq A$
$A \sqsubseteq \exists r.A_1$ $\exists r.A \sqsubseteq A_1$

\subsubsection{Lemma 6.14}\label{lemma-6.14}

Jede EL-TBox $T$ kann in polynomieller Zeit in eine TBox $T^{'}$ in
Normalform gewandelt werden, so dass für alle Konzeptnamen $A,B$ in
$T$ gilt: $T \models A \sqsubseteq B$ gdw
$T \models A \sqsubseteq B$. Dann ist $T'$ \emph{konservative
Erweiterung} von $T$.

\subsubsection{Lemma 6.14}\label{lemma-6.14-1}

Jede EL-TBox $T$ kann durch linear viele Regelanwendungen in TBox in
Normalform transformiert werden, die konservative Erweiterung von $T$
ist.

Die Regeln fügen jeweils Zwischenkonzepte ein.

Beweisskizze: Grad der Abnormalität definieren und zeigen, dass

\begin{enumerate}
\def\labelenumi{\arabic{enumi}.}
\item
  Der Grad ist beschränkt durch $\left| T \right|$
\item
  Jede Regelanwendung verringert den Grad
\item
  TBoxen vom Grad $0$ sind in Normalform
\end{enumerate}

\subsubsection{Algorithmus}\label{algorithmus}

Wende Regeln erschöpfend an um alle Subsumptionen zu berechnen:

$\frac{\ }{A \sqsubseteq A}$ (Wenn $A$ in $T$ vorkommt)
$\frac{\ }{A \sqsubseteq \top}$ (Wenn $A$ in $T$ vorkommt)

$\frac{A \sqsubseteq A_1,\ldots,A \sqsubseteq A_{n},A_1 \sqcap \ldots \sqcap A_{n} \sqsubseteq B}{A \sqsubseteq B}$
$\frac{A \sqsubseteq \exists r.A_1,\ A_1 \sqsubseteq B_1,\exists r.B_1 \sqsubseteq B}{A \sqsubseteq B}$

\textbf{Wo gehört das x hin Vorlesung 17?}

Für eine EL-TBox $T$ sei $T*$ das Ergebnis erschöpfender
Regelanwendungen, die \emph{Saturierung}.

\subsubsection{Theorem 6.16 }\label{theorem-6.16}

Für alle Konzeptnamen $A,B$ in $T$ gilt:
$T \models A \sqsubseteq B$ gdw $A \sqsubseteq B \in T*$

\paragraph{Terminierung}\label{terminierung-1}

Beweisskizze. Jede Regelanwendung erzeugt eine neue Konzeptinklusion
$A \sqsubseteq B$, wobei $A,B$ Konzeptnamen aus $T$. Es gibt nur
endlich viele solcher Inklusionen.

\paragraph{Korrektheit}\label{korrektheit-1}

Beweisskizze. Sei $T = T_{0},\ \ldots,(T_{n} = T*)$ die durch
Regelanwendungen erzeugte Folge von TBoxen. Es genügt zu zeigen:
$T_{i} \models T_{i + 1}$. Zeige dies durch Vorbedingung der
Regelanwendung und der Semantik.

\paragraph{Vollständigkeit}\label{vollstuxe4ndigkeit-1}

\subparagraph{Kanonische
Interpretation}\label{kanonische-interpretation}

Die Kanonische Interpretation $I$ ist:

\begin{itemize}
\item
  $\Delta^{I} = \left\{ d_{A}\ |\ A\ \mathrm{\text{Konzeptname\ in\ T}}* \right\} \cup \left\{ d_{\top} \right\}$
\item
  $A^{I} = \left\{ d_{\text{B\ }} \middle| \ B \sqsubseteq A \in T* \right\}$
\item
  $r^{I} = \left\{ \left( d_{A},d_{B} \right)\  \middle| \ A \sqsubseteq A^{'} \in T*\ \mathrm{\text{und}}\ A^{'} \sqsubseteq \exists r.B \in T*,\ A^{'}\ \mathrm{\text{Konzeptname}} \right\}$
\end{itemize}

Erklärung: Konstruiere Intuitiv Modell aus allen Konzeptnamen.

\hypertarget{lemma-6.18}{\subparagraph{Lemma 6.18}\label{lemma-6.18}}

Die kanonische Interpretation ist ein Model von $T*$.

Beweis: Zeige, dass alle Inklusionen in $T*$ von $I$ erfüllt werden.
Verwende die Inklusionen in Normalform. Überlege dazu, was aus den
Inklusionen mithilfe der Definition gefolgert werden kann.

\subparagraph{Vollständigkeit}\label{vollstuxe4ndigkeit-2}

Angenommen $A \sqsubseteq B \notin T*$. Betrachte Element $d_{A}$
der kanonischen Interpretation $I$. Wegen R1:
$A \sqsubseteq A \in T*$, also nach Def. $I$: $d_{A} \in A^{I}$.
Def. von $I$ und $A \sqsubseteq B \notin T*$ liefern
$d_{A} \notin B$. Da $I$ Modell von $T*$
(\protect\hyperlink{lemma-6.18}{Lemma 6.18}), und damit von $T$, folgt
nicht $T \models A \sqsubseteq B$.

\subsection{Erweiterungen von EL}\label{erweiterungen-von-el}

\hypertarget{el-mit-disjunktion-und-bottom}{\subsubsection{EL mit
Disjunktion und Bottom}\label{el-mit-disjunktion-und-bottom}}

Erfüllbarkeit in $\text{ELU}_{\bot}$ (mit Disjunktion) ist
ExpTime-Vollständig.

Beweisskizze per Reduktion von Erfüllbarkeit von Konzeptname $A$ bzgl.
ALC-TBox $T$:

\begin{enumerate}
\def\labelenumi{\arabic{enumi}.}
\item
  Ersetze Werterestriktion in $T$ durch Existenzrestriktion.
\item
  Bringe $T$ in Negationsnormalform.
\item
  Ersetze $\neg X$ durch $\overset{\overline{}}{X}$ mit
  $\top \sqsubseteq X \sqcup \overset{\overline{}}{X}$ und
  $X \sqcap \overset{\overline{}}{X} \sqsubseteq \bot$
\end{enumerate}

\subsubsection{ELU (mit Disjunktion)}\label{elu-mit-disjunktion}

Erfüllbarkeit in ELU ist ExpTime-Vollständig.

Beweisskizze per Reduktion von
\protect\hyperlink{el-mit-disjunktion-und-bottom}{6.5.1}. Ersetzte
$\bot$ durch $L$ mit $\exists r.L \sqsubseteq L$ für alle
Rollennamen $r$ in $T$.
