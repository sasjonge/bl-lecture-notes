
\subsection{EL}\label{el}

Ein EL-Konzept ist ein ALC-Konzept, in dem nur die Konstruktoren
$\top,\ $ $\sqcap$ und $\exists r.C$ verwendet werden.

\subsection{Simulation}\label{simulation}

Simulation ist gerichtete Bisimulation.
$\left( I_1,d_1 \right) \precsim \left( I_2,d_2 \right)$: es
gibt Simulation $p$ von $I_1$ nach $I_2$ mit
$d_1\text{\ p\ }d_2$.

\hypertarget{lemma-6.3}{\subsubsection{Lemma 6.3}\label{lemma-6.3}}

Seien $I_1,I_2$ Interpretationen, $d_1 \in \Delta^{I_1}$ und
$d_2 \in \Delta^{I_2}$. Wenn
$\left( I_1,d_1 \right) \precsim \left( I_2,d_2 \right)$, dann
gilt für alle EL-Konzepte $C$: $d_1 \in C^{I_1}$ impliziert
$d_2 \in C^{I_2}$.

Beweisskizze per Induktion über die Struktur von $C$. Sei $\rho$
eine Simulation zwischen $I_1$ und $I_2$ mit
$d_1\text{\ $\rho$}\text{\ d}_2$.

\textbf{I.A.} $C = A$ ist Konzeptname. Nach Bedingung 1. der
Bisimulation gilt $d_1 \in A^{I_1}$ impliziert
$d_2 \in A^{I_2}$.

\textbf{I.S.} Unterscheide Fälle gemäß dem äußersten Konstrukt von
$C$.

\begin{enumerate}
\def\labelenumi{\arabic{enumi}.}
\item
  $C = D_1 \sqcap D_2$
\end{enumerate}

\begin{quote}
$d_1 \in C^{I_1}$ gdw. $d_1 \in D_1^{I_1}$und
$d_1 \in D_2^{I_1}$ (Semantik) impliziert.
$d_2 \in D_1^{I_2}$ und $d_2 \in D_2^{I_2}$ (I.V.) gdw.
$d_2 \in C^{I_2}$ (Semantik)
\end{quote}

\begin{enumerate}
\def\labelenumi{\arabic{enumi}.}
\item
  $C = \exists r.D$
\end{enumerate}

\begin{quote}
Hinrichtung und Rückrichtung analog über Semantik, 2. Bedingung der
Simulation, I.V., Semantik.
\end{quote}

\subsubsection{Lemma 6.4}\label{lemma-6.4}

Bisimulation und wechselseitige Simulation sind nicht dasselbe.

Beweisskizze: Zeige Wechselseitige Simulation, die keine Bisimulation
ist:

\includegraphics[width=3.71910in,height=1.83200in]{media/image1.png}

\subsubsection{Lemma 6.6}\label{lemma-6.6}

Jedes EL-Konzept ist erfüllbar bzgl. jeder TBox.

Beweisskizze per Induktion über die Struktur von $C$:

\includegraphics[width=2.50460in,height=1.15200in]{media/image2.png}

\subsection{Subsumtion ohne TBox}\label{subsumtion-ohne-tbox}

Eine Subsumption $C \sqsubseteq D$ gilt in EL im Prinzip gdw. man
$D$ syntaktisch in $C$ „wiederfindet``. Werden Konzepte als
Baummodell dargestellt entspricht „Wiederfinden`` Simulation von
$D$-Baum in $C$-Baum (also Teilgraphenproblem).

\subsubsection{Definition kanonisches
Modell}\label{definition-kanonisches-modell}

Baue aus dem gegebenen Konzept $C$ intuitiv Baummodell.

\hypertarget{lemma-6.8}{\subsubsection{Lemma 6.8}\label{lemma-6.8}}

Für alle EL-Konzepte $C$ gilt: Die Interpretation $I_{C}$ ist Modell
von $C$ mit $d_{w} \in C^{I_{C}}$.

Beweisskizze per Induktion über die Struktur von $C$.

\hypertarget{lemma-6.9}{\subsubsection{Lemma 6.9}\label{lemma-6.9}}

Für alle EL-Konzepte $C$, Interpretation $I$ und
$e \in \Delta^{I}$ gilt: $e \in C^{I}$ gdw.
$\left( I_{C},d_{w} \right) \precsim \left( I,e \right)$.

Beweisskizze.
\subsection{\texorpdfstring{$\EL$}{EL}}\label{sec:el}

Für manche Anwendungen ist \ALC zu komplex:
\begin{itemize}
    \item Auch hoch-optimierte Reasoner können sehr große und komplexe Ontologien oft nicht verarbeiten (oder nur nach intensivem Tuning)
    \item In der Anfragebeantwortung muss man oft mit sehr großen Datenmengen umgehen und braucht schnelle Antworten (\autoref{sec:abox})
\end{itemize}

Wir betrachten die Beschreibungslogik $\EL$:
\begin{itemize}
    \item viel weniger ausdrucksstark als \ALC, Basisoperatoren nur $\sqcap$ und $\exists r.C$
    \item Erfüllbarkeit und Subsumtion in Polyzeit entscheidbar.
\end{itemize}

\begin{definition}[$\EL$]
Ein $\EL$-Konzept ist ein \ALC-Konzept, in dem nur die Konstruktoren
$\top, \sqcap$ und $\exists r.C$ verwendet werden.
\end{definition}

$\EL$ ist beliebt für biomedizinische Ontologien (die sind oft sehr groß und mit hohem Abstraktionsgrad):
\begin{align*}
    \text{Perikardium} &\sqsubseteq \text{Gewebe} \sqcap \exists \text{teilVon}.\text{Herz}\\
    \text{Perikarditis} &\equiv \text{Entzündung} \sqcap \exists \text{ort}.\text{Perikardium}\\
    \text{Entzündung} & \sqsubseteq \text{Krankheit} \sqcap \exists \text{wirktAuf}.\text{Gewebe}
\end{align*}

SNOMED CT ist in unwesentlicher Erweiterung von $\EL$ formuliert. $\EL$ ist Grundlage des EL-Profils von OWL 2.

\subsubsection{Simulation}\label{simulation}

Intuitiv: $\EL$ ist die \enquote{Hälfte von \ALC}; Simulation entspricht der \enquote{Hälfte von Bisimulation}.

\begin{definition}[Simulation]
    Seien $\MI_1$ und $\MI_2$ Interpretationen. Relation $\rho \subset \Delta^{\MI_1} \times \Delta^{\MI_2}$ ist Simulation von $\MI_1$ nach $\MI_2$, wenn gilt:
    \begin{enumerate}
        \item Wenn $d_1 \rho d_2$, dann gilt für alle Konzeptnamen $A$:
            \begin{center}
                $d_1 \in A^{\MI_1}$ impliziert $d_2 \in A^{\MI_2}$
            \end{center}
        \item Wenn $d_1 \rho d_2$ und $(d_1, d'_1) \in r^{\MI_1}$ für beliebigen Rollennamen $r$, dann gibt es ein $d'_2 \in \Delta^{\MI_2}$ mit $d'_1 \rho d'_2$ und $(d_2, d'_2) \in r^{\MI_2}$.
    \end{enumerate}
\end{definition}

\begin{tafel}[Beispiele für Simulationen]\mbox{}
    \begin{center}
        \begin{tikzpicture}
        \begin{scope}[every node/.style={circle,draw,thick}]
            \node (s0) at (0, 1) {};
            \node (s1) at (0, 0) {};
            \node (s2) at (-1, -1) {};
            \node (s3) at (1, -1) {};
            \node (s4) at (4, 1) {};
            \node (s5) at (3, -0) {};
            \node (s6) at (5, -0) {};
            \node (s7) at (3, -1) {};
        \end{scope}
        \begin{scope}[every node/.style={fill=white}]
            \node (name) at (0, 2) {$\MI_1$};
            \node (name) at (4, 2) {$\MI_2$};
            \node [left = 0.3cm of s0] at (s0){$A$};
            \node [left = 0.3cm of s2] at (s2){$A$};
            \node [left = 0.3cm of s3] at (s3){$B$};
            \node [right = 0.3cm of s4] at (s4){$A$};
            \node [right = 0.3cm of s6] at (s4){$A, B$};
            \node [right = 0.3cm of s7] at (s7){$A, B, C$};
            \path [->] (s0) edge node {\scriptsize{r}} (s1);
            \path [->] (s1) edge node {\scriptsize{r}} (s2);
            \path [->] (s1) edge node {\scriptsize{r}} (s3);
            \path [->] (s4) edge node {\scriptsize{r}} (s5);
            \path [->] (s4) edge node {\scriptsize{s}} (s6);
            \path [->] (s5) edge node {\scriptsize{r}} (s7);
            \path[dotted] [->] (s0) edge node {\scriptsize{$\rho$}}(s4);
            \path[dotted] [->] (s1) edge node {\scriptsize{$\rho$}}(s5);
            \path[dotted] [->] (s2) edge [bend right] node {\scriptsize{$\rho$}}(s7);
            \path[dotted] [->] (s3) edge node {\scriptsize{$\rho$}}(s7);
        \end{scope}
    \end{tikzpicture}
\end{center}
\begin{center}
        \begin{tikzpicture}
        \begin{scope}[every node/.style={circle,draw,thick}]
            \node (s0) at (1, 0) {};
            \node (s1) at (0, 1) {};
            \node (s2) at (-1, 0) {};
            \node (s3) at (0, -1) {};
            \node (s4) at (4, 0) {};
        \end{scope}
        \begin{scope}[every node/.style={fill=white}]
            \node (name) at (0, 2) {$\MI_1$};
            \node (name) at (4, 2) {$\MI_2$};
            \node [left = 0.3cm of s0] at (s0){};
            \node [left = 0.3cm of s1] at (s1){$A$};
            \node [left = 0.3cm of s2] at (s2){$A, B$};
            \node [left = 0.3cm of s3] at (s3){$B$};
            \node [right = 0.3cm of s4] at (s4){$A, B$};
            \path [->] (s0) edge node {\scriptsize{s}} (s1);
            \path [->] (s1) edge node {\scriptsize{r}} (s2);
            \path [->] (s2) edge node {\scriptsize{s}} (s3);
            \path [->] (s3) edge node {\scriptsize{r}} (s0);
            \path [->] (s4) edge [looseness=10] node [loop above] {\scriptsize{r, s}} (s4);
            \path[dotted] [->] (s0) edge node {\scriptsize{$\rho$}}(s4);
            \path[dotted] [->] (s1) edge node {\scriptsize{$\rho$}}(s4);
            \path[dotted] [->] (s2) edge [bend right] node {\scriptsize{$\rho$}}(s4);
            \path[dotted] [->] (s3) edge node {\scriptsize{$\rho$}}(s4);
        \end{scope}
    \end{tikzpicture}
    \end{center}
    Wenn in $\MI_1$ nur $A, B, r, s$ verwendet werden, dann gilt für jede Interpretation $\MI$ und jedes Element $d \in \Delta^\MI: (\MI, d) \precsim (\MI, e)$.
\end{tafel}

Beachte: im Gegensatz zu Bisimulation sind Simulationen gerichtet.

Seien $\MI_1$ und $\MI_2$ Interpretationen, $d_1 \in \Delta^{\MI_1}, d_2 \in  \Delta^{\MI_2}$. Wir schreiben  $(\MI_1,d_1) \precsim (\MI_2,d_2)$, wenn es eine Simulation $\rho$ von $\MI_1$ nach $\MI_2$ gibt mit $d_1 \rho d_2$ (wir sagen: $d_1$ wird simuliert von $d_2$).

\begin{theorem}\label{thm:63}
Seien $\MI_1,\MI_2$ Interpretationen, $d_1 \in \Delta^{\MI_1}$ und
$d_2 \in \Delta^{\MI_2}$. Wenn
$(\MI_1,d_1) \precsim (\MI_2,d_2)$, dann
gilt für alle $\EL$-Konzepte $C$:
\begin{center}
$d_1 \in C^{\MI_1}$ impliziert $d_2 \in C^{\MI_2}$.
\end{center}
\end{theorem}
\begin{tafel}[Beweis des Theorems 6.3]
    \begin{proof}
    per Induktion über $C$:
    \begin{itemize}
        \item $C=\top$: trivial
        \item $C=A$: nach Bedingung 1 einer Simulation
        \item $C=D\sqcap E$: nach IV und Semantik
        \item $C=\exists r.D$: nach Bedingung 2 einer Simulation und IV
    \end{itemize}
\end{proof}
\end{tafel}

Intuitiv: Wenn $(\MI_1, d_1) \precsim (\MI_2, d_2)$ und $(\MI_2, d_2) \precsim (\MI_1, d_1)$, dann kann $\EL$ nicht zwischen $d_1$ und $d_2$ \enquote{unterscheiden}.

Achtung: Bisimulation und wechselseitige Simulation sind \emph{nicht dasselbe}
\begin{lemma}
    Es gibt $(\MI_1, d_1)$ und $(\MI_2, d_2)$, so dass
    \begin{itemize}
        \item $(\MI_1, d_1) \precsim (\MI_2, d_2)$ und $(\MI_2, d_2) \precsim (\MI_1, d_1)$
        \item $(\MI_1, d_1) \not\sim (\MI_2, d_2)$
    \end{itemize}
\end{lemma}

\begin{tafel}[Beispiel Lemma 6.4]\label{t:63}\mbox{}
\begin{center}
        \begin{tikzpicture}
        \begin{scope}[every node/.style={circle,draw,thick}]
            \node (s0) at (0, 2) {$d$};
            \node (s1) at (0, 0) {};
            \node (s2) at (3, 2) {$e$};
            \node (s3) at (2, 0) {};
            \node (s4) at (4, 0) {};
        \end{scope}
        \begin{scope}[every node/.style={fill=white}]
            \node (name) at (0, 3) {$\MI_1$};
            \node (name) at (3, 3) {$\MI_2$};
            \node [below = 0.3cm of s1] at (s1){$A$};
            \node [below = 0.3cm of s3] at (s3){$A$};
            \path[dotted] [->] (s2) edge [bend right] (s0);
            \path[dotted] [->] (s3) edge [bend right] (s1);
            \path[dotted] [->] (s4) edge [bend right] node {$*$}(s1);
            \path[dashed] [->] (s0) edge (s2);
            \path[dashed] [->] (s1) edge (s3);
            \path [->] (s0) edge node {$r$} (s1);
            \path [->] (s2) edge node {$r$} (s3);
            \path [->] (s2) edge node {$r$} (s4);
        \end{scope}
    \end{tikzpicture}
    \end{center}

    Wechselseitige Simulation ist gegeben, aber $(\MI_1, d) \not\sim (\MI_2, e)$. Wenn es eine Bisimulation gäbe, müsste sie das Paar $*$ enthalten, was ein Widerspruch zu der 1. Bedingung von Bisimulationen ist.
\end{tafel}

Man kann nun wieder Nicht-Asudrückbarkeitsresultate zeigen:
\begin{lemma}
    Das \ALC-Konzept $\forall r.A$ ist \emph{nicht} in $\EL$ ausdrückbar.
\end{lemma}
\begin{tafel}[continues=t:63]
    \begin{proof}
        Angenommen es gäbe ein $\EL$-Konzept $C$ mit $C \equiv \forall r.A$.
        Dann gilt $d \in C^{\MI_1}$. Da $(\MI_1, d) \precsim (\MI_2, e)$,
        folgt $e \in C^{\MI_2}$. Widerspruch zu $e \not\in {(\forall
        r.A)}^{\MI_2}$.
    \end{proof}
    In $\EL\mathcal{I}$ ist $A \sqsubseteq \forall r.B$ ausdrückbar durch $\exists r^- A \sqsubseteq B$
\end{tafel}

In $\EL$ ist Erfüllbarkeit \emph{kein} interessantes Schlussfolgerungsproblem:

\begin{lemma}
    Jedes $\EL$-Konzept ist erfüllbar bezüglich jeder TBox.
\end{lemma}

\begin{tafel}\mbox{}
    \begin{proof}
    Betrachte folgende Interpretation $\MI$:

\begin{center}
        \begin{tikzpicture}
        \begin{scope}[every node/.style={circle,draw,thick}]
            \node (s0) at (0, 0) {$d$};
        \end{scope}
        \begin{scope}[every node/.style={fill=white}]
            \node [below = 0.4cm of s0] at (s0){$A_1, A_2, \ldots$};
            \path [->] (s0) edge [loop right] node {$r_1, r_2, \ldots$} (s0);
        \end{scope}
    \end{tikzpicture}
    \end{center}

Für jedes $\EL$-Konzept $C$ gilt $C^\MI = \{d\}$. Beweis per Induktion über
die Struktur von $C$.

Also ist $\MI$ Modell jedes $\EL$-Konzepts und damit jeder $\EL$-TBox.
\end{proof}
\end{tafel}

Daher konzentrieren wir uns auf Subsumtion.

\subsection{Subsumtion ohne TBox}\label{subsumtion-ohne-tbox}

Eine Subsumption $C \sqsubseteq D$ gilt in $\EL$ im Prinzip genau dann, wenn
man $D$ syntaktisch in $C$ \enquote{wiederfindet}. Werden Konzepte als
Bäume dargestellt entspricht \enquote{Wiederfinden} Simulation von
$D$-Baum in $C$-Baum (also äquivalent zu einem Teilgraphenproblem).

\begin{definition}[Kanonisches Modell - TODO]
    Wir ordnen induktiv jedem $\EL$-Konzept $C$ eine Interpretation $\MI_C$ zu:

    Wir nennen $\MI_C$ kanonisches Modell, bezeichnen Wurzel stets mit $d_w$.
\end{definition}

Man sieht leicht, dass $\MI_C$ wirklich ein Modell ist:

\begin{lemma}\label{lem:68}
    Für alle $\EL$-Konzepte $C$ gilt: Die Interpretation $\MI_C$ ist Modell von $C$ mit $d_w \in C^{\MI_C}$.
\end{lemma}

\begin{tafel}[Beweis von Lemma 6.8]
    \begin{proof} per Induktion über $C$.
        \begin{itemize}
            \item $C = \top$: trivial $\top^{\MI_C} = \Delta^{\MI_C}$
            \item $C = A$:, dann $I_C = {(d_w)}^A$, also $d_w \in A^{\MI_C}$
            \item $C = D \sqcap E$: dann $I_C = TODO$. Nach IV $d_w \in D^{\MI_d}$ und $d_w \in D^{\MI_e}$. Da die Identität eine Simulationen von $\MI_D$ nach $\MI_C$ bzw. von $\MI_E$ nach $\MI_C$ ist, gilt $d_w \in D^{\MI_C}$ und $d_w \in E^{\MI_C}$, also $d_w \in C^{\MI_C}$ nach Semantik.
            \item $C = \exists r.D$: dann $I_C = TODO$. IV liefert $d \in D^{\MI_D}$, also folgt $d \in D^{\MI_C}$, also $d_w \in {(\exists r.D)}^{\MI_C}$
        \end{itemize}
    \end{proof}
\end{tafel}

Die zentrale Eigenschaft kanonischer Modelle:
\begin{lemma}\label{lem:69}
Für alle $\EL$-Konzepte $C$, Interpretation $\MI$ und
$e \in \Delta^\MI$ gilt:
\begin{center}
$e \in C^\MI$ gdw.
$\left( \MI_C,d_w \right) \precsim \left( I,e \right)$.
\end{center}
\end{lemma}

\begin{tafel}[Beiweis Lemma 6.9]
    Beispiel: $C = A \sqcap \exists r.(\exists r.\top \sqcap \exists s.B) \sqcap \exists s.A$.
\begin{center}
        \begin{tikzpicture}
        \begin{scope}[every node/.style={circle,draw,thick}]
            \node (s0) at (0, 4) {$d_w$};
            \node (s1) at (-1, 2) {};
            \node (s2) at (-2, 0) {};
            \node (s3) at (0, 0) {};
            \node (s4) at (1, 2) {};
            \node (s5) at (4, 4) {$d$};
            \node (s6) at (6, 4) {$f$};
            \node (s7) at (4, 2) {$e$};
        \end{scope}
        \begin{scope}[every node/.style={fill=white}]
            \node [right = 0.3cm of s4] at (s4){$A$};
            \node [right = 0.3cm of s3] at (s3){$B$};
            \node [above = 0.4cm of s5] at (s5){$A$};
            \node [above = 0.4cm of s6] at (s6){$A,B$};
            \path [->] (s0) edge node {$r$} (s1);
            \path [->] (s1) edge node {$r$} (s2);
            \path [->] (s1) edge node {$s$} (s3);
            \path [->] (s0) edge node {$s$} (s4);
            \path [->] (s5) edge node {$s$} (s6);
            \path [->] (s5) edge node {$r$} (s7);
            \path [->] (s6) edge [bend left] node {$r$} (s7);
            \path [->] (s7) edge [loop below] node {$r$} (s7);
            \path [->] (s7) edge node {$s$} (s6);
            \path[dashed] [->] (s0) edge [bend right] (s5);
            \path[dashed] [->] (s1) edge [bend right] (s7);
            \path[dashed] [->] (s2) edge [bend right] (s7);
            \path[dashed] [->] (s3) edge [bend left] (s6);
            \path[dashed] [->] (s4) edge [bend left] (s6);
        \end{scope}
    \end{tikzpicture}
    \end{center}

    \begin{proof}\mbox{}
        \begin{itemize}
            \item \enquote{$\Leftarrow$} Angenommen $(\MI_C, d_w) \precsim (\MI, e)$. \autoref{lem:68} liefert $d_w \in C^{\MI_C}$ und \autoref{thm:63} $e \in C^\MI$.
            \item \enquote{$\Rightarrow$} Per Induktion über $C$:
                \begin{itemize}
                    \item $C = \top$: $\MI_C = d_w, \rho = \{(d_w, e)\}$ ist Simulation von $\MI_C$ nach $\MI$
                    \item $C = A$: $\MI_C = d_w$ analog.
                    \item $C = D \sqcap E$: $\MI_C$ wie in T6.5. IV liefert Simulation für $(\MI_D, d_w) \precsim (\MI, e)$ und $(\MI_E, d_w) \precsim (\MI, e)$. Leicht zu prüfen: $\rho_D \cup \rho_E$ ist Simulation.
                    \item $C = \exists r.D$: Semantik liefert $e' \in D^\MI, (e, e') \in r^\MI$. $\MI_C$ siehe T6.5. IV liefert Simulation $\rho$ für $(\MI_D, d) \precsim (\MI, e')$. $\rho \cup \{(d_w, e)\}$ ist Simulation zeigt $(\MI_C, d_w) \precsim (\MI, e)$.
                \end{itemize}
        \end{itemize}

    \end{proof}
\end{tafel}

Wir zeigen nun, dass $C \sqsubseteq D$ gdw.\ man $\MI_D$ in $\MI_C$ \enquote{wiederfindet}:

\begin{lemma}
Für alle $\EL$-Konzepte $C$, $D$ gilt:
\begin{center}
$C \sqsubseteq D$ gdw.
$\left( \MI_D,d_w \right) \precsim \left( \MI_C,d_w \right)$
\end{center}
\end{lemma}

\begin{tafel}[Beweis Lemma 6.10]
    \begin{proof}\mbox{}
\begin{itemize}
    \item \enquote{$\Rightarrow$}
  Betrachte kanonisches Modell $I_{C}$ von $C$. Wegen
  \autoref{lem:68} gibt es $d_w \in C^{\MI_C}$. Mit $C \sqsubseteq D$
  folgt $d_{W} \in D^{\MI_C}$. Mit \autoref{lem:69}
  folgt $(\MI_D, d_w) \precsim (\MI_C, d_w)$.
\item \enquote{$\Leftarrow$}
  Angenommen
  $\left( \MI_D,d_w \right) \precsim \left( \MI_C,d_w \right)$.
  Betrachte beliebige Interpretation $\MI$ und $d \in C^\MI$. Zu
  zeigen: $d \in D^{I}$. Wegen $d \in C^\MI$ und
  \autoref{lem:69} gilt
  $\left( \MI_C,d_w \right) \precsim \left( \MI,d \right)$.
  Es gibt also Simulation $\rho_1$ von $\MI_D$ nach $\MI_C$ mit $\rho_1(d_w) = d_w$ und Simulation $\rho_2$ von $\MI_C$ nach $\MI$ mit $\rho_2(d_w)=d$. Die Komposition $\rho_1 \cdot \rho_2$ ist Simulation von $\MI_D$ nach $\MI$ mit $(\rho_1 \cdot \rho_2)(d_w) = d$. \autoref{lem:69} liefert $d \in D^\MI$.
\end{itemize}
\end{proof}
\end{tafel}

Folgendes Theorem formuliert den recht einfachen Algorithmus:
\begin{theorem}
    Subsumtion in $\EL$ kann in polynomieller Zeit entschieden werden:

\begin{itemize}
\item
  Konstruiere $\MI_{C}$ und $\MI_{D}$ in polynomieller Zeit.
\item
  Überprüfe in polynomieller Zeit, ob
  $\left( \MI_D,d_w \right) \precsim \left( \MI_C,d_w \right)$.
\end{itemize}
\end{theorem}

\begin{tafel}[Prozedur Theorem 6.11]
    Gegeben $\MI_C, \MI_D$ entscheide $(\MI_D, d_w) \precsim (\MI_C, d_w)$ durch Berechnen der maximalen Simulation.
    Berechne Folge $\rho_0, \rho_1, \ldots$ wie folgt:
    \begin{align*}
        \rho_0 &= \{(e, d) \in \Delta^{\MI_D} \times \Delta^{\MI_C} \mid e \in A^{\MI_D} \implies d \in A^{\MI_C} \text{für alle A}\}\\
        \rho_{i + 1} &= \rho_i \mid \{(e, d) \mid\ \text{es gibt Relation}\ r\ \text{und}\ (e, e')\in r^{\MI_D}\text{, so dass}\ (e', d') \not\in \rho_i\ \text{für alle}\ (d, d') \in r^{\MI_C}
    \end{align*}
    Folge stabilisiert nach $|C| \cdot |D|$ Schritten und liefert die maximale Simulation.
\end{tafel}

\subsection{Subsumption mit TBox}\label{subsumption-mit-tbox}
Wir verwenden ein so genanntes konsequenzbasiertes Verfahren.

Grundidee:
\begin{itemize}
    \item Mit Hilfe von Regeln werden zur TBox nach und nach neue Konzeptinklusionen hinzugefügt.
    \item Am Ende muss man dann nur nach nachschauen, ob die gewünschte Subsumtion in der TBox explizit enthalten ist.
\end{itemize}

In der Praxis haben sich derartige Verfahren als äußert effizient herausgestellt. Sie sind verwandt mit Sequenzkalkülen aus der klassischen Logik.

Wir betrachten o.B.d.A. nur Subsumtion von Konzeptnamen bezüglich TBoxen.

\begin{lemma}\label{lem:612}
    Seien $C, D$ zwei beliebige $\EL$-Konzepte und $\MT$ eine $\EL$-TBox. Sei weiterhin
$\MT' = \MT \cup \left\{ A_C \sqsubseteq C,\ D \sqsubseteq A_D \right\}$,
mit Konzeptnamen $A_C,A_D$, die nicht in $C,D,\MT$ vorkommen. Dann
gilt:
\begin{center}
$\MT \models C \sqsubseteq D$ gdw.\
$\MT' \models A_{C} \sqsubseteq A_{D}$.
\end{center}
\end{lemma}

\autoref{lem:612} liefert eine Polyzeit-Reduktion
\begin{itemize}
    \item von Subsumtion zwischen beliebigen Konzepten bezüglich TBoxen
    \item auf Subsumtion zwischen Konzeptnamen bezüglich TBoxen.
\end{itemize}

Wir betrachten o.B.d.A.\ nur TBoxen in folgender Normalform:

\begin{definition}[Normalform]
    Eine TBox ist in \emph{Normalform}(NF), wenn sie nur Inklusionen folgender
Form enthält: \begin{align*}
    A_1 \sqcap \ldots \sqcap A_n &\sqsubseteq A\\
A &\sqsubseteq \exists r.A_1\\
\exists r.A &\sqsubseteq A_1
\end{align*}
wobei $A, A_1, \ldots, A_n$ Konzeptnamen oder $\top$ sind.
\end{definition}

\begin{lemma}
    Jede $\EL$-TBox $\MT$ kann in polynomieller Zeit in eine TBox $\MT'$ in NF gewandelt werden, so dass für alle Konzeptnamen $A, B$ in $\MT$ gilt:
    \begin{align*}
        \MT \models A\sqsubseteq B\ \text{gdw.}\ \MT' \models A\sqsubseteq B
        \tag{$*$}
    \end{align*}
\end{lemma}
Wenn ($*$) gilt, sagen wir $\MT'$ ist eine konservative
Erweiterung von $\MT$. Um die NF herzustellen, wenden wir
Normalisierungsregeln an.

\begin{tafel}[Beispiel Normalisierung]
    \begin{align*}
        A \sqcap \exists r.B &\sqsubseteq A\\
        A \sqcap B &\sqsubseteq \exists r.A\\
                   &\Downarrow\\
        A \sqcap A_{\exists r.B} &\sqsubseteq A\\
        \exists r.B &\sqsubseteq A_{\exists r.B}\\
        A \sqcap B &\sqsubseteq A_{\exists r.A}\\
        A_{\exists r.A} &\sqsubseteq \exists r.A\\
    \end{align*}
\end{tafel}

\begin{align*}
    C_1 \sqcap \dots \sqcap C_n \sqsubseteq E
    &\rightsquigarrow
    C_i \sqsubseteq A_{C_i},\ C_1 \sqcap \dots \sqcap C_{i - 1} \sqcap A_{C_i} \sqcap C_{i + 1} \sqcap \dots \sqcap C_n \sqsubseteq E \tag{NF1}\\
    &\text{wenn}\ C_i\ \text{Existenzrestriktion}\\
    \exists r.C \sqsubseteq E
    &\rightsquigarrow
    C \sqsubseteq A_C,\ \exists r.A_C \sqsubseteq E \tag{NF2}\\
    C \sqsubseteq \exists r.D
    &\rightsquigarrow
    C \sqsubseteq A_C,\ A_C \sqsubseteq \exists r.D \tag{NF3}\\
    A \sqsubseteq \exists r.C
    &\rightsquigarrow
    A \sqsubseteq \exists r.A_C,\ A_C \sqsubseteq C \tag{NF4}\\
    A \sqsubseteq C_1 \sqcap C_2
    &\rightsquigarrow
    A \sqsubseteq C_1,\ A \sqsubseteq C_2 \tag{NF5}\\
    &\text{wenn}\ C\ \text{weder Konzeptname noch}\ \top
\end{align*}
$A$ ist Konzeptname, $C_i, D, E$ sind beliebige Konzepte; $A_{C_(i)}$ sind neue Konzeptnamen.

\begin{lemma}
Jede $\EL$-TBox $\MT$ kann durch linear viele Regelanwendungen in TBox in
Normalform transformiert werden, die konservative Erweiterung von $\MT$
ist.
\end{lemma}


Die Regeln fügen jeweils Zwischenkonzepte ein.

Beweisskizze: Grad der Abnormalität definieren und zeigen, dass

\begin{enumerate}
\def\labelenumi{\arabic{enumi}.}
\item
  Der Grad ist beschränkt durch $\left| T \right|$
\item
  Jede Regelanwendung verringert den Grad
\item
  TBoxen vom Grad $0$ sind in Normalform
\end{enumerate}

\subsubsection{Algorithmus}\label{algorithmus}

Wende Regeln erschöpfend an um alle Subsumptionen zu berechnen:

$\frac{\ }{A \sqsubseteq A}$ (Wenn $A$ in $T$ vorkommt)
$\frac{\ }{A \sqsubseteq \top}$ (Wenn $A$ in $T$ vorkommt)

$\frac{A \sqsubseteq A_1,\ldots,A \sqsubseteq A_{n},A_1 \sqcap \ldots \sqcap A_{n} \sqsubseteq B}{A \sqsubseteq B}$
$\frac{A \sqsubseteq \exists r.A_1,\ A_1 \sqsubseteq B_1,\exists r.B_1 \sqsubseteq B}{A \sqsubseteq B}$

\textbf{Wo gehört das x hin Vorlesung 17?}

Für eine EL-TBox $T$ sei $T*$ das Ergebnis erschöpfender
Regelanwendungen, die \emph{Saturierung}.

\subsubsection{Theorem 6.16 }\label{theorem-6.16}

Für alle Konzeptnamen $A,B$ in $T$ gilt:
$T \models A \sqsubseteq B$ gdw $A \sqsubseteq B \in T*$

\paragraph{Terminierung}\label{terminierung-1}

Beweisskizze. Jede Regelanwendung erzeugt eine neue Konzeptinklusion
$A \sqsubseteq B$, wobei $A,B$ Konzeptnamen aus $T$. Es gibt nur
endlich viele solcher Inklusionen.

\paragraph{Korrektheit}\label{korrektheit-1}

Beweisskizze. Sei $T = T_{0},\ \ldots,(T_{n} = T*)$ die durch
Regelanwendungen erzeugte Folge von TBoxen. Es genügt zu zeigen:
$T_{i} \models T_{i + 1}$. Zeige dies durch Vorbedingung der
Regelanwendung und der Semantik.

\paragraph{Vollständigkeit}\label{vollstuxe4ndigkeit-1}

\subparagraph{Kanonische
Interpretation}\label{kanonische-interpretation}

Die Kanonische Interpretation $I$ ist:

\begin{itemize}
\item
  $\Delta^{I} = \left\{ d_{A}\ |\ A\ \mathrm{\text{Konzeptname\ in\ T}}* \right\} \cup \left\{ d_{\top} \right\}$
\item
  $A^{I} = \left\{ d_{\text{B\ }} \middle| \ B \sqsubseteq A \in T* \right\}$
\item
  $r^{I} = \left\{ \left( d_{A},d_{B} \right)\  \middle| \ A \sqsubseteq A' \in T*\ \mathrm{\text{und}}\ A' \sqsubseteq \exists r.B \in T*,\ A'\ \mathrm{\text{Konzeptname}} \right\}$
\end{itemize}

Erklärung: Konstruiere Intuitiv Modell aus allen Konzeptnamen.

\hypertarget{lemma-6.18}{\subparagraph{Lemma 6.18}\label{lemma-6.18}}

Die kanonische Interpretation ist ein Model von $T*$.

Beweis: Zeige, dass alle Inklusionen in $T*$ von $I$ erfüllt werden.
Verwende die Inklusionen in Normalform. Überlege dazu, was aus den
Inklusionen mithilfe der Definition gefolgert werden kann.

\subparagraph{Vollständigkeit}\label{vollstuxe4ndigkeit-2}

Angenommen $A \sqsubseteq B \notin T*$. Betrachte Element $d_{A}$
der kanonischen Interpretation $I$. Wegen R1:
$A \sqsubseteq A \in T*$, also nach Def. $I$: $d_{A} \in A^{I}$.
Def. von $I$ und $A \sqsubseteq B \notin T*$ liefern
$d_{A} \notin B$. Da $I$ Modell von $T*$
(\protect\hyperlink{lemma-6.18}{Lemma 6.18}), und damit von $T$, folgt
nicht $T \models A \sqsubseteq B$.

\subsection{Erweiterungen von EL}\label{erweiterungen-von-el}

\hypertarget{el-mit-disjunktion-und-bottom}{\subsubsection{EL mit
Disjunktion und Bottom}\label{el-mit-disjunktion-und-bottom}}

Erfüllbarkeit in $\text{ELU}_{\bot}$ (mit Disjunktion) ist
ExpTime-Vollständig.

Beweisskizze per Reduktion von Erfüllbarkeit von Konzeptname $A$ bzgl.
ALC-TBox $T$:

\begin{enumerate}
\def\labelenumi{\arabic{enumi}.}
\item
  Ersetze Werterestriktion in $T$ durch Existenzrestriktion.
\item
  Bringe $T$ in Negationsnormalform.
\item
  Ersetze $\neg X$ durch $\overset{\overline{}}{X}$ mit
  $\top \sqsubseteq X \sqcup \overset{\overline{}}{X}$ und
  $X \sqcap \overset{\overline{}}{X} \sqsubseteq \bot$
\end{enumerate}

\subsubsection{ELU (mit Disjunktion)}\label{elu-mit-disjunktion}

Erfüllbarkeit in ELU ist ExpTime-Vollständig.

Beweisskizze per Reduktion von
\protect\hyperlink{el-mit-disjunktion-und-bottom}{6.5.1}. Ersetzte
$\bot$ durch $L$ mit $\exists r.L \sqsubseteq L$ für alle
Rollennamen $r$ in $T$.
